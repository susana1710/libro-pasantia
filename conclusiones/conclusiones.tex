\chapter{Conclusiones y Recomendaciones} \label{chap:conclusiones}

En este proyecto de pasantía, realizado para la empresa Digitalica Group C.A, se diseñó y desarrolló un prototipo funcional de una aplicación móvil para el registro de gastos e ingresos, así como un servidor web que se comunica con ella.  La aplicación fue desarrollada para el sistema operativo Android. Con ella, se permite registrar ingresos y gastos, a los cuales se puede asociar un monto, fecha, descripción, fotos y categoría. El servidor web permite hacer la autenticación de usuarios, así como el envío de reportes compuestos por un conjunto de gastos.

El uso de Scrum como marco de trabajo permitió desarrollar el proyecto progresivamente, y tener una parte del producto funcional luego de cada iteración. Esto fue de gran ventaja porque le facilitaba al dueño del producto (\textit(Product Owner)) evaluar constantemente las funcionalidades, verificar que cumplieran con las necesidades del producto y modificarlas de ser necesario. Además, las reuniones diarias (\textit{Daily Scrum}) le permitía a todo el equipo de trabajo estar informado de la evolución del producto, además de ser una oportunidad en que el equipo de desarrollo (en este caso integrado únicamente por el pasante) pudiera solventar cualquier duda o impedimento que surgiera.

A partir de la experiencia que se obtuvo durante el desarrollo del proyecto, complementado con el aprendizaje adquirido a lo largo de toda la carrera universitaria, se identificaron distintas funcionalidades que pudieran ser añadidas o mejoradas.

Actualmente, toda la información es guardada en el dispositivo móvil. El servidor tiene únicamente información de los reportes que son enviados por la aplicación móvil, lo cual es un subconjunto  muy limitado de todos los datos presentes en el dispositivo.

Por otra parte, una de las funcionalidades que tiene la aplicación es poder tomar fotos para cada ingreso/gasto. Estas fotos se hacen a través de la aplicación de cámara instalada por defecto en el dispositivo móvil, lo que implica que se está dejando en manos de una aplicación externa la escogencia de la resolución en que se tomarán las fotos. Esto significa que la foto pudiera estar tomándose con una calidad muy baja o que, por el contrario, la resolución sea demasiado alta y para guardarla en el dispositivo se estaría gastando espacio más espacio del necesario. Por esta razón, se recomienda a la empresa capturar las fotos internamente, sin hacer uso de otra aplicación. Para lograr esto, Android ofrece un API que permite construir un \textit{activity} para el uso personalizado de la cámara \cite{AND4}.

Como fue mencionado en capítulos anteriores, la generación de archivos PDF para los reportes se ofrece únicamente para versiones de Android iguales o mayores que 4.4. Dada la dificultad de encontrar librerías con licencia comercial gratuita para la generación de archivos PDF, que permitieran la creación de y personalización de los archivos, se decidió utilizar una librería nativa de Android, pero que solo está disponible para versiones a partir de la 4.4. Se consideró que limitar esta funcionalidad según la versión de Android podría ser una solución factible, pues el 78,5\% de los usuarios de Android tiene una vesión igual o mayor que 4.4 \cite{USG1}.

Durante el proceso de investigación, se encontraron librerías que permiten la generación de archivos PDF para todas las versiones de Android. Sin embargo, se debe pagar para obtener una licencia comercial. Si se piensa lanzar al mercado la aplicación, se recomienda a la empresa evaluar si dejar de ofrecer la funcionalidad a un porcentaje de usuarios generará mayores pérdidas que la adquisición de una licencia comercial de alguno de los productos que pueda solucionar el problema.

%Por otra parte, 