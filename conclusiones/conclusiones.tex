\chapter{Conclusiones y Recomendaciones} \label{chap:conclusiones}

En este proyecto de pasantía, se desarrolló una versión alfa de una aplicación móvil y un servidor web que permiten agilizar el proceso de reembolso de gastos de los trabajadores de la empresa Digitalica Group C.A. La aplicación fue desarrollada para el sistema operativo Android. Con ella, se permite registrar gastos, a los cuales se puede asociar un monto, fecha, descripción, fotos y categoría. El servidor web permite hacer la autenticación de usuarios, así como el envío de reportes compuestos por un conjunto de gastos. Igualmente, se desarrolló una aplicación web que permite a los supervisores de la empresa revisar, aprobar y rechazar reportes. También se implementaron funcionalidades dirigidas al público en general, y no únicamente a la empresa. Estas funcionalidades incluyen el manejo de ingresos, manejo de múltiples cuentas de gastos/ingresos y manejo de categorías de gastos/ingresos.

Se estudiaron conceptos que facilitaron el desarrollo de la solución propuesta al problema planteado. Utilizando los patrones de arquitectura MVC en el servidor y MVP en la aplicación móvil, se logró hacer una separación clara de las interfaces de usuario de los modelos de datos y controladores/presentadores. Con ello, se permite introducir cambios en algún componente sin afectar los demás.

%En cuanto a los requerimientos establecidos para la aplicación móvil y el servidor, fueron implementados en su totalidad.

El uso de Scrum como marco de trabajo permitió desarrollar el proyecto progresivamente, y tener una parte del producto funcional luego de cada iteración. Esto fue de gran ventaja porque le facilitaba al dueño del producto (\textit{Product Owner}) evaluar constantemente las funcionalidades y verificar que cumplieran con las necesidades del producto. Además, las reuniones diarias (\textit{Daily Scrum}) le permitía a todo el equipo de trabajo estar informado de la evolución del producto, además de ser una oportunidad en que el equipo de desarrollo (en este caso integrado únicamente por el pasante) pudiera solventar cualquier duda o impedimento que surgiera.

Durante el desarrollo del proyecto de pasantía, se identificaron funcionalidades que pudieran ser añadidas o mejoradas, que se explicarán en los siguientes párrafos.

Actualmente, el servidor tiene únicamente información de los reportes que son enviados por la aplicación móvil, lo cual es un subconjunto  muy limitado de todos los datos presentes en el dispositivo; toda la información es guardada localmente en el dispositivo móvil. Esto implica que si un usuario guarda su información en un dispositivo, y posteriormente inicia sesión en otro dispositivo, no podrá ver desde este último la información que había guardado anteriormente; para poder acceder a sus datos, debe iniciar sesión en el dispositivo desde donde los guardó. Se recomienda enviar al servidor todos los datos que almacena la aplicación móvil en los dispositivos, para permitir que un usuario pueda iniciar sesión y consultar sus datos en distintos dispositivos. De esta manera, al iniciar sesión en la aplicación móvil, se piden al servidor y se muestran los datos del usuario correspondiente. 

%Por otra parte, una de las funcionalidades que tiene la aplicación es tomar fotos para cada ingreso/gasto. Estas fotos se capturan a través de la aplicación de cámara instalada por defecto en el dispositivo móvil, lo que implica que se está delegando en una aplicación externa la elección de la resolución en que se tomarán las fotos. Esto significa que la foto pudiera estar tomándose con una calidad muy baja o que, por el contrario, la resolución sea demasiado alta y para guardarla en el dispositivo se estaría gastando más espacio del necesario. Por esta razón, se recomienda a la empresa capturar las fotos internamente, sin hacer uso de otra aplicación. Para lograr esto, Android ofrece un API que permite construir un \textit{activity} para el uso personalizado de la cámara \cite{AND4}.

Para la creación del archivo PDF se presentó una dificultad. Las librerías existentes en Java para generar archivos con dicha extensión hacen uso de otras librerías, que no están soportadas directamente en la plataforma de Android. Por esta razón, se decidió utilizar una librería nativa de Android que facilita la creación de archivos PDF; sin embargo, esta librería sólo puede ser utilizada por dispositivos que tengan una versión de Android a partir de la 4.4. Esta decisión se apoyó en el hecho de que más del 80\% de los usuarios de Android utilizan una versión mayor o igual a 4.4, como se puede ver en el cuadro~\ref{table:androidUsageTable} (datos obtenidos del sitio web oficial para desarrolladores de Android) \cite{USG1}. 

\begin{table}[ht]
\centering
\begin{tabular}{|>{\centering\arraybackslash}p{0.1\textwidth}|>{\centering\arraybackslash}p{0.15\textwidth}|>{\centering\arraybackslash}p{0.05\textwidth}|>{\centering\arraybackslash}p{0.1\textwidth}|}
\hline 
\bfseries {Versión} & \bfseries {Nombre} & \bfseries {API} & \bfseries {Uso}\\ 
\hline 
2.2 & Froyo & 8 & 0.1\% \\ 
\hline 
2.3.x & Gingergread & 15 & 1.5\% \\
 
\hline 
4.0.x & Ice Cream Sandwich & 15 & 1.4\% \\ 
\hline 
4.1.x & Jelly Bean & 16 & 5.6\% \\
4.2.x & & 17 & 7.7\% \\
4.3 & & 18 & 2.3\% \\
\hline 
4.4 & KitKat & 19 & 27.7\% \\ 
\hline 
5.0 & Lollipop & 21 & 13.1\% \\
5.1 & & 22 & 21.9\% \\ 
\hline 
6.0 & Marshmallow & 23 & 18.7\% \\ 
\hline
\end{tabular} 
\caption{Porcentaje de uso de las versiones de Android. No se muestran versiones con menos de 0.1\% de distribución}
\label{table:androidUsageTable}
\end{table}


%Como fue mencionado en capítulos anteriores, la generación de archivos PDF para los reportes se ofrece únicamente para versiones de Android iguales o mayores que 4.4. Dada la dificultad de encontrar librerías con licencia comercial gratuita para la generación de archivos PDF, que permitieran la creación de y personalización de los archivos, se decidió utilizar una librería nativa de Android, pero que solo está disponible para versiones a partir de la 4.4. Se consideró que limitar esta funcionalidad según la versión de Android podría ser una solución factible, pues el 78,5\% de los usuarios de Android tiene una versión igual o mayor que 4.4 \cite{USG1}.

Durante el proceso de investigación, se encontraron librerías que permiten la generación de archivos PDF para todas las versiones de Android. Sin embargo, se debe pagar para obtener una licencia comercial. Si se piensa lanzar al mercado la aplicación, se recomienda a la empresa evaluar si dejar de ofrecer la funcionalidad a un porcentaje de usuarios generará mayores pérdidas que la adquisición de una licencia comercial de alguna de las librerías que pueda solucionar el problema.

%Por otra parte, 