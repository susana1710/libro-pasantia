\chapter{Conclusiones y Recomendaciones} \label{chap:conclusiones}

En este proyecto de pasantía, realizado para la empresa Digitalica Group C.A, se diseñó y desarrolló un prototipo funcional de una aplicación móvil para el registro de gastos e ingresos, así como un servidor web que se comunica con ella.  La aplicación fue desarrollada para el sistema operativo Android. Con ella, se permite registrar ingresos y gastos, a los cuales se puede asociar un monto, fecha, descripción, fotos y categoría. El servidor web permite hacer la autenticación de usuarios, así como el envío de reportes compuestos por un conjunto de gastos.

El uso de Scrum como marco de trabajo permitió desarrollar el proyecto progresivamente, y tener una parte del producto funcional luego de cada iteración. Esto fue de gran ventaja porque le facilitaba al dueño del producto (\textit(Product Owner)) evaluar constantemente las funcionalidades, verificar que cumplieran con las necesidades del producto y modificarlas de ser necesario. Además, las reuniones diarias (\textit{Daily Scrum}) le permitía a todo el equipo de trabajo estar informado de la evolución del producto, además de ser una oportunidad en que el equipo de desarrollo (en este caso integrado únicamente por el pasante) pudiera solventar cualquier duda o impedimento que surgiera.


