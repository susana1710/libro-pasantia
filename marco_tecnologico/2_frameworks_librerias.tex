\section{\textit{Frameworks} y librerías} \label{Frameworks y librerias}

\subsection{AppFuse}
Es un \textit{framework} utilizado para la creación de aplicaciones web en la máquina virtual de Java (JVM por sus siglas en inglés). Dentro de AppFuse, se pueden integrar otras tecnologías como Bootstrap, Maven, Hibernate, Spring, Tapestry, etc \cite{APF1}. Se utilizó para crear la aplicación web mediante el uso de otros \textit{frameworks} (Hibernate, Spring y Tapestry).

\subsection{Hibernate}
Es una herramienta de mapeo objeto-relacional (ORM por sus siglas en inglés), que permite el mapeo de objetos de Java a bases de datos relacionales \cite{HBR1}. Se utilizó para proveer una abstracción sobre los elementos de la base de datos del servidor.

\subsection{Spring}
Es un \textit{framework} utilizado para el desarrollo de sistemas y aplicaciones basadas en la máquina virtual de Java (JVM por sus siglas en inglés). Provee funcionalidades para el desarrollo de servicios web basados en la arquitectura REST \cite{SPRNG0}. Se utilizó para la creación de los servicios web.

\subsection{Tapestry}
Es un \textit{framework} para el desarrollo de aplicaciones web en Java. Toma un enfoque modular al relacionar la interfaz de usuario con clases de Java. El uso de Tapestry para la construcción de aplicaciones implica la creación de archivos HTML, y clases de Java para cada uno de ellos. \cite{ATP1}. Se utilizó para la creación de las vistas y controladores de la aplicación web.

\subsection{Gson}
Es una librería de Java que permite la conversión de objetos a JSON y viceversa \cite{GSN1}. Se utilizó en la aplicación móvil para transformar objetos de Java a archivos con formato JSON para ser enviados al servidor, así como para transformar archivos con formato JSON a objetos de Java en el servidor.

\subsection{Retrofit}
Es una librería de Android que permite realizar peticiones utilizando el protocolo HTTP, mediante una interfaz de Java \cite{RFT1}. Se utilizó para realizar peticiones al servidor desde la aplicación móvil.

%\subsection{JUnit}
%Es una \textit{framework} que permite realizar pruebas unitarias de aplicaciones Java.