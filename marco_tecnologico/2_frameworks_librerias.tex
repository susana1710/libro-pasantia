\section{\textit{Frameworks} y librerías} \label{Frameworks y librerias}

\subsection{AppFuse}
Es un \textit{framework} utilizado para la creación de aplicaciones web en la máquina virtual de Java (JVM por sus siglas en inglés). Dentro de AppFuse, se pueden integrar otras tecnologías como Bootstrap, Maven, Hibernate, Spring, Tapestry, etc \cite{APF1}.

\subsection{Hibernate}
Es una herramienta de mapeo objeto-relacional (ORM por sus siglas en inglés) para Java que permite el mapeo de objetos de Java a bases de datos relacionales \cite{HBR1}.

\subsection{Spring}
Es un \textit{framework} para el desarrollo de aplicaciones en Java.

\subsection{Tapestry}
Es un \textit{framework} para el desarrollo de aplicaciones web en Java, Groovy o Scala.  \cite{ATP1}.

\subsection{Gson}
Es una librería de Java que permite la conversión de objetos a JSON y viceversa \cite{GSN1}
.
\subsection{Retrofit}
Es una librería de Android que permite realizar peticiones HTTP mediante una interfaz de Java\cite{RFT1}.