\section{Herramientas, entornos y lenguajes} \label{Herramientas, entornos y lenguajes}


\subsection{Android}
Es un sistema operativo de código abiertos basado en el sistema operativo Linux, utilizado principalmente para dispositivos móviles\cite{AND1}. Es la plataforma para la cual se desarrolló la aplicación.
s
\subsection{Java}
Es un lenguaje de programación de alto nivel orientado a objetos, que puede correr virtualmente en cualquier computadora \cite{JAV1}. Es el lenguaje en el que están desarrolladas la mayoría de las aplicaciones de Android \cite{AND2}. Se utilizó tanto para la programación de la aplicación móvil como del servidor.

\subsection{MySQL} 
Es un sistema de manejo de base de datos relacional basado en el Lenguaje de Consulta Estructurado (SQL por sus siglas en inglés) \cite{SQL1}. Se utilizó para la gestión de la base de datos del servidor.

\subsection{SQLite}
Es un sistema de manejo de base de datos relacional basado en el Lenguaje de Consulta Estructurado (SQL por sus siglas en inglés) que forma parte del programa principal, en lugar de ser un proceso independiente como ocurre con los sitemas de gestión de base de datos cliente-servidor \cite{SQL2}.

\subsection{Maven}
Es una herramienta que se permite compilar y  manejar proyectos de \textit{software} basados en Java. Se encarga de conectar las dependencias de los paquetes \cite{MVN1}.

\subsection{Jetty}
Es un servidor HTTP que puede funcionar independientemente por sus propios medios, o que puede correr dentro de otra aplicación \cite{JTY1}.

\subsection{Android Studio}
Es el entorno de desarrollo integrado (IDE por su siglas en inglés) oficial para el desarrollo de aplicaciones nativas para Android \cite{ASD1}.

\subsection{Eclipse}
Es un entorno desarrollo integrado (IDE por sus siglas en inglés) utilizado principalmente para desarrollar aplicaciones en Java \cite{ECL1}.

\subsection{Git}
Es un sistema de control de versiones distribuido (DVCS por sus siglas en inglés). Es una herramienta que permite gestionar las distintas versiones de un proyecto de \textit{software} \cite{GIT1}.
