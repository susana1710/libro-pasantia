\section{Herramientas, entornos y lenguajes} \label{Herramientas, entornos y lenguajes}


\subsection{Android}
Es un sistema operativo de código abierto basado en el sistema operativo Linux, utilizado principalmente para dispositivos móviles\cite{AND1}. Es la plataforma para la cual se desarrolló la aplicación. 

La plataforma de Android está basada en una arquitectura que tiene diferentes capas que van desde servicios de bajo nivel del sistema operativo, que interactúan directamente con el \textit{hardware}, hasta las aplicaciones. Además, Android provee un kit de desarrollo de \textit{software} (SDK por sus siglas en inglés), que permite crear aplicaciones \cite{AND3}.

\begin{figure}[ht]
  \centering
  \includegraphics[scale=0.6,type=png,ext=.png,read=.png]{imagenes/android_architecture}
  \caption{Arquitectura de la plataforma de Android}
  \label{fig:arquitecturaAndroid}
\end{figure}

La capa de más bajo nivel es la del kernel de Linux, como se puede observar en la figura~\ref{fig:arquitecturaAndroid}. La plataforma de Android está basada en el kernel de Linux. En ella se proveen servicios del sistema operativo, como el manejo de memoria y procesos \cite{AND3}.

La siguiente capa es la de abstracción de \textit{harwdare} (HAL por sus siglas en inglés). En ella se provee un nivel de abstracción para la comunicación con el \textit{hardware} del dispositivo, y contiene los \textit{drivers} esenciales para el funcionamiento de dicho \textit{hardware}. El HAL está compuesto por múltiples módulos de librerías, cada uno de los cuales implementa una interfaz para un tipo específico de componente de \textit{hardware}. Por ejemplo, se tienen módulos para la cámara y para el \textit{bluetooth}.

Encima de esta capa se encuentra la de librerías nativas, dentro de las cuales se encuentra SQLite, que permite guardar datos de las aplicaciones \cite{AND3}.

En conjunto con la librerías, está la sección de Android \textit{Runtime} en el tercer nivel. Dentro de ella se encuentra un componente llamado ART (\textit{Android Runtime}), que es el entorno de ejecución de las aplicaciones\cite{AND3}.

La cuarta capa de abajo hacia arriba corresponde al \textit{framework} de las aplicaciones (\textit{Application Framework}). En ella se proveen servicios de alto nivel mediante APIs escritos en Java. Dentro de estos servicios destaca el Manejador de Actividades (\textit{Activity Manager}), que se encarga de controlar todas las actividades de una aplicación\cite{AND3}. Una actividad (\textit{activity}) es un componente que provee una pantalla con la cual el usuario final puede interactuar. Generalmente, a cada actividad corresponde una interfaz de usuario \cite{AND6}. El desarrollo de aplicaciones para Android se realiza dentro de esta capa, pues en ella se proveen los servicios necesarios \cite{AND5}.

Por último, se encuentra la capa de aplicaciones (\textit{Android Application}), que es donde se encuentran instaladas las aplicaciones del dispositivo\cite{AND3}.

\subsection{Java}
Es un lenguaje de programación de alto nivel orientado a objetos, que puede correr virtualmente en cualquier computadora \cite{JAV1}. Es el lenguaje en el que están desarrolladas la mayoría de las aplicaciones de Android \cite{AND2}. Se utilizó tanto para la programación de la aplicación móvil como del servidor.

\subsection{MySQL} 
Es un sistema de manejo de base de datos relacional basado en el Lenguaje de Consulta Estructurado (SQL por sus siglas en inglés) \cite{SQL1}. Se utilizó para la gestión de la base de datos del servidor.

\subsection{SQLite}
Es una librería que implementa un motor de base de datos relacional, basado en el Lenguaje de Consulta Estructurado (SQL por sus siglas en inglés). Forma parte del programa principal, en lugar de ser un proceso independiente como ocurre con los sistemas de gestión de base de datos cliente-servidor \cite{SQL2}. Se utiliza para la gestión de la base de datos del dispositivo móvil.

\subsection{Maven}
Es una herramienta que se permite compilar y  manejar proyectos de \textit{software} basados en Java. Se encarga de conectar las dependencias de los paquetes \cite{MVN1}. Se utilizó para manejar la dependencia entre las librerías usadas para el desarrollo del servidor.

\subsection{Jetty}
Es un servidor HTTP que puede funcionar independientemente por sus propios medios, o que puede correr dentro de otra aplicación \cite{JTY1}. Se utilizó para ofrecer los servicios web creados, mediante el protocolo HTTP.

\subsection{Android Studio}
Es el entorno de desarrollo integrado (IDE por su siglas en inglés) oficial para el desarrollo de aplicaciones nativas para Android \cite{ASD1}. Se utilizó para desarrollar la aplicación móvil.

\subsection{Eclipse}
Es un entorno desarrollo integrado (IDE por sus siglas en inglés) utilizado principalmente para desarrollar aplicaciones en Java \cite{ECL1}. Se utilizó para desarrollar el servidor.

\subsection{Git}
Es un sistema de control de versiones distribuido (DVCS por sus siglas en inglés). Es una herramienta que permite gestionar las distintas versiones de un proyecto de \textit{software} \cite{GIT1}. Se utilizó para mantener un control entre las versiones del código que surgieron a lo largo del desarrollo tanto de la aplicación móvil como del servidor.
