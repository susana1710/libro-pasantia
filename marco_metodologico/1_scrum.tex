\section{Scrum} \label{sect:Scrum}

Scrum es un marco de trabajo que se basa en el desarrollo iterativo e incremental de un producto, en lugar del modelo clásico de planificación y ejecución completa \cite{SCRM12}.  Se caracteriza por ser una metodología ligera, fácil de entender y difícil de dominar, que permite entregar incrementos de producto potencialmente productivos \cite{SCRM1}. 

\subsection{Roles} 

En Scrum el desarrollo se realiza por uno o más equipos de trabajo dentro de los cuales existen tres roles: dueño del producto, facilitador de Scrum y el equipo de desarrollo \cite{SCRM12}. 
 
\subsubsection{Dueño del producto (\textit{Product owner})}

Es el representante de los clientes. Dentro del equipo de Scrum, es el líder principal del producto y el responsable de decidir qué funcionalidades serán desarrolladas y la prioridad que tendrá cada una de ellas. Debe comunicar al resto de los involucrados en el proyecto una visión clara de lo que se quiere lograr. Tiene la obligación de asegurar que siempre se entregue un producto con el máximo de valor, por lo que debe colaborar con el resto del equipo para responder cualquier duda que surja \cite{SCRM12}. Este rol fue asumido por el Ing. Chi Wang Zhong.

\subsubsection{Facilitador de Scrum (\textit{Scrum master})}

Actúa como facilitador tanto para el dueño del producto como para el equipo de desarrollo. Es el encargado de ayudar al resto del equipo a entender y cumplir con los principios y prácticas de Scrum. También tiene la responsabilidad de eliminar cualquier impedimento que el equipo no sea capaz de resolver y que afecte su productividad \cite{SCRM12}. Este rol fue asumido por el Lic. Luis Augusto Peña.

\subsubsection{Equipo de desarrollo}

Es el encargado de desarrollar el producto. Es un equipo que está compuesto por arquitectos, programadores, probadores, administradores de base de datos, diseñadores de interfaces, entre otros. Son los responsables de diseñar, desarrollar y probar el producto \cite{SCRM12}. El equipo de desarrollo estuvo integrado únicamente por la pasante Susana Charara.

\subsection{Actividades}

En Scrum, el trabajo se desarrolla en iteraciones de una duración máxima de un mes, llamadas \textit{\textbf{sprints}}. Al final de cada iteración, se debe haber desarrollado una parte del producto final, la cual debe ser completamente funcional. Dentro de cada iteración existe una serie de eventos o actividades que se llevan a cabo: planeación de la iteración, ejecución de la iteración, reuniones diarias, revisión de la iteración y retrospectiva de la iteración \cite{SCRM12}.

\subsubsection{Planeación de la iteración (\textit{Sprint planning})}
Para determinar qué funcionalidades del producto final son las más importantes y próximas a desarrollar, el equipo de trabajo (dueño del producto, facilitador de Scrum y el equipo de desarrollo) realizan una reunión llamada \textit{sprint planning} o planeación de la iteración \cite{SCRM12}.

Durante la reunión, el dueño del producto y el equipo de desarrollo establecen una meta que debe ser cumplida para el final de la iteración. De acuerdo a esta meta, el equipo de desarrollo decide de una manera realista qué incrementos del producto final pueden entregarse al terminar la iteración \cite{SCRM12}.

\subsubsection{Ejecución de la iteración (\textit{Sprint execution})}

Luego de la ejecución de la iteración, el equipo de desarrollo desarrolla todas las tareas acordadas en la reunión. Esto es lo que se conoce como la ejecución de la iteración \cite{SCRM12}.

\subsubsection{Reunión diaria (\textit{Daily scrum})}

Cada día dentro de la ejecución dela iteración , los miembros del equipo de desarrollo se reúnen durante un máximo de 15 minutos con el fin de informar qué se hizo el día anterior, qué se tiene planificado realizar el presente día y qué impedimentos se han presentado durante el desarrollo de su trabajo \cite{SCRM12}.

\subsubsection{Revisión de la iteración (\textit{Sprint review})}

Al final de cada iteración, ocurre un evento que se conoce como \textit{sprint review} o revisión de la iteración. El objetivo de esta actividad es revisar el incremento y realizar las adaptaciones necesarias al prouducto \cite{SCRM12}.

\subsubsection{Retrospectiva de la iteración (\textit{Sprint retrospective})}

Esta actividad ocurre generalmente luego del \textit{sprint review} y antes del próximo \textit{sprint planning}. Es una oportunidad para que todo el equipo se reúna para discutir qué ha funcionado y qué no acerca de Scrum. Al final de la retrospectiva, el equipo de Scrum habrá discutido qué acciones deberán tomarse para mejorar la dinámica en las próximas iteraciones \cite{SCRM2}. 

\subsection{Artefactos}

Dentro de Scrum existen dos herramientas o artefactos que permiten mantener un seguimiento del proyecto: el documento de historias de usuario y la lista de tareas de la iteración. \cite{SCRM2}.

\subsubsection{Documento de historias de usuario (\textit{Product backlog})}

Es una lista de los requerimientos funcionales del producto ordenados según su importancia. El dueño del producto es el responsable de definir qué elementos serán incluidos en esta lista y de colocarlos según su prioridad, de manera que los elementos de mayor valor o prioridad aparezcan al principio de la lista, y los de menos valor al final de la misma \cite{SCRM2}. 

\subsubsection{Lista de tareas de la iteración (\textit{Sprint backlog})}

Es una lista donde se presenta un subconjunto de los elementos del \textit{product backlog} divididos en tareas más pequeñas \cite{SCRM2}. 


