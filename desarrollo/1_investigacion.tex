\section{Investigación} \label{sect:Investigacion}

El objetivo de esta primera fase consistió en familiarizarse con el entorno de trabajo.

Al inicio, se estudiaron diversos patrones de diseño, donde se investigó acerca de patrones ampliamente utilizados en el desarrollo de aplicaciones para Android y en la programación orientada a objetos en general. Esto permitió estructurar la aplicación de una manera ordenada y entendible, permitiendo así que ésta pueda ser extendida facilmente.

Seguidamente, se investigó acerca de los componentes y herramientas para el desarrollo de aplicaciones en Android, así como el uso del IDE Android Studio.

Finalmente, se realizó una aplicación de prueba para poner en práctica algunos conceptos básicos en el desarrollo de aplicaciones para Android. Con esto se aprendió la estructura general de una aplicación, así como el flujo de ejecución.


