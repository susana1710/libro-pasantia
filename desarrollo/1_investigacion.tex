\section{Investigacion} \label{sect:Investigacion}

El objetivo de esta primera fase era familiarizarse con el entorno de trabajo.

En primer lugar se estudiaron diversos patrones de diseño. Para esto, se investigo acerca de patrones ampliamente utilizados en el desarrollo de aplicaciones para Android y en la programacion orientada a objetos en general. Esto permitio estructurar la aplicacion de una manera mas ordenada y entendible, de manera que se pueda extender facilmente.

Luego de esto, se investigó acerca del desarrollo de aplicaciones para Android. Se aprendió a trabajar con el IDE Android Studio, así como los componentes y herramientas para el desarrollo bajo el sistema operativo.

Por último, se realizó una aplicación de prueba para poner en práctica algunos conceptos básicos en el desarrollo de aplicaciones para Android. Con esto se aprendió la estructura general de una aplicación, asi como el flujo de ejecución.


