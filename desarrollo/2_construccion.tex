\section{Desarrollo} \label{sect:desarrollo}

Durante esta fase se implementó progresivamente la versión alfa del prototipo funcional. A continuación se presentará los \textit{sprints} realizados, sus objetivos y los resultados obtenidos al final de cada uno.

\subsection{Sprint 1}

\subsubsection{Objetivos}
	\begin{itemize}
	\item Mostrar información del balance de una cuenta
	\item Permitir la creación de un nuevo gasto
	\end{itemize}

\subsubsection{Resultados}
\begin{itemize}
\item Se creó la interfaz para ver el total de ingresos, gastos y el balance de la cuenta
\item Se crearon las consultas necesarias a la base de datos para obtener el total de ingresos, gastos y balance
\item Se creó la interfaz para la creación de un nuevo gasto
\item Se creó la consulta para insertar un nuevo gasto en la base de datos
\end{itemize}

\subsection{Sprint 2}

\subsubsection{Objetivos}
\begin{itemize}
\item Mostrar la lista de categorías
\item Permitir la creación de un nuevo ingreso
\item Permitir guardar fotos de un $\entry$
\item Asociar un $\entry$ a una categoría
\end{itemize}

\subsubsection{Resultados}
\begin{itemize}
\item Se creó en la base de datos del dispositivo una consulta para guardar la lista de categorías por defecto
\item Se creó la interfaz para listar las categorías guardadas en la aplicación
\item Se adaptó y reutilizó la vista existente para crear un gasto para permitir la creación de un ingreso
\item Se agregó la funcionalidad de tomar fotos relacionadas a un $\entry$
\item Se creó la consulta para guardar en la base de datos el nombre de las fotos tomadas
\item se agregó la funcionalidad para poder asociar un $\entry$ a una categoría existente

\end{itemize}

\subsection{Sprint 3}
\subsubsection{Objetivos}
\begin{itemize}
\item Mostrar la lista de $\entries$ del mes actual
\item Mostrar los detalles de los $\entries$ existentes
\item Permitir editar y borrar un $\entry$ existente
\item Implementar una calculadora para guardar el monto de un $\entry$
\end{itemize}

\subsubsection{Resultados}
\begin{itemize}
\item Se creó la interfaz para mostrar una lista con los gastos y otra con los ingresos del mes actual
\item Se creó la consulta en la base de datos para obtener los $\entries$
\item Se agregó la funcionalidad para ver los detalles de un $\entry$ ya existente, y poder editarlo
\item Se agregó la funcionalidad para eliminar $\entries$
\item Se creó la interfaz para usar la calculadora que permita ingresar el monto de un $\entry$
\end{itemize}

\subsection{Sprint 4}
\subsubsection{Objetivos}
\begin{itemize}
\item Permitir el manejo de nuevas cuentas
\end{itemize}

\subsubsection{Resultados}
\begin{itemize}

\item Se creó la interfaz para crear una nueva cuenta
\item Se creó la consulta en la base de datos para persistir cuentas
\item Se implementó la funcionalidad para listar las cuentas del usuario 
\item Se implementó la funcionalidad para cambiar de cuenta
\item Se implementó la funcionalidad para editar el nombre de una cuenta existente
\item Se implementó la funcionalidad para eliminar cuentas existentes
\item Se implementó la funcionalidad para cambiar el estado de una cuenta: activa o archivada
\item Se implementó la funcionalidad para ver la lista de $\entries$ de una cuenta desde su creacion hasta la fecha actual

\end{itemize}


\subsection{Sprint 5}
\subsubsection{Objetivos}
\begin{itemize}
\item Permitir el manejo de las fotos de un $\entry$
\item Permitir el manejo de las categorias
\end{itemize}

\subsubsection{Resultados}
\begin{itemize}
\item Se agregó la funcionalidad para ver las fotos de un $\entry$
\item Se agregó la funcionalidad para borrar las fotos de un $\entry$
\item Se agregó la funcionalidad para cambiar la ubicacion en el dispositivo de las fotos tomadas de un $\entry$
\item Se agregó la funcionalidad para crear una categoría
\item Se agregó la funcionalidad para eliminar una categoría existente
\item Se agregó la funcionalidad para editar una categoría existentes
\end{itemize}



\subsection{Sprint 6}
\subsubsection{Objetivos}
\begin{itemize}
\item Mostrar un reporte con la lista de $\entries$ asociados a una cuenta, en un rango de fecha dado
\item Mostrar un reporte con el balance total por categorías asociadas a una cuenta, en un rango de fecha dado
\item Permitir el envío de los reportes en un archivo PDF a otras personas

\end{itemize}

\subsubsection{Resultados}
\begin{itemize}
\item Se implementó la funcionalidad para listar los $\entries$ de una cuenta en el rango de fecha ingresado por el usuario
\item Se implementó la funcionalidad para listar el balance total de las categorías de una cuenta en el rango de fecha ingresado por el usuario
\item Se creó la funcionalidad para compartir el reporte con la lista de $\entries$,incluyendo las fotos, de la cuenta en el rango de fecha ingresado.
\item Se creó la funcionalidad para compartir el reporte con la lista de $\entries$, sin incluir las fotos, de la cuenta en el rango de fecha ingresado.
\item Se creó la funcionalidad para compartir el reporte con el balance total de las categorías de la cuenta en el rango de fecha ingresado.
\end{itemize}

\subsection{Sprint 7}
\subsubsection{Objetivos}

\subsubsection{Resultados}



