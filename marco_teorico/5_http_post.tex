\section{HTTP (\textit{Hypertext Transfer Protocol})}\label{HTTP}

Es el protocolo de comunicación que permite la transferencia de recursos en la web. Un recurso es cualquier información que puede ser identificada por un URL. Existen diversos métodos que permiten realizar peticiones con el protocolo HTTP, entre los que están GET y POST \cite{HTTP2}.

\subsection{HTTP GET}

Es el método más común de HTTP, que permite pedir datos de un recurso en específico \cite{HTTP3}. 

\subsection{HTTP POST}

Es uno de los diferentes métodos de peticiones que soporta el protocolo HTTP, que permite enviar datos a un recurso, los cuales deben ser procesados. Se requiere que los datos sean enviados dentro del cuerpo del mensaje \cite{HTTP3}.

Para enviar estos datos dentro de una petición POST, existen diferentes tipos de codificación. Entre estos tipos está Multipart, que permite enviar mensajes con varias partes combinadas en un solo cuerpo \cite{HTTP1}.