\section{Patrones de diseño} \label{sect:Patrones de diseno}

Un patrón de diseño es una solución general y repetible a problemas que suelen presentarse en el proceso de diseño de software. Para que una solución pueda ser considerada como un patrón de diseño debe ser reutilizable, es decir, que se pueda aplicar a diferentes problemas de diseño en diferentes situaciones \cite{DSP0}. A continuación se describen los patrones utilizados en el diseño de la solución del proyecto.


\subsection{\textit{Singleton}}

El \textit{singleton} es un patrón de diseño que asegura que la clase sólo tiene una instancia y provee un acceso global a la misma \cite{DSP1}. Esto es útil cuando se necesita únicamente un objeto para coordinar las acciones en el sistema \cite{SNG0}.

\subsection{\textit{Adapter}}

Transforma la interfaz de una clase en otra interfaz que el cliente espera. Esto permite que una clase que no pueda utilizar la primera interfaz, sí pueda hacerlo a través de la otra \cite{DSP1}.

\subsection{\textit{Factory}} 

