\section{HTTP (\textit{Hypertext Transfer Protocol})}\label{HTTP}

Es un protocolo de comunicación que permite la transferencia de recursos en la web. Un recurso es cualquier información que puede ser identificada por un URL. Existen diversos métodos que permiten realizar peticiones con el protocolo HTTP, entre los que está POST \cite{HTTP2}. 

Dentro de una petición HTTP suelen enviarse encabezados (\textit{headers}) que proveen información de la misma, entre los que se encuentra el \textit{Content-Type}. Con éste, se especifica el tipo de codificación que se usará en el cuerpo de la petición. Estos tipos pueden ser: application/xml, application/json, text/html, multipart/form-data, etc. 

El tipo multipart permite enviar mensajes con varias partes combinadas en un solo cuerpo, y suele usarse también para el envío de archivos. \cite{HTTP1}

\subsection{HTTP POST}

Es uno de los diferentes métodos de peticiones que soporta el protocolo HTTP, que permite enviar datos a un recurso, los cuales deben ser procesados. Se requiere que los datos sean enviados dentro del cuerpo del mensaje \cite{HTTP3}.

%Para enviar estos datos dentro de una petición POST, existen diferentes tipos de codificación. Entre estos tipos está Multipart, que permite enviar mensajes con varias partes combinadas en un solo cuerpo \cite{HTTP1}.
