\section{Patrones de arquitectura} \label{sect:Patrones de arquitectura}
Un patrón de arquitectura es una solución a problemas de arquitectura de \textit{software}, permitiendo definir una estructura. La diferencia entre los patrones de diseño y los de arquitectura es que estos últimos tienen un nivel de abstracción mayor. \cite{PDA1}

\subsection{Modelo Vista Controlador (MVC)}

Es un patrón de arquitectura de \textit{software} que divide una aplicación  en tres partes que están interconectadas: los datos y la lógica de negocio (modelo), la interfaz de usuario (vista) y el módulo encargado de gestionar las comunicaciones (controlador). Esto se utiliza para separar la representación de la información de la forma en que dicha información es presentada al usuario \cite{MVC0}.

\subsection{Modelo Vista Presentador (MVP)}

Es una derivación del patrón Modelo Vista Controlador (MVC) que se utiliza generalmente para construir interfaces de usuario \cite{MVP0}. En este patrón, la interacción entre el modelo y la vista se logra únicamente a través del presentador, mientras que en el MVC la vista en ocasiones puede comunicarse directamente con el modelo. Otra diferencia con el patrón MVC es que en este último las peticiones son recibidas por el controlador. Éste, se comunica con el modelo para pedir los datos y luego se encarga de mostrar la vista adecuada. En el patrón MVP las peticiones son recibidas por la vista y delegadas al presentador, que es quien se comunica con el modelo para obtener los datos \cite{MVP1}.