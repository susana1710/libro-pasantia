\setcounter{page}{3}
\begin{center}
	{\bf Resumen} \pdfbookmark[0]{Resumen}{resumen} % Sets a PDF bookmark for the dedication
\end{center}	

En esta pasantía se desarrolló, para la empresa Digitalica Group, C.A., una aplicación para dispositivos móviles con la cual se pueden realizar reportes de gastos. Actualmente, la empresa Digitalica Group, C.A. ofrece a sus trabajadores el beneficio de realizar reembolsos de gastos en ciertos rubros; el proyecto nació como solución al problema de agilizar el proceso en que los trabajadores hacen llegar a los supervisores la información de estos gastos. Con la aplicación desarrollada, se pueden registrar gastos e ingresos con un monto, fecha y descripción. Se permite también capturar y guardar fotos tanto de los ingresos como de los gastos, de manera que el usuario puede tomar fotos de los recibos de sus gastos. Además, estos gastos/ingresos pueden estar asociados a categorías, que indican el rubro al que pertenecen. La aplicación permite al usuario generar archivos con formato PDF que contienen un reporte con una lista de gastos, en un rango de fecha determinado; estos reportes se pueden enviar a un servidor web. Este servidor también fue desarrollado durante la pasantía. Se creó un servicio web que recibe credenciales de un usuario y las valida; éste es utilizado por la aplicación móvil para autenticar usuarios. Asimismo, se desarrolló un servicio web que recibe archivos PDF con reportes de gastos, el cual es utilizado por la aplicación para enviar reportes. Igualmente, se desarrollaron servicios y una aplicación web, con la cual los supervisores de la empresa pueden revisar y descargar los archivos de los reportes recibidos por el servidor, aprobarlos y rechazarlos.

En este informe se describen los conceptos teóricos que ayudaron al diseño y desarrollo de la solución. Igualmente, se describe el proceso de desarrollo de los tres componentes principales desarrollados durante la pasantía: aplicación nativa para Android, servidor web y aplicación web. El desarrollo de estos involucró el diseño e implementación de los modelos de datos de la aplicación y el servidor, así como múltiples interfaces de usuario que permiten la interacción con los modelos. Para cada componente, se describen el entorno de trabajo y las herramientas utilizadas. Se utilizó Java como lenguaje de programación tanto de la aplicación móvil como del servidor; entre las herramientas que se emplearon en el desarrollo del servidor se pueden mencionar AppFuse, Spring, Hibernate y Tapestry, entre otros.

Por otra parte, se utilizó Scrum como marco de trabajo, dividiendo el desarrollo del proyecto en ocho iteraciones, en las que se implementaron los componentes y funcionalidades necesarios para el cumplimiento de los objetivos de la pasantía.

El objetivo del desarrollo de la pasantía fue ofrecer una solución al problema que actualmente se presenta en la empresa de agilizar el proceso de reembolso sobre ciertos gastos.


