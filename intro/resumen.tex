\setcounter{page}{3}
\begin{center}
	{\bf Resumen} \pdfbookmark[0]{Resumen}{resumen} % Sets a PDF bookmark for the dedication
\end{center}	

En este informe se describe el proceso de desarrollo de una aplicación móvil para el reporte de gastos, realizado para la compañía Digitalica Group, C.A. De forma general, la aplicación permite al usuario registrar gastos e ingresos, y asociarlos a una cuenta. Para cada gasto/ingreso, se puede definir una categoría a la que pertenece, así como capturar y guardar fotos. Además, con la aplicación móvil se puede generar reportes que contienen una lista de gastos y enviarlos a un servidor web, desarrollado igualmente en esta pasantía. 

Se describen los aspectos más relevantes de los tres componentes principales desarrollados durante la pasantía: aplicación nativa para Android, servidor web y aplicación web. Para cada uno de estos se describen el entorno de trabajo y las tecnologías utilizadas; entre ellas, Java, el lenguaje utilizado para la programación tanto de la aplicación móvil como el servidor; herramientas que permitieron el desarrollo del servidor, como AppFuse, Spring, Hibernate y Tapestry, entre otros. Además, se define brevemente Scrum, que fue el marco de trabajo sobre el cual se desarrolló el proyecto. Durante el proyecto fue necesario dominar ciertos conceptos que ayudaron tanto en el diseño como en el desarrollo de la solución, los cuales se definen en el presente informe. Entre ellos se incluyen distintos patrones de diseño y de arquitectura, como el Modelo Vista Controlador (MVC) y el Modelo Vista Presentador (MVP).

El objetivo del desarrollo de la pasantía fue ofrecer una solución al problema que actualmente se presenta en la empresa de agilizar el proceso de reembolso sobre ciertos gastos.


