\setcounter{page}{3}
\begin{center}
	{\bf Resumen} \pdfbookmark[0]{Resumen}{resumen} % Sets a PDF bookmark for the dedication
\end{center}	

En este informe se describe el proceso de desarrollo de una aplicación móvil para el reporte de gastos, realizado para la compañía Digitalica Group C.A. De forma general, la aplicación permite al usuario registrar gastos e ingresos a una cuenta, asociarlos a categorías, así como capturar y guardar fotos de cada uno. Además, puede generar reportes y enviarlos a un servidor web, desarrollado igualmente en esta pasantía. 

Se describen los aspectos más relevantes de los tres componentes principales: aplicación nativa para Android, servidor web y aplicación web. Para cada uno de estos se describen el entorno de trabajo y las tecnologías utilizadas; entre ellas, el lenguaje utilizado para la programación tanto de la aplicación móvil como el servidor: Java; herramientas que permitieron el desarrollo del servidor, como AppFuse, Spring, Hibernate y Tapestry; entre otros. Además, se define brevemente el marco de trabajo sobre el cual se desarrolló el proyecto, Scrum. Durante el proyecto fue necesario dominar ciertos conceptos, los cuales se definen en el presente informe,  entre los que se incluyen varios patrones de diseño y arquitectura.

El objetivo del desarrollo de la pasantía fue ofrecer una solución al problema que actualmente se presenta en la empresa de agilizar el proceso de reembolso sobre ciertos gastos.


