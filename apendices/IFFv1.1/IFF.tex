%\documentclass[12pt,letterpaper]{article}
%\usepackage[utf8]{inputenc}
%\usepackage{amsmath}
%\usepackage{amsfonts}
%\usepackage{amssymb}
%\usepackage[spanish]{babel}
%\usepackage{array}
%\usepackage{longtable}
%\author{Jon Ander Ricchiuti}
%\title{IFF v1.1}
%\begin{document}
%\pagenumbering{arabic}
%\maketitle
%\thispagestyle{empty}
%\newpage
\chapter{Intercambio de Información Financiera (IFF)}

\section*{Introducción}

El protocolo de intercambio de información financiera (IFF), describe la forma correcta de comunicación con un prototipo de sistema bancario creado en Synergy Global Business (SGB). El IFF busca estandarizar el intercambio de información financiera que realizará el prototipo bancario. De esta manera la comunicación es sencilla y a la vez robusta. Para la realización de este protocolo de comunicación se utilizó como base el IFX (Interactive Financial Exchange).
\\
\\ 
La forma de comunicación que el IFF	utiliza es de tipo petición-respuesta (request-response). En cada petición se debe especificar un método. Este método define la naturaleza de la petición.
%\pagenumbering{arabic}
%\newpage

\section{Tipo de datos}

Los tipos de datos que se utilizarán en este estándar son los siguientes:
\begin{itemize}
\item Cadena de caracteres.
\item Enumeración.
\item Tiempo y Hora.
\item Entero.
\item Decimal
\item Booleano.
\end{itemize}

\subsection{Cadena de caracteres}
Las cadenas de caracteres se representan con el nombre de ``Cadena de caracteres'' seguido de la longitud de la misma. Esta longitud es representada entre paréntesis de la siguiente forma.
\begin{itemize}
\item Cadena de caracteres (X-Y), indica que la longitud mínima de la cadena de caracteres es ``X'' y la máxima longitud es ``Y''.
\item Cadena de caracteres (X+), indica que la longitud mínima de la cadena de caracteres es ``X'' pero no tiene longitud máxima para la misma.
\end{itemize}
Si la longitud no es especificada entonces no existe restricción sobre el tamaño de la cadena de caracteres.

\subsection{Enumeración}
Son los valores que puede tomar un campo. Estos pertenecen a un conjunto de cadena de caracteres definidas específicamente para ese campo en particular. Los diferentes tipos de enumeración serán especificados en la siguiente sección.

\subsection{Tiempo y Hora}
Tanto la hora como la fecha son cadenas de caracteres que se representan con un fromato particular. Para la hora el formto es ``\%H:\%M:\%S''. Para la fecha el formato es ``\%Y-\%m-\%d \%H:\%M:\%S''. Donde \%Y representa el año, \%m el mes, \%d el dia, \%H la hora, \%M el minuto, \%S el segundo. Todos los componentes de la hora y fecha se representan con caracteres numéricos.

\subsection{Entero}
Es un número entero y puede representarse de dos formas diferentes. Como un entero de cuatro bytes o como un entero de ocho bytes. Para especificar que es un entero de ocho bytes, debe ser escrito de la siguiente forma: Entero(8).
\\
\\
Si no se especifica que un entero es de ocho bytes entonces se asume que es de cuatro  bytes.

\subsection{Decimal}
Es un número con hasta quince dígitos decimales.

\subsection{Booleano}
Un Booleano representa si una condición se cumple o no. En el caso del IFF un Booleano será representado por medio de un carácter. Es decir, el carácter 'T' será el que representa cuando un estado es cierto y 'F' será el valor de cuando el estado no se cumple.

\section{Tipos de Enumeración}
\subsection{Correspondiente a personVerifyType}
\begin{center}
\begin{tabular}{|>{\centering\arraybackslash}p{0.3\textwidth}|>{\centering\arraybackslash}p{0.3\textwidth}|>{\centering\arraybackslash}p{0.3\textwidth}|}
\hline 
\bfseries {Valor} & \bfseries {Descripción} & \bfseries {Por defecto} \\ 
\hline 
Passport & Pasaporte de crédito & N \\ 
\hline 
CI & Cedula de identidad & N \\
\hline 
\end{tabular} 
\end{center}

\subsection{Correspondiente a nameAddrType}
\begin{center}
\begin{tabular}{|>{\centering\arraybackslash}p{0.3\textwidth}|>{\centering\arraybackslash}p{0.3\textwidth}|>{\centering\arraybackslash}p{0.3\textwidth}|}
\hline 
\bfseries {Valor} & \bfseries {Descripción} & \bfseries {Por defecto} \\ 
\hline 
Customer & Es la dirección del cliente & N \\ 
\hline 
ShipTo & Dirección a la cual algo debería ser enviado por correo & N \\
\hline 
Delivery & Dirección a la cual serán enviadas las facturas en papel & N \\
\hline 
\end{tabular} 
\end{center}

\subsection{Correspondiente a addrType}
\begin{center}
\begin{tabular}{|>{\centering\arraybackslash}p{0.3\textwidth}|>{\centering\arraybackslash}p{0.3\textwidth}|>{\centering\arraybackslash}p{0.3\textwidth}|}
\hline 
\bfseries {Valor} & \bfseries {Descripción} & \bfseries {Por defecto} \\ 
\hline 
Seasonal & Habitación vacacional & N \\ 
\hline 
Primary & Habitación principal & N \\
\hline 
Secondary & Habitación secundaria & N \\
\hline
Business & Dirección de negocio & N \\
\hline 
\end{tabular} 
\end{center}

\subsection{Correspondiente a cardStatusCode}
\begin{center}
\begin{tabular}{|>{\centering\arraybackslash}p{0.3\textwidth}|>{\centering\arraybackslash}p{0.3\textwidth}|>{\centering\arraybackslash}p{0.3\textwidth}|}
\hline 
\bfseries {Valor} & \bfseries {Descripción} & \bfseries {Por defecto} \\ 
\hline 
Active & Activa & N \\ 
\hline 
Expired & Vencida & N \\
\hline 
Blocked & Bloqueada & N \\
\hline
\end{tabular} 
\end{center}

\subsection{Correspondiente a accountStatusCode}
\begin{center}
\begin{tabular}{|>{\centering\arraybackslash}p{0.3\textwidth}|>{\centering\arraybackslash}p{0.3\textwidth}|>{\centering\arraybackslash}p{0.3\textwidth}|}
\hline 
\bfseries {Valor} & \bfseries {Descripción} & \bfseries {Por defecto} \\ 
\hline 
Active & Activa & N \\ 
\hline 
Blocked & Bloqueada & N \\
\hline
\end{tabular} 
\end{center}

\subsection{Correspondiente a cardType}
\begin{center}
\begin{tabular}{|>{\centering\arraybackslash}p{0.3\textwidth}|>{\centering\arraybackslash}p{0.3\textwidth}|>{\centering\arraybackslash}p{0.3\textwidth}|}
\hline 
\bfseries {Valor} & \bfseries {Descripción} & \bfseries {Por defecto} \\ 
\hline 
Credit & Tarjeta de crédito & N \\ 
\hline 
Debit & Tarjeta de débito & N \\
\hline 
\end{tabular} 
\end{center}

\subsection{Correspondiente a brand}
\begin{center}
\begin{tabular}{|>{\centering\arraybackslash}p{0.3\textwidth}|>{\centering\arraybackslash}p{0.3\textwidth}|>{\centering\arraybackslash}p{0.3\textwidth}|}
\hline 
\bfseries {Valor} & \bfseries {Descripción} & \bfseries {Por defecto} \\ 
\hline 
Visa &  & N \\ 
\hline 
MasterCard &  & N \\
\hline 
\end{tabular} 
\end{center}

\subsection{Correspondiente a transType}
\begin{center}
\begin{tabular}{|>{\centering\arraybackslash}p{0.3\textwidth}|>{\centering\arraybackslash}p{0.3\textwidth}|>{\centering\arraybackslash}p{0.3\textwidth}|}
\hline 
\bfseries {Valor} & \bfseries {Descripción} & \bfseries {Por defecto} \\ 
\hline 
Withdrawal & Retiro & N \\ 
\hline 
Deposit & Deposito & N \\
\hline 
Transference & Transferencia & N \\
\hline
\end{tabular} 
\end{center}

\subsection{Correspondiente a acctType}
\begin{center}
\begin{tabular}{|>{\centering\arraybackslash}p{0.3\textwidth}|>{\centering\arraybackslash}p{0.3\textwidth}|>{\centering\arraybackslash}p{0.3\textwidth}|}
\hline 
\bfseries {Valor} & \bfseries {Descripción} & \bfseries {Por defecto} \\ 
\hline 
Saving & Ahorro & N \\ 
\hline 
Current & Corriente & N \\
\hline 
Loan & Préstamo & N \\
\hline
\end{tabular} 
\end{center}

\subsection{Correspondiente a contactInfo}
\begin{center}
\begin{tabular}{|>{\centering\arraybackslash}p{0.3\textwidth}|>{\centering\arraybackslash}p{0.3\textwidth}|>{\centering\arraybackslash}p{0.3\textwidth}|}
\hline 
\bfseries {Valor} & \bfseries {Descripción} & \bfseries {Por defecto} \\ 
\hline 
dayPhone & Teléfono de contacto durante el día & N \\
\hline 
evePhone & Teléfono de contacto durante la tarde & N \\
\hline 
dayFax & Fax de contacto durante el día & N \\
\hline
eveFax & Fax de contacto durante la tarde & N \\
\hline
emailAddr & Dirección de correo electrónica & N \\
\hline
\end{tabular} 
\end{center}
	
\section{Recursos}
El protocolo IFF esta basado en recursos. Los recursos son fuentes de información sobre las cuales se realizan las peticiones. Estos recursos son divididos en dos grandes grupos, los concretos y los abstractos.

\subsection{Recursos concretos}
Los recursos concretos son la representación directa del modelo de datos que expone el core bancario para ofrecer sus servicios. Este tipo de recursos es muy sencillo y son los que permiten realizar las operaciones más básicas. \\

A continuación se presentan los recursos concretos.

\subsubsection{Nombre de usuario ``login''}
Contiene la información que relaciona el nombre de usuario electrónico con su información  en la institución financiera.

\begin{center}
\begin{tabular}{|>{\centering\arraybackslash}p{0.2\textwidth}|>{\centering\arraybackslash}p{0.2\textwidth}|>{\centering\arraybackslash}p{0.2\textwidth}|>{\centering\arraybackslash}p{0.2\textwidth}|}
\hline 
\bfseries {Etiqueta} & \bfseries {Tipo} & \bfseries {Uso} & \bfseries {Descripción} \\ 
\hline 
username & Cadena de caracteres (6-20) & Requerido & ID de ingreso del cliente \\ 
\hline 
password & Cadena de caracteres (6+) & Opcional & Clave de ingreso \\ 
\hline 
custPermId & Cadena de caracteres (32+) & Requerido & ID permanente del cliente. Es asignado por la institución financiera para representar al cliente en el sistema \\ 
\hline 
\end{tabular}
\end{center}

\subsubsection{Cliente ``customer''}
Tiene la información que identifica inequívocamente a un cliente.

\begin{center}
\begin{longtable}{|>{\centering\arraybackslash}p{0.25\textwidth}|>{\centering\arraybackslash}p{0.2\textwidth}|>{\centering\arraybackslash}p{0.15\textwidth}|>{\centering\arraybackslash}p{0.2\textwidth}|}
\hline 
\bfseries {Etiqueta} & \bfseries {Tipo} & \bfseries {Uso} & \bfseries {Descripción} \\ 
\hline 
custPermId & Cadena de caracteres (32+) & Opcional & ID permanente del cliente. Es asignado por la institución financiera para representar al cliente en el sistema \\ 
\hline 
personId & Cadena de caracteres (32+) & Opcional & Relación al objeto ``person'' \\ 
\hline 
custLogin & Cadena de caracteres (6-20) & Opcional & ID permanente del cliente. Es asignado por la institución financiera para representar al cliente en el sistema \\ 
\hline 
personalIdent & Cadena de caracteres (8+) & Requerido & Identificación personal presentada por el cliente \\ 
\hline 
personVerifyType & Enumeración & Requerido & El tipo de documento con el cual se verifica la identidad del cliente \\ 
\hline
dtBCustomer & Fecha & Opcional & Momento en el cual la persona se vuelve cliente de la institución \\ 
\hline
dtLLogin & Fecha & Opcional & Último momento en el cual el cliente utiliza su cuenta \\ 
\hline
group & Cadena de caracteres (32+) & Opcional & Relaciona al cliente con el objeto ``grupo'' \\ 
\hline
\end{longtable}
\end{center}

\subsubsection{Información del banco ``bankInformation''}
Agrupa la información esencial de una agencia bancaria.

\begin{center}
\begin{longtable}{|>{\centering\arraybackslash}p{0.2\textwidth}|>{\centering\arraybackslash}p{0.2\textwidth}|>{\centering\arraybackslash}p{0.2\textwidth}|>{\centering\arraybackslash}p{0.2\textwidth}|}
\hline 
\bfseries {Etiqueta} & \bfseries {Tipo} & \bfseries {Uso} & \bfseries {Descripción} \\ 
\hline 
bankId & Cadena de caracteres (4+) & Opcional & ID que identifica a la agencia bancaria \\ 
\hline 
name & Cadena de caracteres & Opcional & Nombre de la agencia \\ 
\hline 
branchId & Cadena de caracteres & Opcional & ID que identifica a la sucursal \\ 
\hline 
branchName & Cadena de caracteres & Opcional & Nombre de la sucursal \\ 
\hline 
postAddr & Cadena de caracteres & Opcional & Dirección \\ 
\hline
city & Cadena de caracteres & Opcional & Ciudad \\ 
\hline
stateProv & Cadena de caracteres & Opcional & Estado o Provincia \\ 
\hline
postalCode &  Cadena de caracteres (4+) & Opcional & Código postal \\ 
\hline
country & Cadena de caracteres & Opcional & País \\ 
\hline
\end{longtable}
\end{center}

\subsubsection{Dirección ``address''}
Representa la dirección suministrada por el cliente.

\begin{center}
\begin{longtable}{|>{\centering\arraybackslash}p{0.2\textwidth}|>{\centering\arraybackslash}p{0.2\textwidth}|>{\centering\arraybackslash}p{0.2\textwidth}|>{\centering\arraybackslash}p{0.2\textwidth}|}
\hline 
\bfseries {Etiqueta} & \bfseries {Tipo} & \bfseries {Uso} & \bfseries {Descripción} \\ 
\hline 
addressId & Cadena de caracteres & Opcional & ID que identifica a la dirección \\ 
\hline 
custPermId & Cadena de caracteres (32+) & Requerido & ID permanente del cliente. Es asignado por la institución financiera para representar al cliente en el sistema \\
\hline 
nameAddrType & Enumeración & Requerido & Define el uso de la información suministrada \\ 
\hline 
addr & Cadena de caracteres & Opcional & Dirección \\ 
\hline
city & Cadena de caracteres & Opcional & Ciudad \\ 
\hline
stateProv & Cadena de caracteres & Opcional & Estado o Provincia \\ 
\hline
postalCode &  Cadena de caracteres (4+) & Opcional & Código postal \\ 
\hline
country & Cadena de caracteres & Opcional & País \\ 
\hline
addrType & Enumeración & Opcional & Define el tipo de dirección \\ 
\hline
startDt & Hora & Opcional & Hora de inicio \\ 
\hline
endDt & Hora & Opcional & Hora de fin \\ 
\hline
\end{longtable}
\end{center}

\subsubsection{Información de contacto ``contactInfo''}
Información suministrada por el cliente para poder ser contactado en caso de necesitarlo.

\begin{center}
\begin{longtable}{|>{\centering\arraybackslash}p{0.25\textwidth}|>{\centering\arraybackslash}p{0.2\textwidth}|>{\centering\arraybackslash}p{0.15\textwidth}|>{\centering\arraybackslash}p{0.2\textwidth}|}
\hline 
\bfseries {Etiqueta} & \bfseries {Tipo} & \bfseries {Uso} & \bfseries {Descripción} \\ 
\hline 
contactInfoId & Cadena de caracteres (32+) & Opcional & ID que identifica la información de contacto del cliente \\ 
\hline 
custPermId & Cadena de caracteres (32+) & Requerido & ID permanente del cliente. Es asignado por la institución financiera para representar al cliente en el sistema \\
\hline 
custContactPref & Enumeración & Requerido & Representa la manera en la cual el cliente será contactado \\ 
\hline 
prefTimeStart & Hora & Opcional & Hora a partir de la cual puede ser contactado \\ 
\hline
prefTimeEnd & Hora de caracteres & Opcional & Hora a partir de la cual ya no puede ser contactado \\ 
\hline
dayPhone & Cadena de caracteres & Opcional (ver descripción) & Teléfono de contacto durante el día. 
\\ & & & \\
& & & Este campo es requerido si ni ``evePhone'', ``dayFax'', ``eveFax'' o ``emailAddr'' es suministrado \\ 
\hline
evePhone & Cadena de caracteres & Opcional (ver descripción) & Teléfono de contacto durante la tarde. \\ & & & \\
& & & Este campo es requerido si ni ``dayPhone'', ``dayFax'', ``eveFax'' o ``emailAddr'' es suministrado \\ 
\hline
dayFax & Cadena de caracteres & Opcional (ver descripción) & Fax de contacto durante el día. \\ & & & \\
& & & Este campo es requerido si ni ``dayPhone'', ``evePhone'', ``eveFax'' o ``emailAddr'' es suministrado \\ 
\hline
eveFax & Cadena de caracteres & Opcional (ver descripción) & Fax de contacto durante la tarde. \\ & & & \\
& & & Este campo es requerido si ni ``dayPhone'', ``evePhone'', ``dayFax'' o ``emailAddr'' es suministrado \\ 
\hline
emailAddr & Cadena de caracteres & Opcional (ver descripción) & Correo electrónico de contacto. \\ & & & \\
& & & Este campo es requerido si ni ``dayPhone'', ``evePhone'', ``dayFax'' o ``eveFax'' es suministrado \\ 
\hline
\end{longtable}
\end{center}

\subsubsection{Información personal ``personalInfo''}
Contiene la información personal de un cliente.

\begin{center}
\begin{longtable}{|>{\centering\arraybackslash}p{0.2\textwidth}|>{\centering\arraybackslash}p{0.2\textwidth}|>{\centering\arraybackslash}p{0.2\textwidth}|>{\centering\arraybackslash}p{0.2\textwidth}|}
\hline 
\bfseries {Etiqueta} & \bfseries {Tipo} & \bfseries {Uso} & \bfseries {Descripción} \\ 
\hline 
personalInfoId & Cadena de caracteres (32+) & Opcional & ID que identifica a la información personal del cliente \\ 
\hline 
custPermId & Cadena de caracteres (32+) & Requerido & ID permanente del cliente. Es asignado por la institución financiera para representar al cliente en el sistema \\
\hline 
lastName & Cadena de caracteres & Requerido & Apellido del cliente \\ 
\hline 
firstName & Cadena de caracteres & Requerido & Dirección \\ 
\hline
middleName & Cadena de caracteres & Opcional & Ciudad \\ 
\hline
tittlePrefix & Cadena de caracteres & Opcional & Titulo por el cual llamar al cliente. Por ejemplo ``Dr.'' \\ 
\hline
nameSuffix & Cadena de caracteres & Opcional & Sufijo agregado al final del nombre del cliente. Por ejemplo ``Jr.'' \\ 
\hline
\end{longtable}
\end{center}

\subsubsection{Preferencia ``preference''}
Permite al cliente definir cierto comportamiento sobre su cuenta. El cliente puede establecer un monto predeterminado para concepto de retiro sobre una de sus cuentas. También, si se le ha hecho una transferencia al cliente y no se especificó cuenta destino, el dinero será transferido a la cuenta que el cliente haya definido por defecto.

\begin{center}
\begin{longtable}{|>{\centering\arraybackslash}p{0.3\textwidth}|>{\centering\arraybackslash}p{0.15\textwidth}|>{\centering\arraybackslash}p{0.15\textwidth}|>{\centering\arraybackslash}p{0.2\textwidth}|}
\hline 
\bfseries {Etiqueta} & \bfseries {Tipo} & \bfseries {Uso} & \bfseries {Descripción} \\ 
\hline 
preferenceId & Cadena de caracteres (32+) & Opcional & ID que identifica a la información de preferencia del cliente \\ 
\hline 
custPermId & Cadena de caracteres (32+) & Requerido & ID permanente del cliente. Es asignado por la institución financiera para representar al cliente en el sistema \\
\hline 
acctId & Cadena de caracteres (32+) & Opcional & ID de la cuenta a la cual se le aplicaran los consumos por concepto de retiros predefinidos \\ 
\hline 
defaultTranfAccount & Cadena de caracteres (32+) & Opcional & ID de la cuenta a la cual se le aplicaran las transferencias sin cuenta de destino especificada \\ 
\hline
withdrawalAmt & Cadena de caracteres (32+) & Opcional (ver descripción) & Monto de retiro por defecto. 
\\ & & & \\
& & & Este campo es requerido si ``acctId'' es especificado \\
\hline
\end{longtable}
\end{center}

\subsubsection{Transferencias a terceros ``registeredRecipient''}
Contiene los datos de alguna cuenta o tarjeta de otro banco junto con la identificación de sus acreedores.

\begin{center}
\begin{longtable}{|>{\centering\arraybackslash}p{0.25\textwidth}|>{\centering\arraybackslash}p{0.2\textwidth}|>{\centering\arraybackslash}p{0.15\textwidth}|>{\centering\arraybackslash}p{0.2\textwidth}|}
\hline 
\bfseries {Etiqueta} & \bfseries {Tipo} & \bfseries {Uso} & \bfseries {Descripción} \\ 
\hline 
recipientId & Cadena de caracteres (32+) & Opcional & ID que identifica la información acerca de un cliente en otra institución financiera \\ 
\hline 
custPermId & Cadena de caracteres (32+) & Requerido & ID permanente del cliente. Es asignado por la institución financiera para representar al cliente en el sistema \\
\hline 
personId & Cadena de caracteres (32+) & Opcional & ID que identifica al objeto ``person'' \\ 
\hline 
acctNum & Cadena de caracteres & Opcional (ver descripción) & representa el número de cuenta en alguna otra institución financiera. 
\\ & & & \\
& & & Este campo es requerido si ``cardSeqNum'' no es especificado \\ 
\hline
cardSeqNum & Cadena de caracteres & Opcional (ver descripción) & representa el número de tarjeta de alguna otra institución financiera. 
\\ & & & \\
& & & Este campo es requerido si ``acctNum'' no es especificado \\ 
\hline
name & Cadena de caracteres & Requerido & Nombre del beneficiario \\ 
\hline
desc & Cadena de caracteres & Requerido & Descripción \\ 
\hline
maxAmtLimit & Cadena de caracteres & Opcional & Máximo monto permitido para realizar la transferencia \\ 
\hline
personalIdent & Cadena de caracteres (8+) & Requerido & Identificación personal presentada por el cliente \\ 
\hline
personVerifyType & Enumeración & Requerido & El tipo de documento con el cual se verifica la identidad del cliente \\ 
\hline
\end{longtable}
\end{center}

\subsubsection{Persona ``person''}
Contiene los datos que identifican a los clientes como personas. También reúne los
datos de los clientes que tienen cuentas en otros bancos, estos datos provienen de
``registeredRecipient''.

\begin{center}
\begin{longtable}{|>{\centering\arraybackslash}p{0.3\textwidth}|>{\centering\arraybackslash}p{0.15\textwidth}|>{\centering\arraybackslash}p{0.15\textwidth}|>{\centering\arraybackslash}p{0.2\textwidth}|}
\hline 
\bfseries {Etiqueta} & \bfseries {Tipo} & \bfseries {Uso} & \bfseries {Descripción} \\ 
\hline 
personId & Cadena de caracteres (32+) & Opcional & ID que identifica a la información de una persona \\ 
\hline 
name & Cadena de caracteres & Requerido & Nombre \\
\hline 
\end{longtable}
\end{center}

\subsubsection{Conocido ``known''}
Contiene la información de las personas conocidas. De esta forma se puede pueden
realizar transferencias a personas en lugar de a cuentas.

\begin{center}
\begin{longtable}{|>{\centering\arraybackslash}p{0.3\textwidth}|>{\centering\arraybackslash}p{0.15\textwidth}|>{\centering\arraybackslash}p{0.15\textwidth}|>{\centering\arraybackslash}p{0.2\textwidth}|}
\hline 
\bfseries {Etiqueta} & \bfseries {Tipo} & \bfseries {Uso} & \bfseries {Descripción} \\ 
\hline 
knownId & Cadena de caracteres (32+) & Opcional & ID que identifica a la información acerca de un conocido	 \\ 
\hline 
personId & Cadena de caracteres (32+) & Requerido & ID que identifica la información acerca de una persona.
\\ & & & \\
& & & En este caso representa a un conocido \\
\hline 
custPermId & Cadena de caracteres (32) & Requerido & ID permanente del cliente. Es asignado por la institución financiera para representar al cliente en el sistema.
\\ & & & \\
& & & En este caso representa al conocedor \\
\hline 
relationship & Cadena de caracteres & Requerido & Describe el tipo de relación entre el conocedor y el conocido. \\ 
\hline 
status & Booleano & Opcional & Representa si se ha validado que estas dos personas se conocen \\ 
\hline 
\end{longtable}
\end{center}

\subsubsection{Miembro de un grupo ``groupMember''}
Un cliente tiene la capacidad de crear grupo de personas conocidas. De esta forma puede establecer en que cuenta serán ubicados los fondos recibidos por parte de algún miembro del grupo. Un miembro del grupo es aquel cliente que pertenezca a un grupo.

\begin{center}
\begin{longtable}{|>{\centering\arraybackslash}p{0.25\textwidth}|>{\centering\arraybackslash}p{0.2\textwidth}|>{\centering\arraybackslash}p{0.15\textwidth}|>{\centering\arraybackslash}p{0.2\textwidth}|}
\hline 
\bfseries {Etiqueta} & \bfseries {Tipo} & \bfseries {Uso} & \bfseries {Descripción} \\ 
\hline 
groupMemberId & Cadena de caracteres (32+) & Opcional & ID que identifica al objeto ``groupMember'' \\ 
\hline 
custPermId & Cadena de caracteres (32+) & Requerido & ID permanente del cliente. Es asignado por la institución financiera para representar al cliente en el sistema \\
\hline 
member & Cadena de caracteres (32+) & Requerido & Representa al cliente miembro del grupo \\
\hline 
groupId & Cadena de caracteres (32+) & Requerido & Identifica al grupo al cual pertenece un miembro de grupo\\
\hline 
\end{longtable}
\end{center}

\subsubsection{Grupo ``group''}
Es una unidad en la cual un cliente puede agrupar a otros clientes del banco y
predefinir una cuenta en la cual los miembros al grupo transferirán.

\begin{center}
\begin{longtable}{|>{\centering\arraybackslash}p{0.2\textwidth}|>{\centering\arraybackslash}p{0.2\textwidth}|>{\centering\arraybackslash}p{0.2\textwidth}|>{\centering\arraybackslash}p{0.2\textwidth}|}
\hline 
\bfseries {Etiqueta} & \bfseries {Tipo} & \bfseries {Uso} & \bfseries {Descripción} \\ 
\hline 
groupId & Cadena de caracteres (32+) & Opcional & ID que identifica al objeto ``group'' \\ 
\hline
acctId & Cadena de caracteres (32+) & Opcional & ID de la cuenta a la cual se le aplicaran los consumos por concepto de retiros predefinidos \\ 
\hline 
name & Cadena de caracteres & Requerido & Nombre del grupo \\
\hline 
descripción & Cadena de caracteres & Opcional & Descripción del grupo \\
\hline 
\end{longtable}
\end{center}

\subsubsection{Estado de la cuenta ``accountStatus''}
Tiene la información del estado en el cual se encuentra la cuenta.

\begin{center}
\begin{longtable}{|>{\centering\arraybackslash}p{0.25\textwidth}|>{\centering\arraybackslash}p{0.2\textwidth}|>{\centering\arraybackslash}p{0.15\textwidth}|>{\centering\arraybackslash}p{0.2\textwidth}|}
\hline 
\bfseries {Etiqueta} & \bfseries {Tipo} & \bfseries {Uso} & \bfseries {Descripción} \\ 
\hline 
accountStatusId & Cadena de caracteres (32+) & Opcional & ID que identifica al objeto ``accountStatus'' \\ 
\hline
acctId & Cadena de caracteres (32+) & Opcional & ID de la cuenta a la cual se le aplicaran los consumos por concepto de retiros predefinidos \\ 
\hline 
accountStatusCode & Enumeración & Requerido & Representa el estado de la cuenta \\
\hline 
effDt & Fecha & Opcional & Fecha en la cual se hizo efectivo dicho estado \\
\hline 
statusModBy & Cadena de caracteres & Opcional & Tiene la información acerca de quien modificó el estado \\
\hline 
statusDesc & Cadena de caracteres & Opcional & Descripción sobre el estado \\
\hline 
\end{longtable}
\end{center}

\subsubsection{Estado de la tarjeta ``cardStatus''}
Contiene la información del estado en el cual se encuentra la tarjeta.

\begin{center}
\begin{longtable}{|>{\centering\arraybackslash}p{0.25\textwidth}|>{\centering\arraybackslash}p{0.2\textwidth}|>{\centering\arraybackslash}p{0.15\textwidth}|>{\centering\arraybackslash}p{0.2\textwidth}|}
\hline 
\bfseries {Etiqueta} & \bfseries {Tipo} & \bfseries {Uso} & \bfseries {Descripción} \\ 
\hline 
card	StatusId & Cadena de caracteres (32+) & Opcional & ID que identifica al objeto ``cardStatus'' \\ 
\hline
cardEmBossNum & Cadena de caracteres (32+) & Requerido & Número de la tarjeta a la cual pertenece el estado \\ 
\hline 
cardStatusCode & Enumeración & Requerido & Representa el estado de la tarjeta \\
\hline 
effDt & Fecha & Opcional & Fecha en la cual se hizo efectivo dicho estado \\
\hline 
statusModBy & Cadena de caracteres & Opcional & Tiene la información acerca de quien modificó el estado \\
\hline 
statusDesc & Cadena de caracteres & Opcional & Descripción sobre el estado \\
\hline 
\end{longtable}
\end{center}

\subsubsection{Tarjeta ``card''}
Contiene la información de la tarjeta.

\begin{center}
\begin{longtable}{|>{\centering\arraybackslash}p{0.25\textwidth}|>{\centering\arraybackslash}p{0.2\textwidth}|>{\centering\arraybackslash}p{0.15\textwidth}|>{\centering\arraybackslash}p{0.2\textwidth}|}
\hline 
\bfseries {Etiqueta} & \bfseries {Tipo} & \bfseries {Uso} & \bfseries {Descripción} \\ 
\hline 
cardEmBossNum & Cadena de caracteres (32+) & Requerido & Número de la tarjeta \\ 
\hline
acctId & Cadena de caracteres (32+) & Requerido & ID de la cuenta a la cual se le aplicaran los consumos por concepto de retiros predefinidos \\ 
\hline
cardType & Enumeración & Opcional & Tipo de tarjeta \\
\hline 
brand & Enumeración & Opcional & Consorcio al que pertenece la tarjeta \\
\hline 
issuerName & Cadena de caracteres & Opcional & Nombre del tarjetahabiente \\
\hline 
issDt & Fecha & Opcional & Fecha en la cual se emite la tarjeta \\
\hline 
expDt & Fecha & Opcional & Fecha en la cual expira la tarjeta \\
\hline 
\end{longtable}
\end{center}

\subsubsection{Balance ``balance''}
Contiene la información del dinero existente en una cuenta y las transacciones que
la han afectado.

\begin{center}
\begin{longtable}{|>{\centering\arraybackslash}p{0.2\textwidth}|>{\centering\arraybackslash}p{0.2\textwidth}|>{\centering\arraybackslash}p{0.2\textwidth}|>{\centering\arraybackslash}p{0.2\textwidth}|}
\hline 
\bfseries {Etiqueta} & \bfseries {Tipo} & \bfseries {Uso} & \bfseries {Descripción} \\ 
\hline 
acctId & Cadena de caracteres (32+) & Requerido & ID de la cuenta a la cual pertenece el balance \\ 
\hline
transId & Entero (8) & Opcional & ID de la transacción que afectó el balance \\
\hline 
curAmt & Decimal & Requerido & Cantidad de dinero en la cuenta para la fecha \\
\hline 
effDt & Fecha & requerido & Fecha en la cual se afectó el balance \\
\hline 
descr & Cadena de caracteres & Opcional & Descripción \\
\hline 
\end{longtable}
\end{center}

\subsubsection{Transacción ``transaction''}
Cualquier movimiento que afecte algún balance.

\begin{center}
\begin{longtable}{|>{\centering\arraybackslash}p{0.2\textwidth}|>{\centering\arraybackslash}p{0.2\textwidth}|>{\centering\arraybackslash}p{0.2\textwidth}|>{\centering\arraybackslash}p{0.2\textwidth}|}
\hline 
\bfseries {Etiqueta} & \bfseries {Tipo} & \bfseries {Uso} & \bfseries {Descripción} \\ 
\hline
transId & Entero (8) & Opcional & ID de la transacción \\
\hline 
acctId & Cadena de caracteres (32+) & Requerido & ID de la cuenta que realizó la transacción \\ 
\hline 
acctOutFlow & Cadena de caracteres (32+) & Opcional (ver descripción) & ID de la cuenta a la cual se le debitará el dinero.
\\ & & & \\
& & & Este campo es requerido en caso de que la transacción debite de alguna forma dinero de la cuenta \\
\hline 
acctInFlow & Cadena de caracteres (32+) & Opcional (ver descripción) & ID de la cuenta que recibirá el dinero.
\\ & & & \\
& & & Este campo es requerido en caso de que la transacción abone de alguna forma dinero a la cuenta \\
\hline
thirdParty & Cadena de caracteres & Opcional (ver descripción) & ID de la cuenta o tarjeta de un beneficiario en otro banco.
\\ & & & \\
& & & Este campo es requerido en caso de existir un tercero involucrado \\
\hline
amt & Decimal & Requerido & Cantidad de dinero que moverá la transacción \\
\hline 
dueDt & Fecha & Requerido & Fecha en la cual se desea la transacción sea efectuada \\
\hline 
curDt & Fecha & Requerido & Fecha en la cual se realizó la transacción \\
\hline 
transType & Enumeración & Opcional & Tipo de transacción realizada \\
\hline 
\end{longtable}
\end{center}

\subsubsection{Transacción ``transaction''}
Cualquier movimiento que afecte algún balance.

\begin{center}
\begin{longtable}{|>{\centering\arraybackslash}p{0.2\textwidth}|>{\centering\arraybackslash}p{0.2\textwidth}|>{\centering\arraybackslash}p{0.2\textwidth}|>{\centering\arraybackslash}p{0.2\textwidth}|}
\hline 
\bfseries {Etiqueta} & \bfseries {Tipo} & \bfseries {Uso} & \bfseries {Descripción} \\ 
\hline
acctId & Cadena de caracteres (32+) & Requerido & ID de la cuenta que realizó la transacción \\ 
\hline 
bankId & Cadena de caracteres & Requerido & ID que identifica a la agencia bancaria \\
\hline 
custPermId & Cadena de caracteres (32+) & Requerido & ID permanente del cliente. Es asignado por la institución financiera para representar al cliente en el sistema 
\\ & & & \\
& & & Este campo es requerido si la cuenta no es compartida \\
\hline
acctType & Enumeración & Requerido & Define de qué tipo es la cuenta \\
\hline
freeForAll & Booleano & Opcional (ver descripción) & En caso de que la cuenta sea compartida, define si cualquiera puede efectuar operaciones o se necesita la aprobación de todos los participantes.
\\ & & & \\
& & & Es requerido si el campo ``custPermId'' no está definido \\
\hline 
members & Entero & Opcional (ver descripción) & Define cuantos clientes comparten la cuenta
\\ & & & \\
& & & El campo es requerido si ``freeForAll'' está definido \\
\hline 
\end{longtable}
\end{center}

\subsection{Recursos abstractos}
A diferencia de los recursos concretos, los recursos abstractos no representan objetos en la base de datos. Estos recursos ofrecen una información más general y son producto de una recopilación de información sobre varios objetos en la base de datos. También se pueden utilizar estos recursos para ofrecer una forma más sencilla de actualizar alguna información en el sistema. Por ejemplo, si se desea registrar un nuevo cliente, se tendrían que mandar varias peticiones para crear los recursos necesarios para que el cliente sea registrado. Para evitar mandar varias peticiones se podría ofrecer un recurso que recopile toda la información necesaria y este se encargue de crear los recursos concretos necesarios para que el nuevo cliente pueda ser registrado.
%\end{document}