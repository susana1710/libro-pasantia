\chapter*{Glosario}
\noindent
\textbf{Android:} Sistema operativo de código abierto basado en el núcleo de Linux, utilizado pricipalmente en dispositivos móviles.\\ 	\\
\textbf{\textit{Application Programming Interface} (API):} Conjunto de funciones y protocolos que ofrece una librería como capa de abstracción, para ser utilizados por otro \textit{software}.\\ \\
\textbf{Aplicación móvil:} \textit{Software} diseñado para ser ejecutado en dispositivos móviles, como teléfonos inteligentes y tabletas.\\ \\
\textbf{Aplicación web:} Herramientas que los usuarios pueden utilizar para acceder a un servidor web a través de Internet.\\ \\
\textbf{Cliente:} \textit{Software} o usuario que realiza peticiones de tareas a otros ordenadores que actúan como servidores.\\ \\
\textbf{\textit{Framework}:} Estructura conceptual y tecnológica que sirve de base para la organización e implementación de \textit{software}.\\ \\
\textbf{\textit{Hypertext Markup Language} (HTML):} Lenguaje utilizado para la elaboración de páginas web.\\ \\
\textbf{\textit{Integrated Development Environment} (IDE):} Aplicación informática que provee servicios que facilitan el desarrollo de \textit{software} al usuario.\\ \\
\textbf{Librería:} Conjunto de funciones, desarrolladas en un lenguaje de programación, que ofrecen una interfaz definida para la funcionalidad que se invoca.\\ \\
\textbf{Manejador de base de datos:} Colección de \textit{software} que sirve de interfaz entre la base de datos, el usuario y las aplicaciones utilizadas.\\ \\
\textbf{\textit{Object-Relational Maping} (ORM):} Técnica de programación utilizada para convertir datos entre sistemas de tipos incompatibles en lenguajes orientados a objetos.\\ \\
\textbf{\textit{Portable Dcument Format} (PDF):} Formato de almacenamiento de documentos digitales intependiente de plataformas de \textit{software} y \textit{hardware}.\\ \\
\textbf{Scrum:} Marco de desarrollo ágil que se caracteriza por adoptar una estrategia de desarrollo incremental e iterativa.\\ \\
\textbf{\textit{Software Development Kit} (SDK):} Conjunto de herramientas de desarrollo de \textit{software} que le permiten al programador crear aplicaciones para un sistema en concreto.\\ \\
\textbf{Servidor:} Computador en el que se ejecuta continuamente un \textit{Software} que realiza tareas para atender peticiones de un cliente.\\ \\
\textbf{Servidor web:} Computador en el que se ejecuta continuamente un \textit{Software}, al cual se hacen peticiones a través de Internet.\\ \\
\textbf{\textit{Simple Mail Transfer Protocol} (SMTP):} Protocolo de red que se utiliza para el intercambio de mensajes de correo electrónico entre dispositivos.\\ \\
\textbf{\textit{Structured Query language} (SQL):} Lenguaje declarativo que permite realizar operaciones sobre bases de datos relacionales.\\ \\
\textbf{\textit{Uniform Resource Locator} (URL):} Secuencia de caracteres que sigue un estándar y permite denominar recursos dentro del entorno de Internet para que pueden ser localizados.\\