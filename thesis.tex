% file thesis.tex
% Archivo thesis.tex
% Documento maestro que incluye todos los paquetes necesarios para el documento
% principal.

% Documento obtenido por un sinfin de iteraciones de administradores del LDC
% Estructura actual hecha por:
% Jairo Lopez <jairo@ldc.usb.ve>
% Actualizado ligeramente por:
% Alexander Tough 

\documentclass[oneside,12pt,letterpaper]{report}
\tolerance=1000  
\hbadness=10000  
\raggedbottom

% Para escribir algoritmos
\usepackage{listings}
\usepackage{algpseudocode}
\usepackage{algorithmicx}
\usepackage{algorithm}

\usepackage{pdflscape}

% Paquetes para manejar graficos
\usepackage{epsf}
\usepackage[pdftex]{graphicx}
\usepackage{epsfig}
% Simbolos matematicos
\usepackage{latexsym,amssymb}
% Paquetes para presentar una tesis decente.
\usepackage{setspace,cite} % Doble espacio para texto, espacio singular para
                           % los caption y pie de pagina

\usepackage[table]{xcolor}
\usepackage{tikz}
\usetikzlibrary{shapes.geometric,arrows}

\usetikzlibrary{arrows,shapes}
\usepackage{verbatim}

\usepackage{comment}

% Paquetes no utilizados para citas
%\usepackage{mcite} 
%\usepackage{draft} 

\usepackage{wrapfig}
\usepackage{alltt}

% Acentos 
\usepackage[spanish,activeacute,es-noquoting]{babel}

\usepackage[spanish]{translator}
\usepackage[utf8]{inputenc}
\usepackage{color, xcolor, colortbl}
\usepackage{multirow}
\usepackage{subfig}
\usepackage[OT1]{fontenc}
\usepackage{tocbibind}
\usepackage{anysize}
\usepackage{listings} 

% Para poder tener texto asiatico
%\usepackage{CJK}

\usepackage{pdfpages}

% Opciones para los glosarios
\usepackage[style=altlist,toc,numberline,acronym]{glossaries}
\usepackage{url}
\usepackage{amsthm}
\usepackage{amsmath}
\usepackage{fancyhdr} % Necesario para los encabezados
\usepackage{fancyvrb}
\usepackage{makeidx} % En caso de necesitar indices.
\makeindex  % Necesitado para los indices

% Definiciones para definicions, teoremas y lemas
\theoremstyle{definition} \newtheorem{definicion}{Definici\'{o}n}
\theoremstyle{plain} \newtheorem{teorema}{Teorema}
\theoremstyle{plain} \newtheorem{lema}{Lema}

% Para la creacion de los pdfs
\usepackage{hyperref}

% Para resolver el lio del Unicode para la informacion de los PDFs
% En pdftitle coloca el nombre de su proyecto de grado/pasantia.
% En pdfauthor coloca su nombre.
\hypersetup{
    pdftitle = {Desarrollo de una aplicación móvil de control y reportes de gastos},
    pdfauthor={Susana Charara Charara},
    colorlinks,
    citecolor=black,
    filecolor=black,
    linkcolor=black,
    urlcolor=black,
    backref,
    pdftex
}

\definecolor{brown}{rgb}{0.7,0.2,0}
\definecolor{darkgreen}{rgb}{0,0.6,0.1}
\definecolor{darkgrey}{rgb}{0.4,0.4,0.4}
\definecolor{lightgrey}{rgb}{0.95,0.95,0.95}

\usepackage{listings}
\lstnewenvironment{code}{\lstset{basicstyle=\small}}{}

\lstset{escapeinside=~~}
\lstset{
   frame=single,
   framerule=1pt,
   showstringspaces=false,
   basicstyle=\footnotesize\ttfamily,
   keywordstyle=\textbf,
   backgroundcolor=\color{lightgrey}
}

% Crea el glosario
%\makeglossaries

% Incluye el glosario
%\chapter*{Glosario}
\noindent
\textbf{Android:} Sistema operativo de código abierto basado en el núcleo de Linux, utilizado pricipalmente en dispositivos móviles.\\ 	\\
\textbf{\textit{Application Programming Interface} (API):} Conjunto de funciones y protocolos que ofrece una librería como capa de abstracción, para ser utilizados por otro \textit{software}.\\ \\
\textbf{Aplicación móvil:} \textit{Software} diseñado para ser ejecutado en dispositivos móviles, como teléfonos inteligentes y tabletas.\\ \\
\textbf{Aplicación web:} Herramientas que los usuarios pueden utilizar para acceder a un servidor web a través de Internet.\\ \\
\textbf{Cliente:} \textit{Software} o usuario que realiza peticiones de tareas a otros ordenadores que actúan como servidores.\\ \\
\textbf{\textit{Framework}:} Estructura conceptual y tecnológica que sirve de base para la organización e implementación de \textit{software}.\\ \\
\textbf{\textit{Hypertext Markup Language} (HTML):} Lenguaje utilizado para la elaboración de páginas web.\\ \\
\textbf{\textit{Integrated Development Environment} (IDE):} Aplicación informática que provee servicios que facilitan el desarrollo de \textit{software} al usuario.\\ \\
\textbf{Librería:} Conjunto de funciones, desarrolladas en un lenguaje de programación, que ofrecen una interfaz definida para la funcionalidad que se invoca.\\ \\
\textbf{Manejador de base de datos:} Colección de \textit{software} que sirve de interfaz entre la base de datos, el usuario y las aplicaciones utilizadas.\\ \\
\textbf{\textit{Object-Relational Maping} (ORM):} Técnica de programación utilizada para convertir datos entre sistemas de tipos incompatibles en lenguajes orientados a objetos.\\ \\
\textbf{\textit{Portable Dcument Format} (PDF):} Formato de almacenamiento de documentos digitales intependiente de plataformas de \textit{software} y \textit{hardware}.\\ \\
\textbf{Scrum:} Marco de desarrollo ágil que se caracteriza por adoptar una estrategia de desarrollo incremental e iterativa.\\ \\
\textbf{\textit{Software Development Kit} (SDK):} Conjunto de herramientas de desarrollo de \textit{software} que le permiten al programador crear aplicaciones para un sistema en concreto.\\ \\
\textbf{Servidor:} Computador en el que se ejecuta continuamente un \textit{Software} que realiza tareas para atender peticiones de un cliente.\\ \\
\textbf{Servidor web:} Computador en el que se ejecuta continuamente un \textit{Software}, al cual se hacen peticiones a través de Internet.\\ \\
\textbf{\textit{Simple Mail Transfer Protocol} (SMTP):} Protocolo de red que se utiliza para el intercambio de mensajes de correo electrónico entre dispositivos.\\ \\
\textbf{\textit{Structured Query language} (SQL):} Lenguaje declarativo que permite realizar operaciones sobre bases de datos relacionales.\\ \\
\textbf{\textit{Uniform Resource Locator} (URL):} Secuencia de caracteres que sigue un estándar y permite denominar recursos dentro del entorno de Internet para que pueden ser localizados.\\

% Para crear la hoja escaneada de las firmas
\usepackage[absolute]{textpos}

% Pone los nombres y las opciones para mostrar los codigos fuentes
\lstset{language=C, breaklines=true, frame=single, showstringspaces=false,
        showtabs=false, numbers=left, keywordstyle=\color{black},
        basicstyle=\footnotesize, captionpos=b }
\renewcommand{\lstlistingname}{C\'{o}digo fuente}
\renewcommand{\lstlistlistingname}{\'{I}ndice de c\'{o}digos fuentes}

\newcommand{\todo}{ TODO: }

% Dimensiones de la pagina
\setlength{\headheight}{15pt}
\marginsize{3cm}{2cm}{2cm}{2cm}

%%%%%%%%%%%%%%%%%%%%%%%%%%%%%%%%%%%%%%%%%%%%%%%%%%%%%%%%%%%%%%%%%%%%%%%%%%%
%%%%%%%%%%%%%%%%      end of preamble and start of document     %%%%%%%%%%%
%%%%%%%%%%%%%%%%%%%%%%%%%%%%%%%%%%%%%%%%%%%%%%%%%%%%%%%%%%%%%%%%%%%%%%%%%%%
\begin{document}

% Pagina de titulo
% Pagina de titulo
\begin{titlepage}
\begin{center}

% Upper part (aqui ya esta incluido el logo de la USB).
\includegraphics[scale=0.5,type=png,ext=.png,read=.png]{imagenes/cebolla} \\

% Encabezado
\textsc {\large UNIVERSIDAD SIMÓN BOLÍVAR} \\
\textsc{\bfseries DECANATO DE ESTUDIOS PROFESIONALES\\
COORDINACI'ON DE INGENIER'IA DE LA COMPUTACI'ON}

\bigskip
\bigskip
\bigskip
\bigskip
\bigskip
\bigskip
\bigskip
\bigskip
\bigskip

% Title/Titulo
% Aqui ponga el nombre de su proyecto de grado/pasantia larga
\textsc{\bfseries DESARROLLO DE UNA APLICACIÓN MÓVIL DE CONTROL Y REPORTES DE GASTOS}

\bigskip
\bigskip
\bigskip
\bigskip
\bigskip

% Author and supervisor/Autor y tutor
\begin{minipage}{\textwidth}
\centering
Por: \\ SUSANA CHARARA CHARARA \\

\bigskip
\bigskip
\bigskip

Realizado con la asesoría de: \\
Tutor Académico: PROF. XIOMARA CONTRERAS \\
Tutor Industrial: LIC. LUIS AUGUSTO PEÑA PEREIRA
\end{minipage}

\bigskip
\bigskip
\bigskip
\bigskip
\bigskip
\bigskip
\bigskip
\bigskip
\bigskip

% Bottom half
{INFORME DE PASANTÍA LARGA \\ Presentado ante la Ilustre Universidad Simón Bolívar \\
como requisito parcial para optar al título de \\ Ingeniero en Computación} \\

\bigskip
\bigskip
\vfill

% Date/Fecha 
{\large \bfseries Sartenejas, 
%FECHA
OCTUBRE de 2016}

\end{center}
\end{titlepage}

% Pagina de titulo
\begin{titlepage}
\begin{center}

% Upper part (aqui ya esta incluido el logo de la USB).
\includegraphics[scale=0.5,type=png,ext=.png,read=.png]{imagenes/cebolla} \\

% Encabezado
\textsc {\large UNIVERSIDAD SIMÓN BOLÍVAR} \\
\textsc{\bfseries DECANATO DE ESTUDIOS PROFESIONALES\\
COORDINACI'ON DE INGENIER'IA DE LA COMPUTACI'ON}

\bigskip
\bigskip
\bigskip
\bigskip
\bigskip
\bigskip
\bigskip
\bigskip
\bigskip

% Title/Titulo
% Aqui ponga el nombre de su proyecto de grado/pasantia larga
\textsc{\bfseries DESARROLLO DE UNA APLICACIÓN MÓVIL DE CONTROL Y REPORTES DE GASTOS}

\bigskip
\bigskip
\bigskip
\bigskip
\bigskip

% Author and supervisor/Autor y tutor
\begin{minipage}{\textwidth}
\centering
Por: \\ SUSANA CHARARA CHARARA \\

\bigskip
\bigskip
\bigskip

Realizado con la asesoría de: \\
Tutor Académico: PROF. XIOMARA CONTRERAS \\
Tutor Industrial: LIC. LUIS AUGUSTO PEÑA PEREIRA
\end{minipage}

\bigskip
\bigskip
\bigskip
\bigskip
\bigskip
\bigskip
\bigskip
\bigskip
\bigskip

% Bottom half
{INFORME DE PASANTÍA LARGA \\ Presentado ante la Ilustre Universidad Simón Bolívar \\
como requisito parcial para optar al título de \\ Ingeniero en Computación} \\

\bigskip
\bigskip
\vfill

% Date/Fecha 
{\large \bfseries Sartenejas, 
%FECHA
OCTUBRE de 2016}

\end{center}
\end{titlepage}

% Pagina de acta final (vacio)
%\includepdf[pages={1}]{intro/firmas.pdf}

%\setcounter{secnumdepth}{3}
%\setcounter{tocdepth}{4}

% Define encabezado numeros romanos y como se separan los captiulos y las
% secciones
\addtolength{\headheight}{3pt}
\pagenumbering{roman}
\pagestyle{fancyplain}

\renewcommand{\chaptermark}[1]{\markboth{\chaptername\ \thechapter:\,\ #1}{}}
\renewcommand{\sectionmark}[1]{\markright{\thesection\,\ #1}}

\onehalfspacing

\lhead{}
\chead{}
\rhead{}
\renewcommand{\headrulewidth}{0.0pt}
\lfoot{}
\cfoot{\fancyplain{}{\thepage}}
\rfoot{}


% Pagina de resumen
%\setcounter{page}{3}
\begin{center}
	{\bf Resumen} \pdfbookmark[0]{Resumen}{resumen} % Sets a PDF bookmark for the dedication
\end{center}	

En esta pasantía se desarrolló, para la empresa Digitalica Group, C.A., una aplicación para dispositivos móviles con la cual se pueden realizar reportes de gastos. Actualmente, la empresa Digitalica Group, C.A. ofrece a sus trabajadores el beneficio de realizar reembolsos de gastos en ciertos rubros; el proyecto nació como solución al problema de agilizar el proceso en que los trabajadores hacen llegar a los supervisores la información de estos gastos. Con la aplicación desarrollada, se pueden registrar gastos e ingresos con un monto, fecha y descripción. Se permite también capturar y guardar fotos tanto de los ingresos como de los gastos, de manera que el usuario puede tomar fotos de los recibos de sus gastos. Además, estos gastos/ingresos pueden estar asociados a categorías, que indican el rubro al que pertenecen. La aplicación permite al usuario generar archivos con formato PDF que contienen un reporte con una lista de gastos, en un rango de fecha determinado; estos reportes se pueden enviar a un servidor web. Este servidor también fue desarrollado durante la pasantía. Se creó un servicio web que recibe credenciales de un usuario y las valida; éste es utilizado por la aplicación móvil para autenticar usuarios. Asimismo, se desarrolló un servicio web que recibe archivos PDF con reportes de gastos, el cual es utilizado por la aplicación para enviar reportes. Igualmente, se desarrollaron servicios y una aplicación web, con la cual los supervisores de la empresa pueden revisar y descargar los archivos de los reportes recibidos por el servidor, aprobarlos y rechazarlos.

En este informe se describen los conceptos teóricos que ayudaron al diseño y desarrollo de la solución. Igualmente, se describe el proceso de desarrollo de los tres componentes principales desarrollados durante la pasantía: aplicación nativa para Android, servidor web y aplicación web. El desarrollo de estos involucró el diseño e implementación de los modelos de datos de la aplicación y el servidor, así como múltiples interfaces de usuario que permiten la interacción con los modelos. Para cada componente, se describen el entorno de trabajo y las herramientas utilizadas. Se utilizó Java como lenguaje de programación tanto de la aplicación móvil como del servidor; entre las herramientas que se emplearon en el desarrollo del servidor se pueden mencionar AppFuse, Spring, Hibernate y Tapestry, entre otros.

Por otra parte, se utilizó Scrum como marco de trabajo, dividiendo el desarrollo del proyecto en ocho iteraciones, en las que se implementaron los componentes y funcionalidades necesarios para el cumplimiento de los objetivos de la pasantía.

El objetivo del desarrollo de la pasantía fue ofrecer una solución al problema que actualmente se presenta en la empresa de agilizar el proceso de reembolso sobre ciertos gastos.




% Pagina de dedicatoria (opcional)
%\pagebreak

%\setcounter{page}{5}

\vspace*{8cm} 
\pdfbookmark[0]{Dedicatoria}{dedicatoria} % Sets a PDF bookmark for the dedication
\begin{center} 
\large A nuestros padres.\\ Porque nos dieron la vida y nos han guiado 
a ser quienes somos hoy.
\end{center}
\newpage


% Pagina de agradecimientos (opcional)
%\input{intro/agradecimientos.tex}

% Crea la tabla de contenidos
%\tableofcontents

% Crea la lista de cuadros
%\listoftables

% Crea la lista de figuras
%\listoffigures
%\newpage
\phantomsection
%%\setcounter{page}{4}
\chapter*{Lista de Símbolos y Abreviaturas}% Sets a PDF bookmark for the dedication
\textbf{CEO:} Chief Executive Officer (en español Director Ejecutivo)\\ \\
\textbf{COO:} Chief Operations Officer (en español Director de Operaciones)\\ \\
\textbf{CTO:} Chief Technology Officer (en español Director de Tecnología)\\ \\
\textbf{DAO:} Data Access Object (en español Objeto de Acceso a Datos)\\ \\
\textbf{DVCS:} Distributed Version Control System (en español Sistema de Control de Versiones Distribuido)\\ \\
\textbf{DVM:} Dalvik Virtual Machine (en español OMáquina Virtual Dalvik)\\ \\
\textbf{HTTP:} HyperText Transfer Protocol (en español Protocolo de Transferencia de Hipertexto)\\ \\
\textbf{IDE:} Integrated Development Environment (en español Entorno de Desarrollo Integrado)\\ \\
\textbf{JSON:} JavaScript Object Notation (en español Notación de Objetos de JavaScript)\\ \\
\textbf{JVM:} Java Virtual Machine(en español OMáquina Virtual de Java)\\ \\
\textbf{MVC:} Model View Controller (en español Modelo Vista Controlador)\\ \\
\textbf{MVP:} Model View Presenter (en español Modelo Vista Presentador)\\ \\
\textbf{ORM:} Object-Relational Mapping (en español Mapeo Objeto-Relacional)\\ \\
\textbf{PDF:} Portable Document Format (en español Formato de Documento Portátil)\\ \\
\textbf{POJO:} Plain Old Java Object (en español Objeto de Java Plano Antiguo)\\ \\
\textbf{SQL:} Structured Query Language (en español Lenguaje de Consulta Estructurada)\\ \\

%\addcontentsline{toc}{chapter}{Lista de Símbolos y Abreviaturas}

% Crea la lista de codigos fuentes
%\lstlistoflistings

\clearpage

% Define encabezado en numeros arabicos  
\pagenumbering{arabic}

\fancyhf{} % Redefine el encabezado 
\lhead{}
\chead{}
\rhead{\fancyplain{}{\thepage}}
\renewcommand{\headrulewidth}{0.0pt}
\lfoot{}
\cfoot{}
\rfoot{}

\doublespacing

% Incluye los archivos deseados - El contenido de su proyecto de grado/pasantia larga.
\phantomsection
\addcontentsline{toc}{chapter}{Introducción}
%\chapter*{Introducción} \label{sec:Introduccion}
%\pdfbookmark[0]{Introducción}{introduccion} % Sets a PDF bookmark for the dedication

\vspace{5 mm}

En la actualidad, muchas empresas le ofrecen a sus empleados el beneficio de realizar gastos personales y posteriormente iniciar un proceso de reembolso. Este proceso puede tomar mucho tiempo y puede resultar difícil si se hace manualmente, pues implica que el trabajador deba guardar todas las facturas o recibos de los gastos implicados. También dificulta la organización de estos registros por parte de la empresa, porque al mantener todos los recibos en físico, estos se pueden extraviar.

Este problema llevó a la conclusión de que es necesario agilizar todo el proceso, y dada la facilidad de acceso que se tiene hoy en día a un dispositivo móvil inteligente, se decidió crear una aplicación que permita mantener el registro de gastos. 

Actualmente existen aplicaciones que sirven para llevar un registro de gastos. Sin embargo, estas aplicaciones no satisfacen las necesidades de la empresa por diversas razones:

\begin{itemize}
\item En primer lugar, la empresa realiza el reembolso de gastos en ciertos rubros y para esto se debe especificar a qué categoría pertenece cada gasto. Las aplicaciones ya existentes pueden limitar el proceso al no contar con la posibilidad de poder asociar un gasto a un rubro cubierto por la empresa.
\item Se desea crear archivos de reportes de los gastos registrados por los trabajadores. Se desea que estos reportes puedan ser personalizados y presenten una estructura particular.
\item Se desea mantener toda la información referente a estos reportes de manera centralizada de manera que se pueda acceder a la misma de una manera más fácil.
\item Si el proceso de reembolso sufre alguna modificación, se desea contar con una aplicación que tome en cuenta los nuevos cambios.
\end{itemize}

Por las razones expuestas anteriormente, se tomó la decisión de crear una aplicación móvil que permita llevar un registro de gastos. Esta aplicación debe contar con las siguientes funcionalidades:

\begin{itemize}
\item Registrar gastos con su fecha, monto, una breve descripción, fotos y categoría a la que pertenecen
\item Organizar y consultar gastos con criterios de búsqueda pre-establecidos
\item Crear reportes personalizados con los gastos y enviar dichos reportes a un servidor
\item Enviar reportes mediante archivos con formato PDF a los supervisores
\item Controlar el estado de aprobación de los reportes
\end{itemize}

En este informe se describe el proceso de desarrollo de una aplicación móvil para el registro de gastos.

El informe se estructura de la siguiente manera: En el Capítulo 1 se describe de una forma general la empresa; en el Capítulo 2 se definen los conceptos teóricos estudiados para el desarrollo del proyecto; en el Capítulo 3 se mencionan las herramientas y tecnologías que facilitaron el desarrollo; en el Capítulo 4 se describe brevemente el marco de trabajo utilizado, Scrum; en el Capítulo 5 se describen todas las fases involucradas tanto en el diseño como el desarrollo de la solución; en el capítulo 6 se exponen las conclusiones y recomendaciones que surgieron luego de la investigación y desarrollo del proyecto; finalmente, se muestran las referencias bibliográficas consultadas.
%
%% Entorno empresarial.
\chapter{Entorno Empresarial} \label{chap:Entorno Empresarial}

\vspace{5 mm}


% % Marco Teorico.
\chapter{Marco Teórico} \label{chap:Marco Teorico}

En este capítulo se exponen los conceptos teóricos estudiados, tanto para entender las tecnologías utilizadas como para realizar el desarrollo de \textit{software} requerido. Los conceptos estudiados formaron parte fundamental del proceso de diseño y desarrollo de la solución.

%\input{marco_teorico/1_Web2.tex}
%\input{marco_teorico/2_SOA.tex}
%\input{marco_teorico/3_REST.tex}
%\input{marco_teorico/4_ROA.tex}
%\input{marco_teorico/5_EDP.tex}
%\input{marco_teorico/6_DBMS.tex}

% % Marco Teorico.
\chapter{Marco Tecnológico} \label{chap:Marco Tecnologico}
En este capítulo se definen las diferentes herramientas utilizadas a lo largo de todo el proyecto. Dichas herramientas permitieron el desarrollo del \textit{software} planteado como solución al problema.
\vspace{5 mm}



%\section{Twisted} \label{sect:Twisted}


%\section{Couchbase} \label{sect:Couchbase}

%\section{PostgreSQL} \label{sect:PostgreSQL}


%\section{Elasticsearch} \label{sect:Elasticsearch}

%\section{Logstash} \label{sect:Logstash}

%\section{Kibana} \label{sect:Kibana}


%\section{NGINX} \label{sect:NGINX}

%\section{Xen Project} \label{sect:Xen Project}



 % Marco Metodologico.
\chapter{Marco Metodológico} \label{chap:Marco Metodologico}

A continuación se describe el procedimiento seguido para el desarrollo del proyecto. Se decidió utilizar el marco de trabajo Scrum, dado que éste es utilizado por la empresa para el desarrollo de \textit{software}.
\section{Scrum} \label{sect:Scrum}

Scrum es un marco de trabajo que se basa en el desarrollo iterativo e incremental de un producto, en lugar del modelo clásico de planificación y ejecución completa. \cite{SCRM0} Se caracteriza por ser una metodología ligera, fácil de entender y difícil de dominar, que permite entregar incrementos de producto potencialmente productivos. \cite{SCRM1}

\subsection{Roles} 

En Scrum el desarrollo se realiza por uno o más equipos de trabajo dentro de los cuales existen tres roles: \textit{Product owner} (jefe del producto), \textit{ScrumMaster} (jefe de Scrum) y el equipo de desarrollo. \cite{SCRM12}
 
\subsubsection{\textit{Product owner}}

Es el representante de los clientes. Dentro del equipo de Scrum, es el líder principal del producto y el responsable de decidir qué funcionalidades serán desarrolladas y la prioridad que tendrá cada una de ellas. Debe comunicar al resto de los involucrados en el proyecto una visión clara de lo que se quiere lograr. Tiene la obligación de asegurar que siempre se entregue un producto con el máximo de valor, por lo que debe colaborar con el resto del equipo para responder cualquier duda que surja. \cite{SCRM12}

\subsubsection{\textit{ScrumMaster}}

Actúa como facilitador tanto para el \textit{product owner} como para el equipo de desarrollo. Es el encargado de ayudar al resto del equipo a entender y cumplir con los principios y prácticas de Scrum.También tiene la responsabilidad de eliminar cualquier impedimento que el equipo no sea capaz de resolver y que afecte su productividad. \cite{SCRM12}

\subsubsection{Equipo de desarrollo}

Es el encargado de desarrollar el producto. Es un equipo que está compuesto por arquitectos, programadores, probadores, administradores de base de datos, diseñadores de interfaces, entre otros. Son los responsables de diseñar, desarrollar y probar el producto. \cite{SCRM12}

\subsection{Actividades}

En Scrum, el trabajo se desarrolla en interaciones de una duración máxima de un mes, llamadas \textit{\textbf{sprints}}. Al final de cada \textit{sprint}, se debe haber desarrollado una parte del producto final, la cual debe ser completamente funciona. Dentro de cada iteración existe una serie de eventos o actividades que se llevan a cabo: el \textit{sprint planning}, la ejecución del \textit{sprint}, el \textit{daily scrum} y el \textit{sprint review}.\cite{SCRM12}

\subsubsection{\textit{Sprint planning}}s
Para determinar qué funcionalidades del producto final son las más importantes y próximas a desarrollar, el equipo de trabajo (\textit{product owner}, \textit{ScrumMaster} y el equipo de desarrollo) realizan una reunión llamada \textit{sprint planning}.\cite{SCRM12}

Durante la reunión, el \textit{product owner} y el equipo de desarrollo establecen una meta que debe ser cumplida para el final del \textit{sprint}. De acuerdo a esta meta, el equipo de desarrollo decide de una manera realista qué incrementos del producto final pueden entregarse al terminar el \textit{sprint}.\cite{SCRM12}

\subsubsection{Ejecución del \textit{sprint}}

Luego del \textit{sprint planning}, el equipo de desarrollo desarrolla todas las tareas acordadas en la reunión. Esto es lo que se conoce como la ejecución del \textit{sprint}.\cite{SCRM12}

\subsubsection{\textit{Daily scrum}}

Cada día dentro de la ejecución del \textit{sprint}, los miembros del equipo de desarrollo se reúnen durante un máximo de 15 minutos con el fin de informar qué se hizo el día anterior, qué se tiene planificado realizar el presente día y qué impedimentos se han presentado durante el desarrollo de su trabajo.\cite{SCRM12}

\subsubsection{\textit{Sprint review}}

Al final de cada \textit{sprint}, ocurre un evento que se conoce como \textit{sprint review} o revisión del \textit{sprint}. El objetivo de esta actividad es revisar el incremento y realizar las adaptaciones necesarias al prouducto.\cite{SCRM12}

\subsection{Artefactos}

Dentro de Scrum existen dos herramientas o artefactos que permiten mantener un seguimiento del proyecto: el \textit{product backlog}y el \textit{sprint backlog}.\cite{SCRM2}

\subsubsection{\textit{Product backlog}}

Es una lista de los requerimientos funcionales del producto ordenados según su importancia. EL \textit{product owner} es el responsable de definir qué elementos serán incluidos en esta lista y de colocarlos según su prioridad, de manera que los elementos de mayor valor o prioridad aparezcan al principio de la lista, y los de menos valor al final de la misma. \cite{SCRM2}

\subsubsection{\textit{Sprint backlogs}}

Es una lista donde se presenta un subconjunto de los elementos del \textit{product backlog} divididos en tareas más pequeñas. \cite{SCRM2}



\section{Aplicación de Scrum en el desarrollo del proyecto} \label{sect:Scrum}

%\chapter{Diseño de la Solución}\label{chapter:Diseno de la solucion}


%\section{Investigacion} \label{sect:Investigacion}

El objetivo de esta primera fase era familiarizarse con el entorno de trabajo. En primer lugar se investigó lo necesario para conocer la metodología ágil \textit{Scrum} y poder implementarla en el desarrollo del proyecto. También se aprendió a trabajar con el manejador de versiones Git y se presentaron los lineamientos para el uso del mismo dentro de la empresa. Por otra parte, se estudiaron diversos patrones de diseño y cómo podrían ser empleados para la estucturación de la aplicación.

Por último, se realizó una aplicación de prueba para 



%\chapter{Desarrollo}\label{chapter:desarrollo}


%\input{pruebas/0_pruebas.tex}
%
%\chapter{Retos Enfrentados y Logros Adicionales} 

%
%\chapter{Conclusiones y Recomendaciones} \label{chap:conclusiones}

En este proyecto de pasantía, realizado para la empresa Digitalica Group C.A, se diseñó y desarrolló un prototipo funcional de una aplicación móvil para el registro de gastos e ingresos, así como un servidor web que se comunica con ella.  La aplicación fue desarrollada para el sistema operativo Android. Con ella, se permite registrar ingresos y gastos, a los cuales se puede asociar un monto, fecha, descripción, fotos y categoría. El servidor web permite hacer la autenticación de usuarios, así como el envío de reportes compuestos por un conjunto de gastos.

El uso de Scrum como marco de trabajo permitió tener 



% Crea el glosario 
%\makeglossaries
%\printglossaries

% Establece las citas y bibliografia
\bibliographystyle{ieeetr}
\bibliography{myrefs}

% Crea el apendice
%\appendix
%\includepdf[pages=-]{apendices/IFFv1.1/IFF.pdf}
%%\documentclass[12pt,letterpaper]{article}
%\usepackage[utf8]{inputenc}
%\usepackage{amsmath}
%\usepackage{amsfonts}
%\usepackage{amssymb}
%\usepackage[spanish]{babel}
%\usepackage{array}
%\usepackage{longtable}
%\author{Jon Ander Ricchiuti}
%\title{IFF v1.1}
%\begin{document}
%\pagenumbering{arabic}
%\maketitle
%\thispagestyle{empty}
%\newpage
\chapter{Intercambio de Información Financiera (IFF)}

\section*{Introducción}

El protocolo de intercambio de información financiera (IFF), describe la forma correcta de comunicación con un prototipo de sistema bancario creado en Synergy Global Business (SGB). El IFF busca estandarizar el intercambio de información financiera que realizará el prototipo bancario. De esta manera la comunicación es sencilla y a la vez robusta. Para la realización de este protocolo de comunicación se utilizó como base el IFX (Interactive Financial Exchange).
\\
\\ 
La forma de comunicación que el IFF	utiliza es de tipo petición-respuesta (request-response). En cada petición se debe especificar un método. Este método define la naturaleza de la petición.
%\pagenumbering{arabic}
%\newpage

\section{Tipo de datos}

Los tipos de datos que se utilizarán en este estándar son los siguientes:
\begin{itemize}
\item Cadena de caracteres.
\item Enumeración.
\item Tiempo y Hora.
\item Entero.
\item Decimal
\item Booleano.
\end{itemize}

\subsection{Cadena de caracteres}
Las cadenas de caracteres se representan con el nombre de ``Cadena de caracteres'' seguido de la longitud de la misma. Esta longitud es representada entre paréntesis de la siguiente forma.
\begin{itemize}
\item Cadena de caracteres (X-Y), indica que la longitud mínima de la cadena de caracteres es ``X'' y la máxima longitud es ``Y''.
\item Cadena de caracteres (X+), indica que la longitud mínima de la cadena de caracteres es ``X'' pero no tiene longitud máxima para la misma.
\end{itemize}
Si la longitud no es especificada entonces no existe restricción sobre el tamaño de la cadena de caracteres.

\subsection{Enumeración}
Son los valores que puede tomar un campo. Estos pertenecen a un conjunto de cadena de caracteres definidas específicamente para ese campo en particular. Los diferentes tipos de enumeración serán especificados en la siguiente sección.

\subsection{Tiempo y Hora}
Tanto la hora como la fecha son cadenas de caracteres que se representan con un fromato particular. Para la hora el formto es ``\%H:\%M:\%S''. Para la fecha el formato es ``\%Y-\%m-\%d \%H:\%M:\%S''. Donde \%Y representa el año, \%m el mes, \%d el dia, \%H la hora, \%M el minuto, \%S el segundo. Todos los componentes de la hora y fecha se representan con caracteres numéricos.

\subsection{Entero}
Es un número entero y puede representarse de dos formas diferentes. Como un entero de cuatro bytes o como un entero de ocho bytes. Para especificar que es un entero de ocho bytes, debe ser escrito de la siguiente forma: Entero(8).
\\
\\
Si no se especifica que un entero es de ocho bytes entonces se asume que es de cuatro  bytes.

\subsection{Decimal}
Es un número con hasta quince dígitos decimales.

\subsection{Booleano}
Un Booleano representa si una condición se cumple o no. En el caso del IFF un Booleano será representado por medio de un carácter. Es decir, el carácter 'T' será el que representa cuando un estado es cierto y 'F' será el valor de cuando el estado no se cumple.

\section{Tipos de Enumeración}
\subsection{Correspondiente a personVerifyType}
\begin{center}
\begin{tabular}{|>{\centering\arraybackslash}p{0.3\textwidth}|>{\centering\arraybackslash}p{0.3\textwidth}|>{\centering\arraybackslash}p{0.3\textwidth}|}
\hline 
\bfseries {Valor} & \bfseries {Descripción} & \bfseries {Por defecto} \\ 
\hline 
Passport & Pasaporte de crédito & N \\ 
\hline 
CI & Cedula de identidad & N \\
\hline 
\end{tabular} 
\end{center}

\subsection{Correspondiente a nameAddrType}
\begin{center}
\begin{tabular}{|>{\centering\arraybackslash}p{0.3\textwidth}|>{\centering\arraybackslash}p{0.3\textwidth}|>{\centering\arraybackslash}p{0.3\textwidth}|}
\hline 
\bfseries {Valor} & \bfseries {Descripción} & \bfseries {Por defecto} \\ 
\hline 
Customer & Es la dirección del cliente & N \\ 
\hline 
ShipTo & Dirección a la cual algo debería ser enviado por correo & N \\
\hline 
Delivery & Dirección a la cual serán enviadas las facturas en papel & N \\
\hline 
\end{tabular} 
\end{center}

\subsection{Correspondiente a addrType}
\begin{center}
\begin{tabular}{|>{\centering\arraybackslash}p{0.3\textwidth}|>{\centering\arraybackslash}p{0.3\textwidth}|>{\centering\arraybackslash}p{0.3\textwidth}|}
\hline 
\bfseries {Valor} & \bfseries {Descripción} & \bfseries {Por defecto} \\ 
\hline 
Seasonal & Habitación vacacional & N \\ 
\hline 
Primary & Habitación principal & N \\
\hline 
Secondary & Habitación secundaria & N \\
\hline
Business & Dirección de negocio & N \\
\hline 
\end{tabular} 
\end{center}

\subsection{Correspondiente a cardStatusCode}
\begin{center}
\begin{tabular}{|>{\centering\arraybackslash}p{0.3\textwidth}|>{\centering\arraybackslash}p{0.3\textwidth}|>{\centering\arraybackslash}p{0.3\textwidth}|}
\hline 
\bfseries {Valor} & \bfseries {Descripción} & \bfseries {Por defecto} \\ 
\hline 
Active & Activa & N \\ 
\hline 
Expired & Vencida & N \\
\hline 
Blocked & Bloqueada & N \\
\hline
\end{tabular} 
\end{center}

\subsection{Correspondiente a accountStatusCode}
\begin{center}
\begin{tabular}{|>{\centering\arraybackslash}p{0.3\textwidth}|>{\centering\arraybackslash}p{0.3\textwidth}|>{\centering\arraybackslash}p{0.3\textwidth}|}
\hline 
\bfseries {Valor} & \bfseries {Descripción} & \bfseries {Por defecto} \\ 
\hline 
Active & Activa & N \\ 
\hline 
Blocked & Bloqueada & N \\
\hline
\end{tabular} 
\end{center}

\subsection{Correspondiente a cardType}
\begin{center}
\begin{tabular}{|>{\centering\arraybackslash}p{0.3\textwidth}|>{\centering\arraybackslash}p{0.3\textwidth}|>{\centering\arraybackslash}p{0.3\textwidth}|}
\hline 
\bfseries {Valor} & \bfseries {Descripción} & \bfseries {Por defecto} \\ 
\hline 
Credit & Tarjeta de crédito & N \\ 
\hline 
Debit & Tarjeta de débito & N \\
\hline 
\end{tabular} 
\end{center}

\subsection{Correspondiente a brand}
\begin{center}
\begin{tabular}{|>{\centering\arraybackslash}p{0.3\textwidth}|>{\centering\arraybackslash}p{0.3\textwidth}|>{\centering\arraybackslash}p{0.3\textwidth}|}
\hline 
\bfseries {Valor} & \bfseries {Descripción} & \bfseries {Por defecto} \\ 
\hline 
Visa &  & N \\ 
\hline 
MasterCard &  & N \\
\hline 
\end{tabular} 
\end{center}

\subsection{Correspondiente a transType}
\begin{center}
\begin{tabular}{|>{\centering\arraybackslash}p{0.3\textwidth}|>{\centering\arraybackslash}p{0.3\textwidth}|>{\centering\arraybackslash}p{0.3\textwidth}|}
\hline 
\bfseries {Valor} & \bfseries {Descripción} & \bfseries {Por defecto} \\ 
\hline 
Withdrawal & Retiro & N \\ 
\hline 
Deposit & Deposito & N \\
\hline 
Transference & Transferencia & N \\
\hline
\end{tabular} 
\end{center}

\subsection{Correspondiente a acctType}
\begin{center}
\begin{tabular}{|>{\centering\arraybackslash}p{0.3\textwidth}|>{\centering\arraybackslash}p{0.3\textwidth}|>{\centering\arraybackslash}p{0.3\textwidth}|}
\hline 
\bfseries {Valor} & \bfseries {Descripción} & \bfseries {Por defecto} \\ 
\hline 
Saving & Ahorro & N \\ 
\hline 
Current & Corriente & N \\
\hline 
Loan & Préstamo & N \\
\hline
\end{tabular} 
\end{center}

\subsection{Correspondiente a contactInfo}
\begin{center}
\begin{tabular}{|>{\centering\arraybackslash}p{0.3\textwidth}|>{\centering\arraybackslash}p{0.3\textwidth}|>{\centering\arraybackslash}p{0.3\textwidth}|}
\hline 
\bfseries {Valor} & \bfseries {Descripción} & \bfseries {Por defecto} \\ 
\hline 
dayPhone & Teléfono de contacto durante el día & N \\
\hline 
evePhone & Teléfono de contacto durante la tarde & N \\
\hline 
dayFax & Fax de contacto durante el día & N \\
\hline
eveFax & Fax de contacto durante la tarde & N \\
\hline
emailAddr & Dirección de correo electrónica & N \\
\hline
\end{tabular} 
\end{center}
	
\section{Recursos}
El protocolo IFF esta basado en recursos. Los recursos son fuentes de información sobre las cuales se realizan las peticiones. Estos recursos son divididos en dos grandes grupos, los concretos y los abstractos.

\subsection{Recursos concretos}
Los recursos concretos son la representación directa del modelo de datos que expone el core bancario para ofrecer sus servicios. Este tipo de recursos es muy sencillo y son los que permiten realizar las operaciones más básicas. \\

A continuación se presentan los recursos concretos.

\subsubsection{Nombre de usuario ``login''}
Contiene la información que relaciona el nombre de usuario electrónico con su información  en la institución financiera.

\begin{center}
\begin{tabular}{|>{\centering\arraybackslash}p{0.2\textwidth}|>{\centering\arraybackslash}p{0.2\textwidth}|>{\centering\arraybackslash}p{0.2\textwidth}|>{\centering\arraybackslash}p{0.2\textwidth}|}
\hline 
\bfseries {Etiqueta} & \bfseries {Tipo} & \bfseries {Uso} & \bfseries {Descripción} \\ 
\hline 
username & Cadena de caracteres (6-20) & Requerido & ID de ingreso del cliente \\ 
\hline 
password & Cadena de caracteres (6+) & Opcional & Clave de ingreso \\ 
\hline 
custPermId & Cadena de caracteres (32+) & Requerido & ID permanente del cliente. Es asignado por la institución financiera para representar al cliente en el sistema \\ 
\hline 
\end{tabular}
\end{center}

\subsubsection{Cliente ``customer''}
Tiene la información que identifica inequívocamente a un cliente.

\begin{center}
\begin{longtable}{|>{\centering\arraybackslash}p{0.25\textwidth}|>{\centering\arraybackslash}p{0.2\textwidth}|>{\centering\arraybackslash}p{0.15\textwidth}|>{\centering\arraybackslash}p{0.2\textwidth}|}
\hline 
\bfseries {Etiqueta} & \bfseries {Tipo} & \bfseries {Uso} & \bfseries {Descripción} \\ 
\hline 
custPermId & Cadena de caracteres (32+) & Opcional & ID permanente del cliente. Es asignado por la institución financiera para representar al cliente en el sistema \\ 
\hline 
personId & Cadena de caracteres (32+) & Opcional & Relación al objeto ``person'' \\ 
\hline 
custLogin & Cadena de caracteres (6-20) & Opcional & ID permanente del cliente. Es asignado por la institución financiera para representar al cliente en el sistema \\ 
\hline 
personalIdent & Cadena de caracteres (8+) & Requerido & Identificación personal presentada por el cliente \\ 
\hline 
personVerifyType & Enumeración & Requerido & El tipo de documento con el cual se verifica la identidad del cliente \\ 
\hline
dtBCustomer & Fecha & Opcional & Momento en el cual la persona se vuelve cliente de la institución \\ 
\hline
dtLLogin & Fecha & Opcional & Último momento en el cual el cliente utiliza su cuenta \\ 
\hline
group & Cadena de caracteres (32+) & Opcional & Relaciona al cliente con el objeto ``grupo'' \\ 
\hline
\end{longtable}
\end{center}

\subsubsection{Información del banco ``bankInformation''}
Agrupa la información esencial de una agencia bancaria.

\begin{center}
\begin{longtable}{|>{\centering\arraybackslash}p{0.2\textwidth}|>{\centering\arraybackslash}p{0.2\textwidth}|>{\centering\arraybackslash}p{0.2\textwidth}|>{\centering\arraybackslash}p{0.2\textwidth}|}
\hline 
\bfseries {Etiqueta} & \bfseries {Tipo} & \bfseries {Uso} & \bfseries {Descripción} \\ 
\hline 
bankId & Cadena de caracteres (4+) & Opcional & ID que identifica a la agencia bancaria \\ 
\hline 
name & Cadena de caracteres & Opcional & Nombre de la agencia \\ 
\hline 
branchId & Cadena de caracteres & Opcional & ID que identifica a la sucursal \\ 
\hline 
branchName & Cadena de caracteres & Opcional & Nombre de la sucursal \\ 
\hline 
postAddr & Cadena de caracteres & Opcional & Dirección \\ 
\hline
city & Cadena de caracteres & Opcional & Ciudad \\ 
\hline
stateProv & Cadena de caracteres & Opcional & Estado o Provincia \\ 
\hline
postalCode &  Cadena de caracteres (4+) & Opcional & Código postal \\ 
\hline
country & Cadena de caracteres & Opcional & País \\ 
\hline
\end{longtable}
\end{center}

\subsubsection{Dirección ``address''}
Representa la dirección suministrada por el cliente.

\begin{center}
\begin{longtable}{|>{\centering\arraybackslash}p{0.2\textwidth}|>{\centering\arraybackslash}p{0.2\textwidth}|>{\centering\arraybackslash}p{0.2\textwidth}|>{\centering\arraybackslash}p{0.2\textwidth}|}
\hline 
\bfseries {Etiqueta} & \bfseries {Tipo} & \bfseries {Uso} & \bfseries {Descripción} \\ 
\hline 
addressId & Cadena de caracteres & Opcional & ID que identifica a la dirección \\ 
\hline 
custPermId & Cadena de caracteres (32+) & Requerido & ID permanente del cliente. Es asignado por la institución financiera para representar al cliente en el sistema \\
\hline 
nameAddrType & Enumeración & Requerido & Define el uso de la información suministrada \\ 
\hline 
addr & Cadena de caracteres & Opcional & Dirección \\ 
\hline
city & Cadena de caracteres & Opcional & Ciudad \\ 
\hline
stateProv & Cadena de caracteres & Opcional & Estado o Provincia \\ 
\hline
postalCode &  Cadena de caracteres (4+) & Opcional & Código postal \\ 
\hline
country & Cadena de caracteres & Opcional & País \\ 
\hline
addrType & Enumeración & Opcional & Define el tipo de dirección \\ 
\hline
startDt & Hora & Opcional & Hora de inicio \\ 
\hline
endDt & Hora & Opcional & Hora de fin \\ 
\hline
\end{longtable}
\end{center}

\subsubsection{Información de contacto ``contactInfo''}
Información suministrada por el cliente para poder ser contactado en caso de necesitarlo.

\begin{center}
\begin{longtable}{|>{\centering\arraybackslash}p{0.25\textwidth}|>{\centering\arraybackslash}p{0.2\textwidth}|>{\centering\arraybackslash}p{0.15\textwidth}|>{\centering\arraybackslash}p{0.2\textwidth}|}
\hline 
\bfseries {Etiqueta} & \bfseries {Tipo} & \bfseries {Uso} & \bfseries {Descripción} \\ 
\hline 
contactInfoId & Cadena de caracteres (32+) & Opcional & ID que identifica la información de contacto del cliente \\ 
\hline 
custPermId & Cadena de caracteres (32+) & Requerido & ID permanente del cliente. Es asignado por la institución financiera para representar al cliente en el sistema \\
\hline 
custContactPref & Enumeración & Requerido & Representa la manera en la cual el cliente será contactado \\ 
\hline 
prefTimeStart & Hora & Opcional & Hora a partir de la cual puede ser contactado \\ 
\hline
prefTimeEnd & Hora de caracteres & Opcional & Hora a partir de la cual ya no puede ser contactado \\ 
\hline
dayPhone & Cadena de caracteres & Opcional (ver descripción) & Teléfono de contacto durante el día. 
\\ & & & \\
& & & Este campo es requerido si ni ``evePhone'', ``dayFax'', ``eveFax'' o ``emailAddr'' es suministrado \\ 
\hline
evePhone & Cadena de caracteres & Opcional (ver descripción) & Teléfono de contacto durante la tarde. \\ & & & \\
& & & Este campo es requerido si ni ``dayPhone'', ``dayFax'', ``eveFax'' o ``emailAddr'' es suministrado \\ 
\hline
dayFax & Cadena de caracteres & Opcional (ver descripción) & Fax de contacto durante el día. \\ & & & \\
& & & Este campo es requerido si ni ``dayPhone'', ``evePhone'', ``eveFax'' o ``emailAddr'' es suministrado \\ 
\hline
eveFax & Cadena de caracteres & Opcional (ver descripción) & Fax de contacto durante la tarde. \\ & & & \\
& & & Este campo es requerido si ni ``dayPhone'', ``evePhone'', ``dayFax'' o ``emailAddr'' es suministrado \\ 
\hline
emailAddr & Cadena de caracteres & Opcional (ver descripción) & Correo electrónico de contacto. \\ & & & \\
& & & Este campo es requerido si ni ``dayPhone'', ``evePhone'', ``dayFax'' o ``eveFax'' es suministrado \\ 
\hline
\end{longtable}
\end{center}

\subsubsection{Información personal ``personalInfo''}
Contiene la información personal de un cliente.

\begin{center}
\begin{longtable}{|>{\centering\arraybackslash}p{0.2\textwidth}|>{\centering\arraybackslash}p{0.2\textwidth}|>{\centering\arraybackslash}p{0.2\textwidth}|>{\centering\arraybackslash}p{0.2\textwidth}|}
\hline 
\bfseries {Etiqueta} & \bfseries {Tipo} & \bfseries {Uso} & \bfseries {Descripción} \\ 
\hline 
personalInfoId & Cadena de caracteres (32+) & Opcional & ID que identifica a la información personal del cliente \\ 
\hline 
custPermId & Cadena de caracteres (32+) & Requerido & ID permanente del cliente. Es asignado por la institución financiera para representar al cliente en el sistema \\
\hline 
lastName & Cadena de caracteres & Requerido & Apellido del cliente \\ 
\hline 
firstName & Cadena de caracteres & Requerido & Dirección \\ 
\hline
middleName & Cadena de caracteres & Opcional & Ciudad \\ 
\hline
tittlePrefix & Cadena de caracteres & Opcional & Titulo por el cual llamar al cliente. Por ejemplo ``Dr.'' \\ 
\hline
nameSuffix & Cadena de caracteres & Opcional & Sufijo agregado al final del nombre del cliente. Por ejemplo ``Jr.'' \\ 
\hline
\end{longtable}
\end{center}

\subsubsection{Preferencia ``preference''}
Permite al cliente definir cierto comportamiento sobre su cuenta. El cliente puede establecer un monto predeterminado para concepto de retiro sobre una de sus cuentas. También, si se le ha hecho una transferencia al cliente y no se especificó cuenta destino, el dinero será transferido a la cuenta que el cliente haya definido por defecto.

\begin{center}
\begin{longtable}{|>{\centering\arraybackslash}p{0.3\textwidth}|>{\centering\arraybackslash}p{0.15\textwidth}|>{\centering\arraybackslash}p{0.15\textwidth}|>{\centering\arraybackslash}p{0.2\textwidth}|}
\hline 
\bfseries {Etiqueta} & \bfseries {Tipo} & \bfseries {Uso} & \bfseries {Descripción} \\ 
\hline 
preferenceId & Cadena de caracteres (32+) & Opcional & ID que identifica a la información de preferencia del cliente \\ 
\hline 
custPermId & Cadena de caracteres (32+) & Requerido & ID permanente del cliente. Es asignado por la institución financiera para representar al cliente en el sistema \\
\hline 
acctId & Cadena de caracteres (32+) & Opcional & ID de la cuenta a la cual se le aplicaran los consumos por concepto de retiros predefinidos \\ 
\hline 
defaultTranfAccount & Cadena de caracteres (32+) & Opcional & ID de la cuenta a la cual se le aplicaran las transferencias sin cuenta de destino especificada \\ 
\hline
withdrawalAmt & Cadena de caracteres (32+) & Opcional (ver descripción) & Monto de retiro por defecto. 
\\ & & & \\
& & & Este campo es requerido si ``acctId'' es especificado \\
\hline
\end{longtable}
\end{center}

\subsubsection{Transferencias a terceros ``registeredRecipient''}
Contiene los datos de alguna cuenta o tarjeta de otro banco junto con la identificación de sus acreedores.

\begin{center}
\begin{longtable}{|>{\centering\arraybackslash}p{0.25\textwidth}|>{\centering\arraybackslash}p{0.2\textwidth}|>{\centering\arraybackslash}p{0.15\textwidth}|>{\centering\arraybackslash}p{0.2\textwidth}|}
\hline 
\bfseries {Etiqueta} & \bfseries {Tipo} & \bfseries {Uso} & \bfseries {Descripción} \\ 
\hline 
recipientId & Cadena de caracteres (32+) & Opcional & ID que identifica la información acerca de un cliente en otra institución financiera \\ 
\hline 
custPermId & Cadena de caracteres (32+) & Requerido & ID permanente del cliente. Es asignado por la institución financiera para representar al cliente en el sistema \\
\hline 
personId & Cadena de caracteres (32+) & Opcional & ID que identifica al objeto ``person'' \\ 
\hline 
acctNum & Cadena de caracteres & Opcional (ver descripción) & representa el número de cuenta en alguna otra institución financiera. 
\\ & & & \\
& & & Este campo es requerido si ``cardSeqNum'' no es especificado \\ 
\hline
cardSeqNum & Cadena de caracteres & Opcional (ver descripción) & representa el número de tarjeta de alguna otra institución financiera. 
\\ & & & \\
& & & Este campo es requerido si ``acctNum'' no es especificado \\ 
\hline
name & Cadena de caracteres & Requerido & Nombre del beneficiario \\ 
\hline
desc & Cadena de caracteres & Requerido & Descripción \\ 
\hline
maxAmtLimit & Cadena de caracteres & Opcional & Máximo monto permitido para realizar la transferencia \\ 
\hline
personalIdent & Cadena de caracteres (8+) & Requerido & Identificación personal presentada por el cliente \\ 
\hline
personVerifyType & Enumeración & Requerido & El tipo de documento con el cual se verifica la identidad del cliente \\ 
\hline
\end{longtable}
\end{center}

\subsubsection{Persona ``person''}
Contiene los datos que identifican a los clientes como personas. También reúne los
datos de los clientes que tienen cuentas en otros bancos, estos datos provienen de
``registeredRecipient''.

\begin{center}
\begin{longtable}{|>{\centering\arraybackslash}p{0.3\textwidth}|>{\centering\arraybackslash}p{0.15\textwidth}|>{\centering\arraybackslash}p{0.15\textwidth}|>{\centering\arraybackslash}p{0.2\textwidth}|}
\hline 
\bfseries {Etiqueta} & \bfseries {Tipo} & \bfseries {Uso} & \bfseries {Descripción} \\ 
\hline 
personId & Cadena de caracteres (32+) & Opcional & ID que identifica a la información de una persona \\ 
\hline 
name & Cadena de caracteres & Requerido & Nombre \\
\hline 
\end{longtable}
\end{center}

\subsubsection{Conocido ``known''}
Contiene la información de las personas conocidas. De esta forma se puede pueden
realizar transferencias a personas en lugar de a cuentas.

\begin{center}
\begin{longtable}{|>{\centering\arraybackslash}p{0.3\textwidth}|>{\centering\arraybackslash}p{0.15\textwidth}|>{\centering\arraybackslash}p{0.15\textwidth}|>{\centering\arraybackslash}p{0.2\textwidth}|}
\hline 
\bfseries {Etiqueta} & \bfseries {Tipo} & \bfseries {Uso} & \bfseries {Descripción} \\ 
\hline 
knownId & Cadena de caracteres (32+) & Opcional & ID que identifica a la información acerca de un conocido	 \\ 
\hline 
personId & Cadena de caracteres (32+) & Requerido & ID que identifica la información acerca de una persona.
\\ & & & \\
& & & En este caso representa a un conocido \\
\hline 
custPermId & Cadena de caracteres (32) & Requerido & ID permanente del cliente. Es asignado por la institución financiera para representar al cliente en el sistema.
\\ & & & \\
& & & En este caso representa al conocedor \\
\hline 
relationship & Cadena de caracteres & Requerido & Describe el tipo de relación entre el conocedor y el conocido. \\ 
\hline 
status & Booleano & Opcional & Representa si se ha validado que estas dos personas se conocen \\ 
\hline 
\end{longtable}
\end{center}

\subsubsection{Miembro de un grupo ``groupMember''}
Un cliente tiene la capacidad de crear grupo de personas conocidas. De esta forma puede establecer en que cuenta serán ubicados los fondos recibidos por parte de algún miembro del grupo. Un miembro del grupo es aquel cliente que pertenezca a un grupo.

\begin{center}
\begin{longtable}{|>{\centering\arraybackslash}p{0.25\textwidth}|>{\centering\arraybackslash}p{0.2\textwidth}|>{\centering\arraybackslash}p{0.15\textwidth}|>{\centering\arraybackslash}p{0.2\textwidth}|}
\hline 
\bfseries {Etiqueta} & \bfseries {Tipo} & \bfseries {Uso} & \bfseries {Descripción} \\ 
\hline 
groupMemberId & Cadena de caracteres (32+) & Opcional & ID que identifica al objeto ``groupMember'' \\ 
\hline 
custPermId & Cadena de caracteres (32+) & Requerido & ID permanente del cliente. Es asignado por la institución financiera para representar al cliente en el sistema \\
\hline 
member & Cadena de caracteres (32+) & Requerido & Representa al cliente miembro del grupo \\
\hline 
groupId & Cadena de caracteres (32+) & Requerido & Identifica al grupo al cual pertenece un miembro de grupo\\
\hline 
\end{longtable}
\end{center}

\subsubsection{Grupo ``group''}
Es una unidad en la cual un cliente puede agrupar a otros clientes del banco y
predefinir una cuenta en la cual los miembros al grupo transferirán.

\begin{center}
\begin{longtable}{|>{\centering\arraybackslash}p{0.2\textwidth}|>{\centering\arraybackslash}p{0.2\textwidth}|>{\centering\arraybackslash}p{0.2\textwidth}|>{\centering\arraybackslash}p{0.2\textwidth}|}
\hline 
\bfseries {Etiqueta} & \bfseries {Tipo} & \bfseries {Uso} & \bfseries {Descripción} \\ 
\hline 
groupId & Cadena de caracteres (32+) & Opcional & ID que identifica al objeto ``group'' \\ 
\hline
acctId & Cadena de caracteres (32+) & Opcional & ID de la cuenta a la cual se le aplicaran los consumos por concepto de retiros predefinidos \\ 
\hline 
name & Cadena de caracteres & Requerido & Nombre del grupo \\
\hline 
descripción & Cadena de caracteres & Opcional & Descripción del grupo \\
\hline 
\end{longtable}
\end{center}

\subsubsection{Estado de la cuenta ``accountStatus''}
Tiene la información del estado en el cual se encuentra la cuenta.

\begin{center}
\begin{longtable}{|>{\centering\arraybackslash}p{0.25\textwidth}|>{\centering\arraybackslash}p{0.2\textwidth}|>{\centering\arraybackslash}p{0.15\textwidth}|>{\centering\arraybackslash}p{0.2\textwidth}|}
\hline 
\bfseries {Etiqueta} & \bfseries {Tipo} & \bfseries {Uso} & \bfseries {Descripción} \\ 
\hline 
accountStatusId & Cadena de caracteres (32+) & Opcional & ID que identifica al objeto ``accountStatus'' \\ 
\hline
acctId & Cadena de caracteres (32+) & Opcional & ID de la cuenta a la cual se le aplicaran los consumos por concepto de retiros predefinidos \\ 
\hline 
accountStatusCode & Enumeración & Requerido & Representa el estado de la cuenta \\
\hline 
effDt & Fecha & Opcional & Fecha en la cual se hizo efectivo dicho estado \\
\hline 
statusModBy & Cadena de caracteres & Opcional & Tiene la información acerca de quien modificó el estado \\
\hline 
statusDesc & Cadena de caracteres & Opcional & Descripción sobre el estado \\
\hline 
\end{longtable}
\end{center}

\subsubsection{Estado de la tarjeta ``cardStatus''}
Contiene la información del estado en el cual se encuentra la tarjeta.

\begin{center}
\begin{longtable}{|>{\centering\arraybackslash}p{0.25\textwidth}|>{\centering\arraybackslash}p{0.2\textwidth}|>{\centering\arraybackslash}p{0.15\textwidth}|>{\centering\arraybackslash}p{0.2\textwidth}|}
\hline 
\bfseries {Etiqueta} & \bfseries {Tipo} & \bfseries {Uso} & \bfseries {Descripción} \\ 
\hline 
card	StatusId & Cadena de caracteres (32+) & Opcional & ID que identifica al objeto ``cardStatus'' \\ 
\hline
cardEmBossNum & Cadena de caracteres (32+) & Requerido & Número de la tarjeta a la cual pertenece el estado \\ 
\hline 
cardStatusCode & Enumeración & Requerido & Representa el estado de la tarjeta \\
\hline 
effDt & Fecha & Opcional & Fecha en la cual se hizo efectivo dicho estado \\
\hline 
statusModBy & Cadena de caracteres & Opcional & Tiene la información acerca de quien modificó el estado \\
\hline 
statusDesc & Cadena de caracteres & Opcional & Descripción sobre el estado \\
\hline 
\end{longtable}
\end{center}

\subsubsection{Tarjeta ``card''}
Contiene la información de la tarjeta.

\begin{center}
\begin{longtable}{|>{\centering\arraybackslash}p{0.25\textwidth}|>{\centering\arraybackslash}p{0.2\textwidth}|>{\centering\arraybackslash}p{0.15\textwidth}|>{\centering\arraybackslash}p{0.2\textwidth}|}
\hline 
\bfseries {Etiqueta} & \bfseries {Tipo} & \bfseries {Uso} & \bfseries {Descripción} \\ 
\hline 
cardEmBossNum & Cadena de caracteres (32+) & Requerido & Número de la tarjeta \\ 
\hline
acctId & Cadena de caracteres (32+) & Requerido & ID de la cuenta a la cual se le aplicaran los consumos por concepto de retiros predefinidos \\ 
\hline
cardType & Enumeración & Opcional & Tipo de tarjeta \\
\hline 
brand & Enumeración & Opcional & Consorcio al que pertenece la tarjeta \\
\hline 
issuerName & Cadena de caracteres & Opcional & Nombre del tarjetahabiente \\
\hline 
issDt & Fecha & Opcional & Fecha en la cual se emite la tarjeta \\
\hline 
expDt & Fecha & Opcional & Fecha en la cual expira la tarjeta \\
\hline 
\end{longtable}
\end{center}

\subsubsection{Balance ``balance''}
Contiene la información del dinero existente en una cuenta y las transacciones que
la han afectado.

\begin{center}
\begin{longtable}{|>{\centering\arraybackslash}p{0.2\textwidth}|>{\centering\arraybackslash}p{0.2\textwidth}|>{\centering\arraybackslash}p{0.2\textwidth}|>{\centering\arraybackslash}p{0.2\textwidth}|}
\hline 
\bfseries {Etiqueta} & \bfseries {Tipo} & \bfseries {Uso} & \bfseries {Descripción} \\ 
\hline 
acctId & Cadena de caracteres (32+) & Requerido & ID de la cuenta a la cual pertenece el balance \\ 
\hline
transId & Entero (8) & Opcional & ID de la transacción que afectó el balance \\
\hline 
curAmt & Decimal & Requerido & Cantidad de dinero en la cuenta para la fecha \\
\hline 
effDt & Fecha & requerido & Fecha en la cual se afectó el balance \\
\hline 
descr & Cadena de caracteres & Opcional & Descripción \\
\hline 
\end{longtable}
\end{center}

\subsubsection{Transacción ``transaction''}
Cualquier movimiento que afecte algún balance.

\begin{center}
\begin{longtable}{|>{\centering\arraybackslash}p{0.2\textwidth}|>{\centering\arraybackslash}p{0.2\textwidth}|>{\centering\arraybackslash}p{0.2\textwidth}|>{\centering\arraybackslash}p{0.2\textwidth}|}
\hline 
\bfseries {Etiqueta} & \bfseries {Tipo} & \bfseries {Uso} & \bfseries {Descripción} \\ 
\hline
transId & Entero (8) & Opcional & ID de la transacción \\
\hline 
acctId & Cadena de caracteres (32+) & Requerido & ID de la cuenta que realizó la transacción \\ 
\hline 
acctOutFlow & Cadena de caracteres (32+) & Opcional (ver descripción) & ID de la cuenta a la cual se le debitará el dinero.
\\ & & & \\
& & & Este campo es requerido en caso de que la transacción debite de alguna forma dinero de la cuenta \\
\hline 
acctInFlow & Cadena de caracteres (32+) & Opcional (ver descripción) & ID de la cuenta que recibirá el dinero.
\\ & & & \\
& & & Este campo es requerido en caso de que la transacción abone de alguna forma dinero a la cuenta \\
\hline
thirdParty & Cadena de caracteres & Opcional (ver descripción) & ID de la cuenta o tarjeta de un beneficiario en otro banco.
\\ & & & \\
& & & Este campo es requerido en caso de existir un tercero involucrado \\
\hline
amt & Decimal & Requerido & Cantidad de dinero que moverá la transacción \\
\hline 
dueDt & Fecha & Requerido & Fecha en la cual se desea la transacción sea efectuada \\
\hline 
curDt & Fecha & Requerido & Fecha en la cual se realizó la transacción \\
\hline 
transType & Enumeración & Opcional & Tipo de transacción realizada \\
\hline 
\end{longtable}
\end{center}

\subsubsection{Transacción ``transaction''}
Cualquier movimiento que afecte algún balance.

\begin{center}
\begin{longtable}{|>{\centering\arraybackslash}p{0.2\textwidth}|>{\centering\arraybackslash}p{0.2\textwidth}|>{\centering\arraybackslash}p{0.2\textwidth}|>{\centering\arraybackslash}p{0.2\textwidth}|}
\hline 
\bfseries {Etiqueta} & \bfseries {Tipo} & \bfseries {Uso} & \bfseries {Descripción} \\ 
\hline
acctId & Cadena de caracteres (32+) & Requerido & ID de la cuenta que realizó la transacción \\ 
\hline 
bankId & Cadena de caracteres & Requerido & ID que identifica a la agencia bancaria \\
\hline 
custPermId & Cadena de caracteres (32+) & Requerido & ID permanente del cliente. Es asignado por la institución financiera para representar al cliente en el sistema 
\\ & & & \\
& & & Este campo es requerido si la cuenta no es compartida \\
\hline
acctType & Enumeración & Requerido & Define de qué tipo es la cuenta \\
\hline
freeForAll & Booleano & Opcional (ver descripción) & En caso de que la cuenta sea compartida, define si cualquiera puede efectuar operaciones o se necesita la aprobación de todos los participantes.
\\ & & & \\
& & & Es requerido si el campo ``custPermId'' no está definido \\
\hline 
members & Entero & Opcional (ver descripción) & Define cuantos clientes comparten la cuenta
\\ & & & \\
& & & El campo es requerido si ``freeForAll'' está definido \\
\hline 
\end{longtable}
\end{center}

\subsection{Recursos abstractos}
A diferencia de los recursos concretos, los recursos abstractos no representan objetos en la base de datos. Estos recursos ofrecen una información más general y son producto de una recopilación de información sobre varios objetos en la base de datos. También se pueden utilizar estos recursos para ofrecer una forma más sencilla de actualizar alguna información en el sistema. Por ejemplo, si se desea registrar un nuevo cliente, se tendrían que mandar varias peticiones para crear los recursos necesarios para que el cliente sea registrado. Para evitar mandar varias peticiones se podría ofrecer un recurso que recopile toda la información necesaria y este se encargue de crear los recursos concretos necesarios para que el nuevo cliente pueda ser registrado.
%\end{document}
%\input{apendices/Ejemplos_del_lenguaje.tex}
%\input{apendices/Gramaticas.tex}

%\printglossary

\end{document}
