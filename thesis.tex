% file thesis.tex
% Archivo thesis.tex
% Documento maestro que incluye todos los paquetes necesarios para el documento
% principal.

% Documento obtenido por un sinfin de iteraciones de administradores del LDC
% Estructura actual hecha por:
% Jairo Lopez <jairo@ldc.usb.ve>
% Actualizado ligeramente por:
% Alexander Tough 

\documentclass[oneside,12pt,letterpaper]{report}
\tolerance=1000  
\hbadness=10000  
\raggedbottom

% Macros personalizados
\newcommand{\entry}{\textit{entry}}
\newcommand{\entries}{\textit{entries}}
\newcommand{\activity}{\textit{activity}}

% Para escribir algoritmos
\usepackage{listings}
\usepackage{algpseudocode}
\usepackage{algorithmicx}
\usepackage{algorithm}

\usepackage{pdflscape}

% Paquetes para manejar graficos
\usepackage{epsf}
\usepackage[pdftex]{graphicx}
\usepackage{epsfig}
% Simbolos matematicos
\usepackage{latexsym,amssymb}
% Paquetes para presentar una tesis decente.
\usepackage{setspace,cite} % Doble espacio para texto, espacio singular para
                           % los caption y pie de pagina

\usepackage[table]{xcolor}
\usepackage{tikz}
\usetikzlibrary{shapes.geometric,arrows}

\usetikzlibrary{arrows,shapes}
\usepackage{verbatim}

\usepackage{comment}

% Paquetes no utilizados para citas
%\usepackage{mcite} 
%\usepackage{draft} 

\usepackage{wrapfig}
\usepackage{alltt}

% Acentos 
\usepackage[spanish,activeacute,es-noquoting]{babel}

\usepackage[spanish]{translator}
\usepackage[utf8]{inputenc}
\usepackage{color, xcolor, colortbl}
\usepackage{multirow}
\usepackage{subfig}
\usepackage[OT1]{fontenc}
\usepackage{tocbibind}
\usepackage{anysize}
\usepackage{listings} 

% Para poder tener texto asiatico
%\usepackage{CJK}

\usepackage{pdfpages}

% Opciones para los glosarios
\usepackage[style=altlist,toc,numberline,acronym]{glossaries}
\usepackage{url}
\usepackage{amsthm}
\usepackage{amsmath}
\usepackage{fancyhdr} % Necesario para los encabezados
\usepackage{fancyvrb}
\usepackage{makeidx} % En caso de necesitar indices.
\makeindex  % Necesitado para los indices

% Definiciones para definicions, teoremas y lemas
\theoremstyle{definition} \newtheorem{definicion}{Definici\'{o}n}
\theoremstyle{plain} \newtheorem{teorema}{Teorema}
\theoremstyle{plain} \newtheorem{lema}{Lema}

% Para la creacion de los pdfs
\usepackage{hyperref}

% Para resolver el lio del Unicode para la informacion de los PDFs
% En pdftitle coloca el nombre de su proyecto de grado/pasantia.
% En pdfauthor coloca su nombre.
\hypersetup{
    pdftitle = {Desarrollo de una aplicación móvil de control y reportes de gastos},
    pdfauthor={Susana Charara Charara},
    colorlinks,
    citecolor=black,
    filecolor=black,
    linkcolor=black,
    urlcolor=black,
    backref,
    pdftex
}

\definecolor{brown}{rgb}{0.7,0.2,0}
\definecolor{darkgreen}{rgb}{0,0.6,0.1}
\definecolor{darkgrey}{rgb}{0.4,0.4,0.4}
\definecolor{lightgrey}{rgb}{0.95,0.95,0.95}

\usepackage{listings}
\lstnewenvironment{code}{\lstset{basicstyle=\small}}{}

\lstset{escapeinside=~~}
\lstset{
   frame=single,
   framerule=1pt,
   showstringspaces=false,
   basicstyle=\footnotesize\ttfamily,
   keywordstyle=\textbf,
   backgroundcolor=\color{lightgrey}
}

% Crea el glosario
%\makeglossaries

% Incluye el glosario
%\chapter*{Glosario}
\noindent
\textbf{Android:} Sistema operativo de código abierto basado en el núcleo de Linux, utilizado pricipalmente en dispositivos móviles.\\ 	\\
\textbf{\textit{Application Programming Interface} (API):} Conjunto de funciones y protocolos que ofrece una librería como capa de abstracción, para ser utilizados por otro \textit{software}.\\ \\
\textbf{Aplicación móvil:} \textit{Software} diseñado para ser ejecutado en dispositivos móviles, como teléfonos inteligentes y tabletas.\\ \\
\textbf{Aplicación web:} Herramientas que los usuarios pueden utilizar para acceder a un servidor web a través de Internet.\\ \\
\textbf{Cliente:} \textit{Software} o usuario que realiza peticiones de tareas a otros ordenadores que actúan como servidores.\\ \\
\textbf{\textit{Framework}:} Estructura conceptual y tecnológica que sirve de base para la organización e implementación de \textit{software}.\\ \\
\textbf{\textit{Hypertext Markup Language} (HTML):} Lenguaje utilizado para la elaboración de páginas web.\\ \\
\textbf{\textit{Integrated Development Environment} (IDE):} Aplicación informática que provee servicios que facilitan el desarrollo de \textit{software} al usuario.\\ \\
\textbf{Librería:} Conjunto de funciones, desarrolladas en un lenguaje de programación, que ofrecen una interfaz definida para la funcionalidad que se invoca.\\ \\
\textbf{Manejador de base de datos:} Colección de \textit{software} que sirve de interfaz entre la base de datos, el usuario y las aplicaciones utilizadas.\\ \\
\textbf{\textit{Object-Relational Maping} (ORM):} Técnica de programación utilizada para convertir datos entre sistemas de tipos incompatibles en lenguajes orientados a objetos.\\ \\
\textbf{\textit{Portable Dcument Format} (PDF):} Formato de almacenamiento de documentos digitales intependiente de plataformas de \textit{software} y \textit{hardware}.\\ \\
\textbf{Scrum:} Marco de desarrollo ágil que se caracteriza por adoptar una estrategia de desarrollo incremental e iterativa.\\ \\
\textbf{\textit{Software Development Kit} (SDK):} Conjunto de herramientas de desarrollo de \textit{software} que le permiten al programador crear aplicaciones para un sistema en concreto.\\ \\
\textbf{Servidor:} Computador en el que se ejecuta continuamente un \textit{Software} que realiza tareas para atender peticiones de un cliente.\\ \\
\textbf{Servidor web:} Computador en el que se ejecuta continuamente un \textit{Software}, al cual se hacen peticiones a través de Internet.\\ \\
\textbf{\textit{Simple Mail Transfer Protocol} (SMTP):} Protocolo de red que se utiliza para el intercambio de mensajes de correo electrónico entre dispositivos.\\ \\
\textbf{\textit{Structured Query language} (SQL):} Lenguaje declarativo que permite realizar operaciones sobre bases de datos relacionales.\\ \\
\textbf{\textit{Uniform Resource Locator} (URL):} Secuencia de caracteres que sigue un estándar y permite denominar recursos dentro del entorno de Internet para que pueden ser localizados.\\

% Para crear la hoja escaneada de las firmas
\usepackage[absolute]{textpos}

% Pone los nombres y las opciones para mostrar los codigos fuentes
\lstset{language=C, breaklines=true, frame=single, showstringspaces=false,
        showtabs=false, numbers=left, keywordstyle=\color{black},
        basicstyle=\footnotesize, captionpos=b }
\renewcommand{\lstlistingname}{C\'{o}digo fuente}
\renewcommand{\lstlistlistingname}{\'{I}ndice de c\'{o}digos fuentes}

\newcommand{\todo}{ TODO: }

% Dimensiones de la pagina
\setlength{\headheight}{15pt}
\marginsize{3cm}{2cm}{2cm}{2cm}

%%%%%%%%%%%%%%%%%%%%%%%%%%%%%%%%%%%%%%%%%%%%%%%%%%%%%%%%%%%%%%%%%%%%%%%%%%%
%%%%%%%%%%%%%%%%      end of preamble and start of document     %%%%%%%%%%%
%%%%%%%%%%%%%%%%%%%%%%%%%%%%%%%%%%%%%%%%%%%%%%%%%%%%%%%%%%%%%%%%%%%%%%%%%%%
\begin{document}

% Pagina de titulo
% Pagina de titulo
\begin{titlepage}
\begin{center}

% Upper part (aqui ya esta incluido el logo de la USB).
\includegraphics[scale=0.5,type=png,ext=.png,read=.png]{imagenes/cebolla} \\

% Encabezado
\textsc {\large UNIVERSIDAD SIMÓN BOLÍVAR} \\
\textsc{\bfseries DECANATO DE ESTUDIOS PROFESIONALES\\
COORDINACI'ON DE INGENIER'IA DE LA COMPUTACI'ON}

\bigskip
\bigskip
\bigskip
\bigskip
\bigskip
\bigskip
\bigskip
\bigskip
\bigskip

% Title/Titulo
% Aqui ponga el nombre de su proyecto de grado/pasantia larga
\textsc{\bfseries DESARROLLO DE UNA APLICACIÓN MÓVIL DE CONTROL Y REPORTES DE GASTOS}

\bigskip
\bigskip
\bigskip
\bigskip
\bigskip

% Author and supervisor/Autor y tutor
\begin{minipage}{\textwidth}
\centering
Por: \\ SUSANA CHARARA CHARARA \\

\bigskip
\bigskip
\bigskip

Realizado con la asesoría de: \\
Tutor Académico: PROF. XIOMARA CONTRERAS \\
Tutor Industrial: LIC. LUIS AUGUSTO PEÑA PEREIRA
\end{minipage}

\bigskip
\bigskip
\bigskip
\bigskip
\bigskip
\bigskip
\bigskip
\bigskip
\bigskip

% Bottom half
{INFORME DE PASANTÍA LARGA \\ Presentado ante la Ilustre Universidad Simón Bolívar \\
como requisito parcial para optar al título de \\ Ingeniero en Computación} \\

\bigskip
\bigskip
\vfill

% Date/Fecha 
{\large \bfseries Sartenejas, 
%FECHA
OCTUBRE de 2016}

\end{center}
\end{titlepage}

% Pagina de titulo
\begin{titlepage}
\begin{center}

% Upper part (aqui ya esta incluido el logo de la USB).
\includegraphics[scale=0.5,type=png,ext=.png,read=.png]{imagenes/cebolla} \\

% Encabezado
\textsc {\large UNIVERSIDAD SIMÓN BOLÍVAR} \\
\textsc{\bfseries DECANATO DE ESTUDIOS PROFESIONALES\\
COORDINACI'ON DE INGENIER'IA DE LA COMPUTACI'ON}

\bigskip
\bigskip
\bigskip
\bigskip
\bigskip
\bigskip
\bigskip
\bigskip
\bigskip

% Title/Titulo
% Aqui ponga el nombre de su proyecto de grado/pasantia larga
\textsc{\bfseries DESARROLLO DE UNA APLICACIÓN MÓVIL DE CONTROL Y REPORTES DE GASTOS}

\bigskip
\bigskip
\bigskip
\bigskip
\bigskip

% Author and supervisor/Autor y tutor
\begin{minipage}{\textwidth}
\centering
Por: \\ SUSANA CHARARA CHARARA \\

\bigskip
\bigskip
\bigskip

Realizado con la asesoría de: \\
Tutor Académico: PROF. XIOMARA CONTRERAS \\
Tutor Industrial: LIC. LUIS AUGUSTO PEÑA PEREIRA
\end{minipage}

\bigskip
\bigskip
\bigskip
\bigskip
\bigskip
\bigskip
\bigskip
\bigskip
\bigskip

% Bottom half
{INFORME DE PASANTÍA LARGA \\ Presentado ante la Ilustre Universidad Simón Bolívar \\
como requisito parcial para optar al título de \\ Ingeniero en Computación} \\

\bigskip
\bigskip
\vfill

% Date/Fecha 
{\large \bfseries Sartenejas, 
%FECHA
OCTUBRE de 2016}

\end{center}
\end{titlepage}

% Pagina de acta final (vacio)
%\includepdf[pages={1}]{intro/firmas.pdf}

\setcounter{secnumdepth}{4}
\setcounter{tocdepth}{5}

% Define encabezado numeros romanos y como se separan los captiulos y las
% secciones
\addtolength{\headheight}{3pt}
\pagenumbering{roman}
\pagestyle{fancyplain}

\renewcommand{\chaptermark}[1]{\markboth{\chaptername\ \thechapter:\,\ #1}{}}
\renewcommand{\sectionmark}[1]{\markright{\thesection\,\ #1}}

\onehalfspacing

\lhead{}
\chead{}
\rhead{}
\renewcommand{\headrulewidth}{0.0pt}
\lfoot{}
\cfoot{\fancyplain{}{\thepage}}
\rfoot{}


% Pagina de resumen
\setcounter{page}{3}
\begin{center}
	{\bf Resumen} \pdfbookmark[0]{Resumen}{resumen} % Sets a PDF bookmark for the dedication
\end{center}	

En esta pasantía se desarrolló, para la empresa Digitalica Group, C.A., una aplicación para dispositivos móviles con la cual se pueden realizar reportes de gastos. Actualmente, la empresa Digitalica Group, C.A. ofrece a sus trabajadores el beneficio de realizar reembolsos de gastos en ciertos rubros; el proyecto nació como solución al problema de agilizar el proceso en que los trabajadores hacen llegar a los supervisores la información de estos gastos. Con la aplicación desarrollada, se pueden registrar gastos e ingresos con un monto, fecha y descripción. Se permite también capturar y guardar fotos tanto de los ingresos como de los gastos, de manera que el usuario puede tomar fotos de los recibos de sus gastos. Además, estos gastos/ingresos pueden estar asociados a categorías, que indican el rubro al que pertenecen. La aplicación permite al usuario generar archivos con formato PDF que contienen un reporte con una lista de gastos, en un rango de fecha determinado; estos reportes se pueden enviar a un servidor web. Este servidor también fue desarrollado durante la pasantía. Se creó un servicio web que recibe credenciales de un usuario y las valida; éste es utilizado por la aplicación móvil para autenticar usuarios. Asimismo, se desarrolló un servicio web que recibe archivos PDF con reportes de gastos, el cual es utilizado por la aplicación para enviar reportes. Igualmente, se desarrollaron servicios y una aplicación web, con la cual los supervisores de la empresa pueden revisar y descargar los archivos de los reportes recibidos por el servidor, aprobarlos y rechazarlos.

En este informe se describen los conceptos teóricos que ayudaron al diseño y desarrollo de la solución. Igualmente, se describe el proceso de desarrollo de los tres componentes principales desarrollados durante la pasantía: aplicación nativa para Android, servidor web y aplicación web. El desarrollo de estos involucró el diseño e implementación de los modelos de datos de la aplicación y el servidor, así como múltiples interfaces de usuario que permiten la interacción con los modelos. Para cada componente, se describen el entorno de trabajo y las herramientas utilizadas. Se utilizó Java como lenguaje de programación tanto de la aplicación móvil como del servidor; entre las herramientas que se emplearon en el desarrollo del servidor se pueden mencionar AppFuse, Spring, Hibernate y Tapestry, entre otros.

Por otra parte, se utilizó Scrum como marco de trabajo, dividiendo el desarrollo del proyecto en ocho iteraciones, en las que se implementaron los componentes y funcionalidades necesarios para el cumplimiento de los objetivos de la pasantía.

El objetivo del desarrollo de la pasantía fue ofrecer una solución al problema que actualmente se presenta en la empresa de agilizar el proceso de reembolso sobre ciertos gastos.




% Pagina de dedicatoria (opcional)
%\pagebreak

%\setcounter{page}{5}

\vspace*{8cm} 
\pdfbookmark[0]{Dedicatoria}{dedicatoria} % Sets a PDF bookmark for the dedication
\begin{center} 
\large A nuestros padres.\\ Porque nos dieron la vida y nos han guiado 
a ser quienes somos hoy.
\end{center}
\newpage


% Pagina de agradecimientos (opcional)
%\input{intro/agradecimientos.tex}

% Crea la tabla de contenidos
\tableofcontents

% Crea la lista de cuadros
%\listoftables

% Crea la lista de figuras
\listoffigures
%\newpage
\phantomsection
%\setcounter{page}{4}
\chapter*{Lista de Símbolos y Abreviaturas}% Sets a PDF bookmark for the dedication
\textbf{CEO:} Chief Executive Officer (en español Director Ejecutivo)\\ \\
\textbf{COO:} Chief Operations Officer (en español Director de Operaciones)\\ \\
\textbf{CTO:} Chief Technology Officer (en español Director de Tecnología)\\ \\
\textbf{DAO:} Data Access Object (en español Objeto de Acceso a Datos)\\ \\
\textbf{DVCS:} Distributed Version Control System (en español Sistema de Control de Versiones Distribuido)\\ \\
\textbf{DVM:} Dalvik Virtual Machine (en español OMáquina Virtual Dalvik)\\ \\
\textbf{HTTP:} HyperText Transfer Protocol (en español Protocolo de Transferencia de Hipertexto)\\ \\
\textbf{IDE:} Integrated Development Environment (en español Entorno de Desarrollo Integrado)\\ \\
\textbf{JSON:} JavaScript Object Notation (en español Notación de Objetos de JavaScript)\\ \\
\textbf{JVM:} Java Virtual Machine(en español OMáquina Virtual de Java)\\ \\
\textbf{MVC:} Model View Controller (en español Modelo Vista Controlador)\\ \\
\textbf{MVP:} Model View Presenter (en español Modelo Vista Presentador)\\ \\
\textbf{ORM:} Object-Relational Mapping (en español Mapeo Objeto-Relacional)\\ \\
\textbf{PDF:} Portable Document Format (en español Formato de Documento Portátil)\\ \\
\textbf{POJO:} Plain Old Java Object (en español Objeto de Java Plano Antiguo)\\ \\
\textbf{SQL:} Structured Query Language (en español Lenguaje de Consulta Estructurada)\\ \\

\addcontentsline{toc}{chapter}{Lista de Símbolos y Abreviaturas}

% Crea la lista de codigos fuentes
%\lstlistoflistings

\clearpage

% Define encabezado en numeros arabicos  
\pagenumbering{arabic}

\fancyhf{} % Redefine el encabezado 
\lhead{}
\chead{}
\rhead{\fancyplain{}{\thepage}}
\renewcommand{\headrulewidth}{0.0pt}
\lfoot{}
\cfoot{}
\rfoot{}

\doublespacing

% Incluye los archivos deseados - El contenido de su proyecto de grado/pasantia larga.
\phantomsection
\addcontentsline{toc}{chapter}{Introducción}
\chapter*{Introducción} \label{sec:Introduccion}
%\pdfbookmark[0]{Introducción}{introduccion} % Sets a PDF bookmark for the dedication

\vspace{5 mm}

En la actualidad, muchas empresas le ofrecen a sus empleados el beneficio de realizar gastos personales y posteriormente iniciar un proceso de reembolso. Este proceso puede tomar mucho tiempo y puede resultar difícil si se hace manualmente, pues implica que el trabajador deba guardar todas las facturas o recibos de los gastos implicados. También dificulta la organización de estos registros por parte de la empresa, porque al mantener todos los recibos en físico, estos se pueden extraviar.

Este problema llevó a la conclusión de que es necesario agilizar todo el proceso, y dada la facilidad de acceso que se tiene hoy en día a un dispositivo móvil inteligente, se decidió crear una aplicación que permita mantener el registro de gastos. 

Actualmente existen aplicaciones que sirven para llevar un registro de gastos. Sin embargo, estas aplicaciones no satisfacen las necesidades de la empresa por diversas razones:

\begin{itemize}
\item En primer lugar, la empresa realiza el reembolso de gastos en ciertos rubros y para esto se debe especificar a qué categoría pertenece cada gasto. Las aplicaciones ya existentes pueden limitar el proceso al no contar con la posibilidad de poder asociar un gasto a un rubro cubierto por la empresa.
\item Se desea crear archivos de reportes de los gastos registrados por los trabajadores. Se desea que estos reportes puedan ser personalizados y presenten una estructura particular.
\item Se desea mantener toda la información referente a estos reportes de manera centralizada de manera que se pueda acceder a la misma de una manera más fácil.
\item Si el proceso de reembolso sufre alguna modificación, se desea contar con una aplicación que tome en cuenta los nuevos cambios.
\end{itemize}

Por las razones expuestas anteriormente, se tomó la decisión de crear una aplicación móvil que permita llevar un registro de gastos. Esta aplicación debe contar con las siguientes funcionalidades:

\begin{itemize}
\item Registrar gastos con su fecha, monto, una breve descripción, fotos y categoría a la que pertenecen
\item Organizar y consultar gastos con criterios de búsqueda pre-establecidos
\item Crear reportes personalizados con los gastos y enviar dichos reportes a un servidor
\item Enviar reportes mediante archivos con formato PDF a los supervisores
\item Controlar el estado de aprobación de los reportes
\end{itemize}

En este informe se describe el proceso de desarrollo de una aplicación móvil para el registro de gastos.

El informe se estructura de la siguiente manera: En el Capítulo 1 se describe de una forma general la empresa; en el Capítulo 2 se definen los conceptos teóricos estudiados para el desarrollo del proyecto; en el Capítulo 3 se mencionan las herramientas y tecnologías que facilitaron el desarrollo; en el Capítulo 4 se describe brevemente el marco de trabajo utilizado, Scrum; en el Capítulo 5 se describen todas las fases involucradas tanto en el diseño como el desarrollo de la solución; en el capítulo 6 se exponen las conclusiones y recomendaciones que surgieron luego de la investigación y desarrollo del proyecto; finalmente, se muestran las referencias bibliográficas consultadas.
%
\input{empresa/0_empresa.tex}
\section{Descripción de la empresa} \label{Descripcion de la empresa}

Digitalica Group, C.A. es una empresa que se dedica al desarrollo de soluciones móviles, aplicaciones empresariales y \textit{middleware}. Se caracteriza por estar a la vanguardia, utilizando tecnologías innovadoras\cite{DIG1}. Sus principales servicios son:

\begin{itemize}
\item Consultoría en el área de desarrollo de aplicaciones móviles: Durante los últimos años se ha brindado apoyo en el área de soluciones móviles a empresas del sector educativo.
\item Desarrollo de \textit{software} y aplicaciones móviles para clientes. Se han desarrollado aplicaciones para medios de comunicación nacionales e internacionales, así como para empresas del sector educativo.
\item Ofrecer \textit{software} como un servicio: Actualmente, los productos que ofrece la empresa están enfocados en el área de noticias, medios de comunicación y registro de gastos e ingresos. 
\end{itemize}

\section{Misión} \label{Mision}

"Proveer soluciones y servicios de consultoría basados en tecnolías móviles, utilizando la mejores herramientas, metodologías y estándares de calidad en la industria" \cite{DIG1}.
\section{Visión} \label{Vision}
"Ser el líder en Latinoamérica en brindar soluciones innovadoras basadas en tecnologías móviles, con un equipo creativo con los conocimientos más sólidos en tecnologías de información para móviles, para cometir en el mercado global" \cite{DIG1}.
\section{Estructura organizacional} \label{Estuctura organizacional}

La estructura organizacional de la empresa está comprendida principalmente por la Junta de Directores, como se muestra en la figura 1.1.

\begin{figure}[ht]
  \centering
  \includegraphics[scale=0.45,type=png,ext=.png,read=.png]{imagenes/estructura_empresa}
  \caption{Estructura organizacional de la empresa}
  \label{fig:estructuraEmpresa}
\end{figure}

La Junta de Directores está compuesta por tres cargos: Director de Operaciones (COO por sus siglas en inglés), Director Ejecutivo (CEO por sus siglas en inglés) y Director de Tecnología (CTO por sus siglas en inglés).

El COO se encarga de gestionar las adquisiciones de la empresa y de mantener la infraestructura tecnológica.

El CEO se encarga de manejar la Dirección de Ventas, dirigir el Departamento Legal junto al abogado y dirigir el Departamento de Finanzas y Recursos Humanos junto al contador.

El CTO se encarga de dirigir el departamento de desarrollo,por lo que el equipo de desarrollo de \textit{software} se comunica y le rinde cuentas a él.



\section{Ubicación del pasante} \label{Ubicacion del pasante}

Los pasantes se encuentran en el Departamento de Desarrollo, dirigido por el CTO, y donde se encuentan los especialistas de todas las áreas de desarrollo de \textit{software}. Esto permite que los pasantes estén en constante contacto con personas que puedan asistirlos y solventar las dudas que puedan surgir durante el desarrollo.



 % Marco Teorico.
\chapter{Marco Teórico} \label{chap:Marco Teorico}

En este capítulo se exponen los conceptos teóricos estudiados, tanto para entender las tecnologías utilizadas como para realizar el desarrollo de \textit{software} requerido. Los conceptos estudiados formaron parte fundamental del proceso de diseño y desarrollo de la solución.

\section{Patrones de diseño} \label{sect:Patrones de diseno}

Un patrón de diseño es una solución general y repetible a problemas que suelen presentarse en el proceso de diseño de software. Para que una solución pueda ser considerada como un patrón de diseño debe ser reutilizable, es decir, que se pueda aplicar a diferentes problemas de diseño en diferentes situaciones \cite{DSP0}. A continuación se describen los patrones utilizados en el diseño de la solución del proyecto.

\subsection{Modelo Vista Controlador (MVC)}

Es un patrón de arquitectura de \textit{software} que divide una aplicación  en tres partes que están interconectadas: los datos y la lógica de negocio (modelo), la interfaz de usuario (vista) y el módulo encargado de gestionar las comunicaciones (controlador). Esto se utiliza para separar la representación de la información de la forma en que dicha información es presentada al usuario.

\subsection{Modelo Vista Presentador (MVP)}

Es una derivación del patrón Modelo Vista Controlador (MVC) que se utiliza generalmente para construir interfaces de usuario \cite{MVP0}. En este patrón, la interacción entre el modelo y la vista se logra únicamente a través del presentador, mientras que en el MVC la vista en ocasiones puede comunicarse directamente con el modelo. Otra diferencia con el patrón MVC es que en este último las peticiones son recibidas por el controlador, éste se comunica con el modelo para pedir los datos y luego se encarga de mostrar la vista adecuada. En el patrón MVP las peticiones son recibidas por la vista y delegadas al presentador, que es quien se comunica con el modelo para obtener los datos \cite{MVP1}.

\subsection{\textit{Singleton}}

El \textit{singleton} es un patrón de diseño que asegura que la clase sólo tiene una instancia y provee un acceso global a la misma \cite{DSP1}. Esto es útil cuando se necesita únicamente un objeto para coordinar las acciones en el sistema \cite{SNG0}.

\subsection{\textit{Adapter}}

Transforma la interfaz de una clase en otra interfaz que el cliente espera. Esto permite que una clase que no pueda utilizar la primera interfaz, sí pueda hacerlo a través de la otra \cite{DSP1}. 


\section{Patrones de arquitectura} \label{sect:Patrones de arquitectura}
Un patrón de arquitectura es una solución a problemas de arquitectura de \textit{software}, permitiendo definir una estructura. La diferencia entre los patrones de diseño y los de arquitectura es que estos últimos tienen un nivel de abstracción mayor. \cite{PDA1}

\subsection{Modelo Vista Controlador (MVC)}

Es un patrón de arquitectura de \textit{software} que divide una aplicación  en tres partes que están interconectadas: los datos y la lógica de negocio (modelo), la interfaz de usuario (vista) y el módulo encargado de gestionar las comunicaciones (controlador). Esto se utiliza para separar la representación de la información de la forma en que dicha información es presentada al usuario \cite{MVC0}.

\subsection{Modelo Vista Presentador (MVP)}

Es una derivación del patrón Modelo Vista Controlador (MVC) que se utiliza generalmente para construir interfaces de usuario \cite{MVP0}. En este patrón, la interacción entre el modelo y la vista se logra únicamente a través del presentador, mientras que en el MVC la vista en ocasiones puede comunicarse directamente con el modelo. Otra diferencia con el patrón MVC es que en este último las peticiones son recibidas por el controlador. Éste, se comunica con el modelo para pedir los datos y luego se encarga de mostrar la vista adecuada. En el patrón MVP las peticiones son recibidas por la vista y delegadas al presentador, que es quien se comunica con el modelo para obtener los datos \cite{MVP1}.
%\input{marco_teorico/2_SOA.tex}
\section{Arquitectura cliente-servidor} \label{sect:Patrones de arquitectura}

Es un modelo de arquitectura de sistema distribuido, donde existe un conjunto de procesos que proporcionan servicios (servidores) a otros procesos (clientes). Por lo general, el servidor no conoce la identidad ni el número de sus clientes, pero el cliente sí conoce la identidad del servidor \cite{ACS0}. Dentro de esta arquitectura existen cuatro tipos \cite{ACS1}:

\begin{itemize}
\item Cliente activo, servidor pasivo: El cliente procesa toda la información.
\item Cliente pasivo, servidor pasivo: Tanto el cliente como el servidor pasan información.
\item Cliente pasivo, servidor activo: El servidor procesa toda la información y el cliente presenta los datos.
\item Cliente activo, servidor activo: Tanto el servidor como el cliente procesan la información.
\end{itemize}
\section{Servicio web} \label{Servicio web}

Es un conjunto de funcionalidades diseñadas para permitirs la comunicación entre diferentes dispositivos a través de una red, utilizando un conjunto de protocolos y estándares \cite{WBS0}.
\section{HTTP (\textit{Hypertext Transfer Protocol})}\label{HTTP}

Es el protocolo de comunicación que permite la transferencia de recursos en la web. Un recurso es cualquier información que puede ser identificada por un URL. Existen diversos métodos que permiten realizar peticiones con el protocolo HTTP, entre los que está POST \cite{HTTP2}. 

Dentro de una petición HTTP suelen enviarse encabezados (\textit{headers}) que proveen información de la misma, entre los que se encuentra el \textit{Content-Type}. Con éste, se especifica el tipo de codificación que se usará en el cuerpo de la petición. Estos tipos pueden ser: application/xml, application/json, text/html, multipart/form-data, etc. 

El tipo multipart permite enviar mensajes con varias partes combinadas en un solo cuerpo, y suele usarse también para el envío de archivos. \cite{HTTP1}

\subsection{HTTP POST}

Es uno de los diferentes métodos de peticiones que soporta el protocolo HTTP, que permite enviar datos a un recurso, los cuales deben ser procesados. Se requiere que los datos sean enviados dentro del cuerpo del mensaje \cite{HTTP3}.

%Para enviar estos datos dentro de una petición POST, existen diferentes tipos de codificación. Entre estos tipos está Multipart, que permite enviar mensajes con varias partes combinadas en un solo cuerpo \cite{HTTP1}.

\input{marco_teorico/6_json.tex}
\section{POJO (\textit{Plain Old Java Object})}

Son las siglas utilizadas para referirse a objetos de Java que no extiende ninguna clase ni implementa alguna interfaz \cite{POJO0}.
\section{DAO (\textit(Data Access Object))}

Es un objeto que provee una interfaz abstracta para la comunicación entre la aplicación y algún mecanismo de persitencia, como una base de datos. Se utilizan para permitir realizar operaciones sobre la base de datos, sin exponer detalles de la misma \cite{DAO0}.
\section{\textit{Manager}}

Es un objeto que se utiliza para actuar como puente entre la capa de persistencia (base de datos) y la capa web. Dentro de este componente se incluye la lógica de negocios de la aplicación. Se suele implementar en conjunto con DAO's. \cite{MNG0}

 % Marco Teorico.
\chapter{Marco Tecnológico} \label{chap:Marco Tecnologico}
En este capítulo se definen las diferentes herramientas utilizadas a lo largo de todo el proyecto. Dichas herramientas permitieron el desarrollo del \textit{software} planteado como solución al problema.
\vspace{5 mm}



\section{Herramientas, entornos y lenguajes} \label{Herramientas, entornos y lenguajes}


\subsection{Android}
Es un sistema operativo de código abiertos basado en el sistema operativo Linux, utilizado principalmente para dispositivos móviles\cite{AND1}. Es la plataforma para la cual se desarrolló la aplicación. 

La plataforma de Android está basada en una arquitectura que tiene diferentes capas que van desde servicios de bajo nivel del sistema operativo, que interactúan directamente con el \textit{hardware}, hasta las aplicaciones. Además, Android provee un kit de desarrollo de \textit{software} (SDK por sus siglas en inglés), que permite crear aplicaciones \cite{AND3}.

\begin{figure}[ht]
  \centering
  \includegraphics[scale=0.6,type=png,ext=.png,read=.png]{imagenes/android_architecture}
  \caption{Arquitectura de la plataforma de Android}
  \label{fig:androidArchitecture}
\end{figure}

La capa de más bajo nivel es la del kernel de Linux, como se puede observar en la figura 3.1. Provee un nivel de abstracción para la comunicación con el \textit{hardware} del dispositivo, y contiene los \textit{drivers} esenciales para el funcionamiento de dicho \textit{hardware}. Además, provee servicios del sistema operativo, como el manejo de memoria y procesos \cite{AND3}.

Encima de esta capa se encuentra la de librerías, dentro de las cuales se encuentra SQLite, que permite guardar datos de las aplicaciones \cite{AND3}.

En conjunto con la librerías, está la sección de Android \textit{Runtime} en el segundo nivel. Dentro de ella se encuentra un componente llamado Máquina Virtual de Dalvik (DVM por sus siglas en inglés), que es una máquina virtual de Java diseñada y optimizada especialmente para Android. Además de la DVM, provee un conjunto de librerías que permite desarrollar aplicaciones en Android utilizando el lenguaje de programación Java \cite{AND3}.

La tercera capa de abajo hacia arriba corresponde al \textit{framework} de las aplicaciones. En ella se proveen servicios de alto nivel mediante clases de Java. Estos servicios pueden ser utilizados dentro de las aplicaciones. Dentro de estos servicios destaca el Manejador de Actividades (\textit{Activity Manager}), que se encarga de controlar todas las actividades de una aplicación. Una actividad (\textit{activity}) es un componente que provee una pantalla con la cual el usuario final puede interactuar. Generalmente, a cada actividad corresponde una interfaz de usuario \cite{AND3}.

Por último, se encuentra la capa de aplicaciones, que es donde se encuentran instaladas las aplicaciones \cite{AND3}.

\subsection{Java}
Es un lenguaje de programación de alto nivel orientado a objetos, que puede correr virtualmente en cualquier computadora \cite{JAV1}. Es el lenguaje en el que están desarrolladas la mayoría de las aplicaciones de Android \cite{AND2}. Se utilizó tanto para la programación de la aplicación móvil como del servidor.

\subsection{MySQL} 
Es un sistema de manejo de base de datos relacional basado en el Lenguaje de Consulta Estructurado (SQL por sus siglas en inglés) \cite{SQL1}. Se utilizó para la gestión de la base de datos del servidor.

\subsection{SQLite}
Es un sistema de manejo de base de datos relacional basado en el Lenguaje de Consulta Estructurado (SQL por sus siglas en inglés) que forma parte del programa principal, en lugar de ser un proceso independiente como ocurre con los sitemas de gestión de base de datos cliente-servidor \cite{SQL2}.

\subsection{Maven}
Es una herramienta que se permite compilar y  manejar proyectos de \textit{software} basados en Java. Se encarga de conectar las dependencias de los paquetes \cite{MVN1}.

\subsection{Jetty}
Es un servidor HTTP que puede funcionar independientemente por sus propios medios, o que puede correr dentro de otra aplicación \cite{JTY1}.

\subsection{Android Studio}
Es el entorno de desarrollo integrado (IDE por su siglas en inglés) oficial para el desarrollo de aplicaciones nativas para Android \cite{ASD1}.

\subsection{Eclipse}
Es un entorno desarrollo integrado (IDE por sus siglas en inglés) utilizado principalmente para desarrollar aplicaciones en Java \cite{ECL1}.

\subsection{Git}
Es un sistema de control de versiones distribuido (DVCS por sus siglas en inglés). Es una herramienta que permite gestionar las distintas versiones de un proyecto de \textit{software} \cite{GIT1}.

%\input{marco_tecnologico/2_librerias.tex}
\section{\textit{Frameworks} y librerías} \label{Frameworks y librerias}

\subsection{AppFuse}
Es un \textit{framework} utilizado para la creación de aplicaciones web en la máquina virtual de Java (JVM por sus siglas en inglés). Dentro de AppFuse, se pueden integrar otras tecnologías como Bootstrap, Maven, Hibernate, Spring, Tapestry, etc \cite{APF1}.

\subsection{Hibernate}
Es una herramienta de mapeo objeto-relacional (ORM por sus siglas en inglés) que permite el mapeo de objetos de Java a bases de datos relacionales \cite{HBR1}.

\subsection{Spring}
Es un \textit{framework} para el desarrollo de aplicaciones en Java.

\subsection{Tapestry}
Es un \textit{framework} para el desarrollo de aplicaciones web en Java, Groovy o Scala.  \cite{ATP1}.

\subsection{Gson}
Es una librería de Java que permite la conversión de objetos a JSON y viceversa \cite{GSN1}
.
\subsection{Retrofit}
Es una librería de Android que permite realizar peticiones HTTP mediante una interfaz de Java\cite{RFT1}.
%\section{Elasticsearch} \label{sect:Elasticsearch}

%\section{Logstash} \label{sect:Logstash}

%\section{Kibana} \label{sect:Kibana}


%\section{NGINX} \label{sect:NGINX}

%\section{Xen Project} \label{sect:Xen Project}



 % Marco Metodologico.
\chapter{Marco Metodológico} \label{chap:Marco Metodologico}

A continuación se describe el procedimiento seguido para el desarrollo del proyecto. Se decidió utilizar el marco de trabajo Scrum, dado que éste es utilizado por la empresa para el desarrollo de \textit{software}.
\section{Scrum} \label{sect:Scrum}

Scrum es un marco de trabajo que se basa en el desarrollo iterativo e incremental de un producto, en lugar del modelo clásico de planificación y ejecución completa. \cite{SCRM0} Se caracteriza por ser una metodología ligera, fácil de entender y difícil de dominar, que permite entregar incrementos de producto potencialmente productivos. \cite{SCRM1}

\subsection{Roles} 

En Scrum el desarrollo se realiza por uno o más equipos de trabajo dentro de los cuales existen tres roles: \textit{Product owner} (jefe del producto), \textit{ScrumMaster} (jefe de Scrum) y el equipo de desarrollo. \cite{SCRM12}
 
\subsubsection{\textit{Product owner}}

Es el representante de los clientes. Dentro del equipo de Scrum, es el líder principal del producto y el responsable de decidir qué funcionalidades serán desarrolladas y la prioridad que tendrá cada una de ellas. Debe comunicar al resto de los involucrados en el proyecto una visión clara de lo que se quiere lograr. Tiene la obligación de asegurar que siempre se entregue un producto con el máximo de valor, por lo que debe colaborar con el resto del equipo para responder cualquier duda que surja. \cite{SCRM12}

\subsubsection{\textit{ScrumMaster}}

Actúa como facilitador tanto para el \textit{product owner} como para el equipo de desarrollo. Es el encargado de ayudar al resto del equipo a entender y cumplir con los principios y prácticas de Scrum.También tiene la responsabilidad de eliminar cualquier impedimento que el equipo no sea capaz de resolver y que afecte su productividad. \cite{SCRM12}

\subsubsection{Equipo de desarrollo}

Es el encargado de desarrollar el producto. Es un equipo que está compuesto por arquitectos, programadores, probadores, administradores de base de datos, diseñadores de interfaces, entre otros. Son los responsables de diseñar, desarrollar y probar el producto. \cite{SCRM12}

\subsection{Actividades}

En Scrum, el trabajo se desarrolla en interaciones de una duración máxima de un mes, llamadas \textit{\textbf{sprints}}. Al final de cada \textit{sprint}, se debe haber desarrollado una parte del producto final, la cual debe ser completamente funciona. Dentro de cada iteración existe una serie de eventos o actividades que se llevan a cabo: el \textit{sprint planning}, la ejecución del \textit{sprint}, el \textit{daily scrum} y el \textit{sprint review}.\cite{SCRM12}

\subsubsection{\textit{Sprint planning}}s
Para determinar qué funcionalidades del producto final son las más importantes y próximas a desarrollar, el equipo de trabajo (\textit{product owner}, \textit{ScrumMaster} y el equipo de desarrollo) realizan una reunión llamada \textit{sprint planning}.\cite{SCRM12}

Durante la reunión, el \textit{product owner} y el equipo de desarrollo establecen una meta que debe ser cumplida para el final del \textit{sprint}. De acuerdo a esta meta, el equipo de desarrollo decide de una manera realista qué incrementos del producto final pueden entregarse al terminar el \textit{sprint}.\cite{SCRM12}

\subsubsection{Ejecución del \textit{sprint}}

Luego del \textit{sprint planning}, el equipo de desarrollo desarrolla todas las tareas acordadas en la reunión. Esto es lo que se conoce como la ejecución del \textit{sprint}.\cite{SCRM12}

\subsubsection{\textit{Daily scrum}}

Cada día dentro de la ejecución del \textit{sprint}, los miembros del equipo de desarrollo se reúnen durante un máximo de 15 minutos con el fin de informar qué se hizo el día anterior, qué se tiene planificado realizar el presente día y qué impedimentos se han presentado durante el desarrollo de su trabajo.\cite{SCRM12}

\subsubsection{\textit{Sprint review}}

Al final de cada \textit{sprint}, ocurre un evento que se conoce como \textit{sprint review} o revisión del \textit{sprint}. El objetivo de esta actividad es revisar el incremento y realizar las adaptaciones necesarias al prouducto.\cite{SCRM12}

\subsection{Artefactos}

Dentro de Scrum existen dos herramientas o artefactos que permiten mantener un seguimiento del proyecto: el \textit{product backlog}y el \textit{sprint backlog}.\cite{SCRM2}

\subsubsection{\textit{Product backlog}}

Es una lista de los requerimientos funcionales del producto ordenados según su importancia. EL \textit{product owner} es el responsable de definir qué elementos serán incluidos en esta lista y de colocarlos según su prioridad, de manera que los elementos de mayor valor o prioridad aparezcan al principio de la lista, y los de menos valor al final de la misma. \cite{SCRM2}

\subsubsection{\textit{Sprint backlogs}}

Es una lista donde se presenta un subconjunto de los elementos del \textit{product backlog} divididos en tareas más pequeñas. \cite{SCRM2}




\chapter{Desarrollo de la aplicacion}\label{chapter:Desarrollo de la aplicacion}

En este capítulo se describe el trabajo realizado para el desarrollo de la aplicación. Este proceso se dividió en tres fases: investigación, dise~no y desarrollo del sistema que incluye tanto la aplicación móvil como el componente servidor.

El proyecto se desarrollo bajo el marco de trabajo de \textit{Scrum}, descrito en el capitulo 1. 
\section{Investigación} \label{sect:Investigacion}

El objetivo de esta primera fase era familiarizarse con el entorno de trabajo.

En primer lugar se estudiaron diversos patrones de diseño. Para esto, se investigó acerca de patrones ampliamente utilizados en el desarrollo de aplicaciones para Android y en la programación orientada a objetos en general. Esto permitió estructurar la aplicación de una manera más ordenada y entendible, de manera que se pueda extender fácilmente.

Luego de esto, se investigó acerca del desarrollo de aplicaciones para Android. Se aprendió a trabajar con el IDE Android Studio, así como los componentes y herramientas para el desarrollo bajo el sistema operativo.

Por último, se realizó una aplicación de prueba para poner en práctica algunos conceptos básicos en el desarrollo de aplicaciones para Android. Con esto se aprendió la estructura general de una aplicación, así como el flujo de ejecución.



\section{Diseno} \label{sect:Diseno}
\section{Desarrollo} \label{sect:desarrollo}

Durante esta fase se implementó progresivamente la versión alfa del prototipo funcional. 

Se creó un componente móvil y un servidor para dar solución al problema de registro y control de gastos.
Para cumplir con los requerimientos del componente móvil, se desarrolló una aplicación para el sistema operativo Android, y se trabajó con el IDE Android Studio.

Se implementaron vistas para crear, editar y eliminar gastos. Para estos gastos, se debe guardar su monto, fecha, descripción, fotos de recibos y categoría a la que pertenecen. Posteriormente, se decidió extender la funcionalidad de creación de gastos para permitir también el manejo de ingresos. 

También se crearon vistas para el manejo de categorías, que incluyen la creación, edición y eliminación de las mismas.

Luego de la creación de las principales funcionalidades del sistema, mencionadas en los párrafos anteriores, se introdujo el concepto de "cuentas". Hasta el momento, los gastos estaban asociadosa una única cuenta por defecto. Por esta razón, también se creó una funcionalidad para agregar nuevas cuentas, y poder crear nuevos gastos dentro de dichas cuentas.

Como se explicó en la sección anterior, es necesario guardar toda esta información en la base de datos local de cada dispositivo. Para ello, se utilizó la librería SQLite.

Además, también se crearon dos vistas para reportes de gastos: una para reportes de balance general, y otra para reporte de gastos por categorías.
La vista de reporte de balance general muestra una lista de gastos e ingresos filtrados por fecha. La vista de reporte por categorías muestra una lista de categorías con la suma total de los gastos e ingresos de dichas categorías, filtradas por fecha.

Para mantener la lógica separada de la interfaz gráfica, se utilizó el patrón de arquitectura MVP o Modelo Vista Presentador, descrito en el capítulo 2. De esta manera, toda acción en la vista que requiera realizar operaciones sobre la base de datos, se realizó por medio de una capa intermedia: el presentador.

Por otra parte, se creó un servidor al que se pueda enviar información desde la aplicación móvil. 

Para permitir el acceso a la aplicación móvil, se creó un servicio web que permita hacer la autenticación del usuario. La aplicación envia al servidor un JSON con los datos del usuario (nombre de usuario y contraseña) por medio de una petición HTTP POST.

Se creó un servicio web que recibe un archivo PDF que contiene el reporte de gastos. Además del archivo, recibe la suma de gastos por cada categoría y el total de gastos. Estos datos se enviaron por medio de una petición HTTP Post Multipart, compuesta por un archivo y un JSON. Para la conversión de objetos de Java a JSON, y viceversa, se utilizó la librería GSON tanto en el servidor como en la aplicación móvil. Para persistir esta información, se utilizó Hibernate y MySQL como manejador de base de datos.

También se creó una aplicación web que permita aprobar o rechazar los reportes enviados al servidor.

El desarrollo del componente servidor se realizó con el IDE Eclipse. Se utilizó el \textit{framework} AppFuse, que integra a su vez otros \textit{frameworks} que facilitan el desarrollo web. Se utilizó Spring para la implementación de los servicios web que permiten recibir datos desde la aplicación móvil. Para crear la aplicación web, se utilizó Tapestry. Para el despliegue del servidor, se utilizó Maven y Jetty.

El proyecto se desarrolló bajo el marco de trabajo de \textit{Scrum}, descrito en el capitulo 1. Durante la ejecución de las iteraciones, se implementaron las funcionalidades planificadas, así como las pruebas necesarias para validarlas.

A continuación se presentarán las iteraciones (\textit{sprints}), durante las cuales se desarrollaron las funcionalidades descritas en los párrafos anteriores. Para cada iteración, se mencionarán sus objetivos, resultados y una breve descripción de las actividades realizadas.

\subsection{Sprint 1}

\subsubsection{Objetivos}
	\begin{itemize}
	\item Mostrar información del balance de una cuenta
	\item Permitir la creación de un nuevo gasto
	\end{itemize}

\subsubsection{Resultados}
\begin{itemize}
\item Se creó una vista para mostrar el total de ingresos,gastos y balance general de una cuenta.
\item Se creó una vista para permitir la creación de un nuevo gasto.
\end{itemize}

\subsubsection{Actividades}
En la primera parte del \textit{sprint} se implementó el modelo de datos del dispositivo móvil.

Se creó la vista principal de la aplicación móvil, donde se muestra el total de ingresos y el de gastos, así como el balance general. Para esto, fue necesario crear consultas a la base de datos que permitieran obtener el total de ingresos/gastos de una cuenta.  También se realizó una consulta que permite obtener el balance total de una cuenta.

Por otro lado se implementó una vista que permite crear un nuevo gasto, con su fecha, monto y descripción.

Como se expuso anteriormente, la comunicación entre el modelo y la vista se hace a través de una capa intermedia: el presentador. Por esta razón, también fue necesaria la creación de un presentador para cada vista que permita hacer consultas a la base de datos.

\subsection{Sprint 2}

\subsubsection{Objetivos}
\begin{itemize}
\item Mostrar la lista de categorías
\item Permitir la creación de un nuevo ingreso
\item Permitir guardar fotos de un $\entry$
\item Asociar un ingreso/gasto a una categoría
\end{itemize}

\subsubsection{Resultados}
\begin{itemize}
\item Se creó una vista para mostrar la lista de categorías a las que puede pertenecer un gasto.
\item Se adaptó y reutilizó la vista existente para crear un gasto para permitir la creación de un ingreso
\item Se agregó la funcionalidad para tomar fotos relacionadas a un ingreso/gasto
\item Se agregó la funcionalidad para poder asociar un ingreso/gasto a una categoría existente

\end{itemize}

\subsubsection{Actividades}

En este \textit{sprint} se creó la vista para mostrar las categorías presentes en el dispositivo.

Se creó una consulta para guardar en la base de datos una nueva categoría con su nombre y un ícono que la represente. Además, se creó una consulta para guardar en la base de datos una lista de categorías por defecto.

También se adaptó la vista existente para la creación de un gasto, de manera que se pueda reutilizar esta vista para la creación de un ingreso. 

Por otra parte, se implementó la funcionalidad para tomar y guardar fotos asociadas a un ingreso/gasto. Para esto se tuvo que crear un método que permita abrir la aplicación de la cámara del dispositivo desde la aplicación móvil. Luego, se tuvo que adaptar la vista para la creación de un ingreso/gasto para mostrar las miniaturas de las fotos tomadas. Se puso un límite por parte de la empresa para que el máximo de fotos que se puedan guardar por ingreso/gasto sean 3.

Por último, se modificó la vista para la creación de un ingreso/gasto para que se permitiera asociar una categoría. Se tenía como requerimiento guardar las cuatro categorías que se utilizaron más recientemente. Para esto, se tuvo que crear los métodos necesarios para persistir en el dispositivo móvil las categorias más recientes. Dado que esta información no tiene una estructura compleja, se decidió utilizar un objeto que provee la plataforma de Android, llamado \textit{SharedPreferences}, que permite guardar un conjunto de pares clave-valor. En este caso, se guardó una lista con las categorías más recientes.

\subsection{Sprint 3}
\subsubsection{Objetivos}
\begin{itemize}
\item Mostrar la lista de ingresos/gastos del mes actual
\item Mostrar los detalles de los ingresos/gastos existentes
\item Permitir editar y borrar un ingreso/gasto existente
\item Implementar una calculadora para guardar el monto de un ingreso/gasto
\end{itemize}

\subsubsection{Resultados}
\begin{itemize}
\item Se creó la vista para mostrar una lista con los gastos y otra con los ingresos del mes actual.
\item Se agregó la funcionalidad para ver los detalles de un ingreso/gasto ya existente, y poder editarlo
\item Se agregó la funcionalidad para eliminar ingresos/gastos
\item Se creó la interfaz para usar la calculadora que permita ingresar el monto de un ingreso/gasto
\end{itemize}

\subsubsection{Actividades}
En primer lugar, se adaptó la vista principal de la aplicación para mostrar el total de ingresos y gastos únicamente durante el mes actual. Para esto, se tuvo que modificar la consulta a la base de datos para obtener el total de ingresos/gastos dado un rango de fecha.

Se implementó la funcionalidad para navegar a la lista de ingresos o de gastos del mes desde la vista principal de la aplicación. Para esto, se creó una vista general donde se pueda mostrar una lista de ingresos/gastos, con su descripción (o categoría, en caso de que no tenga descripción), su fecha y su monto.

Por otra parte, se creó la funcionalidad para eliminar ingresos/gastos existentes desde la vista mencionada en el párrafo anterior.

También se creó la funcionalidad para editar un ingreso/gasto existente. Para esto no fue necesario crear una vista nueva, sino que se adaptó la que se tenía para la creación de un nuevo ingreso/gasto, de manera que se muestre la información relacionada adicho $entry$ (fecha, monto, categoría, descripción y fotos).

Por último, se tenía como requerimiento mostrar una calculadora para ingresar el monto de un ingreso/gasto, en lugar de un teclado numérico común. Por esta razón, se tuvo que modificar la vista de creación de un ingreso/gasto para incluir la nueva funcionalidad. 

Para este último requerimiento fue necesario crear un teclado virtual personalizado, pues el teclado numérico del dispositivo no incluye los operadores matemáticos básicos. Se creó el formato del teclado, el cual incluye las teclas presentes en una calculadora básica. También se tuvo que manejar el uso de las teclas: detectar qué tecla fue presionada y realizar una acción en base a esta. Se utilizó una librería para evaluar fórmulas dada una cadena de caracteres.

Para la calculadora, se tenía que mostrar el símbolo decimal de acuerdo con el país para el cual está configurado el teléfono, ya que en algunos países se usa la coma (,) como separador y en otros el punto (.). Por esta razón, se tuvo que obtener el país de configuración del teléfono para decidir qué símbolo decimal mostrar.



\subsection{Sprint 4}
\subsubsection{Objetivos}
\begin{itemize}
\item Permitir el manejo de nuevas cuentas
\end{itemize}

\subsubsection{Resultados}
\begin{itemize}

\item Se creó una vista para crear una nueva cuenta
\item Se implementó la funcionalidad para listar las cuentas del usuario 
\item Se implementó la funcionalidad para cambiar de cuenta
\item Se implementó la funcionalidad para editar el nombre de una cuenta existente
\item Se implementó la funcionalidad para eliminar cuentas existentes
\item Se implementó la funcionalidad para cambiar el estado de una cuenta: activa o archivada
\item Se implementó la funcionalidad para ver la lista de ingresos/gastos de una cuenta desde su creacion hasta la fecha actual

\end{itemize}

\subsubsection{Actividades}
En la ejecución de los \textit{sprints} anteriores se estuvo trabajando con una sola cuenta creada por defecto en la aplicación. Para este \textit{sprint}, se decidió agregar la funcionalidad para poder manejar nuevas cuentas. 

Para esto, se creó la vista que permite agregar una nueva cuenta, con su nombre y la moneda en la que van a estar sus ingresos/gastos.

Se creó una vista para mostrar la lista de cuentas guardadas. Se implementó la funcionalidad para cambiar el estado de una cuenta: activa o archivada. La lista de cuentas se muestra segun su estado, es decir, se muestra una lista para la lista de cuentas activas y una para las cuentas archivadas. 

Además, se implementaron las funcionalidades para editar el nombre de una cuenta y eliminar cuentas ya existentes.

También se creó la funcionalidad para cambiar la cuenta que se está mostrando actualmente en el dispositivo.

Por último, se adapto la vista existente para mostrar la lista de ingresos/gastos, de manera que se pueda mostrar todos los ingresos/gastos (en una misma lista) asociados a la cuenta que se muestra actualmente.




\subsection{Sprint 5}
\subsubsection{Objetivos}
\begin{itemize}
\item Permitir el manejo de las fotos de un ingreso/gasto
\item Permitir el manejo de las categorias
\end{itemize}

\subsubsection{Resultados}
\begin{itemize}
\item Se agregó la funcionalidad para ver las fotos de un ingreso/gasto
\item Se agregó la funcionalidad para borrar las fotos de un ingreso/gasto
\item Se agregó la funcionalidad para cambiar la ubicacion en el dispositivo de las fotos tomadas de un ingreso/gasto
\item Se agregó la funcionalidad para crear una categoría
\item Se agregó la funcionalidad para eliminar una categoría existente
\item Se agregó la funcionalidad para editar una categoría existentes
\end{itemize}

\subsubsection{Actividades}
La primera parte del \textit{sprint} se dedicó al manejo de las fotos. En primer lugar se creó la vista para poder ver en pantalla completa las fotos de un ingreso/gasto (esto se hace desde la vista de creación de un ingreso/gasto). También se implementó la funcionalidad para eliminar una foto.

Por otra parte, se creo la vista de configuraciones de la aplicación. En esta vista se agregó una opción para cambiar la ubicación en el dispositivo en la que se guardan los archivos de las fotos: memoria interna o memoria extraíble.

La egunda parte del \textit{sprint} se dedicó al manejo de las categorías. Se creó la interfaz para agregar una nueva categoría. Por último, se agregaron las funcionalidades para editar y eliminar categorias existentes.


\subsection{Sprint 6}
\subsubsection{Objetivos}
\begin{itemize}
\item Mostrar un reporte con la lista de ingresos/gastos asociados a una cuenta, en un rango de fecha dado
\item Mostrar un reporte con el balance total por categorías asociadas a una cuenta, en un rango de fecha dado
\item Permitir el envío de los reportes en un archivo PDF a otras personas

\end{itemize}

\subsubsection{Resultados}
\begin{itemize}
\item Se implementó la funcionalidad para listar los ingresos/gastos de una cuenta en el rango de fecha ingresado por el usuario
\item Se implementó la funcionalidad para listar el balance total de las categorías de una cuenta en el rango de fecha ingresado por el usuario
\item Se creó la funcionalidad para compartir el reporte con la lista de $\entries$,incluyendo las fotos, de la cuenta en el rango de fecha ingresado.
\item Se creó la funcionalidad para compartir el reporte con la lista de $\entries$, sin incluir las fotos, de la cuenta en el rango de fecha ingresado.
\item Se creó la funcionalidad para compartir el reporte con el balance total de las categorías de la cuenta en el rango de fecha ingresado.
\end{itemize}

\subsubsection{Actividades}
Este \textit{sprint} se dedicó exclusivamente al manejo de los reportes. Se tenía como requerimiento el manejo de dos tipos de reportes: reporte de balance general y reporte por categorías. Para la generación de ambos reportes, se puede especificar una cuenta y un rango de fechas. El reporte de balance general incluye la lista de todos los ingresos/gastos de la cuenta escogida, dentro del rango de fechas establecido. El reporte por categorías incluye únicamente el monto total de todos los ingresos/gastos (es decir, la suma de ingresos y gastos) divididos por categorías, de la cuenta y rango de fechas escogidos.

Se creó una vista para mostrar el reporte de balance general, que muestra una lista con los ingresos/gastos asociados, sus montos, fechas y descripción (o categoría, en caso de que la descripción sea nula). También se creó una vista para el reporte por categoría, donde se muestra una lista con con el nombre de cada categoría y la suma total de los ingresos y gastos correspondientes.

Luego, se creó la funcionalidad para generar un archivo PDF con la información de los reportes descritos anteriormente. Para esto se usó una librería que provee la plataforma de Android para la creación de archivos PDF.  

Se tenía como requerimiento que el usuario pueda escoger generar un archivo con el reporte de balance general incluyendo las fotos de los ingresos/gastos asociados, o sin incluirlas.

Luego se implementaron los métodos necesarios para crear un archivo PDF con el reporte del balance general. Esta funcionalidad luego se adaptó para permitir también la creación del reporte por categorías. Para la escritura del PDF, se tuvo que lidiar manualmente con la paginación. 

Por último, se creó la funcionalidad para compartir este archivo PDF con otras personas a través de otras aplicaciones instaladas en el dispositivo móvil. Dentro de estas aplicaciones se incluyen las de correo electrónico y mensajería móvil que soporten el envío de archivos.

Durante la creacion del archivo PDF se presentaron diferentes dificultades. En primer lugar, las librerias existentes en Java para generar archivos con dicha extension hacen uso de otras librerias, las cuales no son soportadas directamente en la plataforma de Android. Por esta razon, en un principio se decidio utilizar una adaptación de una librería de Java para poder ser utilizada en Android. Sin embargo, no se encontró documentación suficiente que permitiera entender su uso. Por esta razón, se decidió finalmente utilizar una librería nativa de Android que facilita la creación de archivos PDF, pero que sólo está disponible para versiones de Android a partir de la 4.4. Esta decisión se apoyó en el hecho de que aproximadamente el 78,5\% de los usuarios de Android utilizan una versión mayor o igual a 4.4 \cite{USG1}. 


\subsection{Sprint 7}
\subsubsection{Objetivos}
\begin{itemize}
\item Crear un servicio web para permitir la autenticación de un usuario
\item Crear un servicio web para recibir un reporte de gastos
\item Crear una aplicación web para mostrar los reportes recibidos
\item Permitir la aprobación o rechazo a través de la aplicación web de los reportes recibidos
\end{itemize}
\subsubsection{Resultados}
\begin{itemize}
\item Se diseñó e implementó el modelo de datos del servidor web
\item Se creó un servicio al cual la aplicación móvil se puede conectar para la autenticación de usuarios
\item Se creó un servicio a través del cual la aplicación móvil puede enviar reportes de gastos
\item Se creó una aplicación web, a través de la cual se permite la creación de nuevos usuarios y el manejo de reportes (listar, aprobar y rechazar reportes recibidos)
\end{itemize}

\subsubsection{Actividades}

En esta iteración se creó el servidor web al cual la aplicación móvil puede enviar peticiones.

En primer lugar, se creó el proyecto en AppFuse, con Tapestry como \textit{framework} para el desarrollo de la aplicación web. Utilizando Hibernate con MySQL, se crearon las tablas necesarias para implementar el modelo de datos mostrado en la figura 5.3.

Luego, se creó el servicio web para la autenticación de un usuario. 

Una vez creado, se implementó en la aplicación móvil una funcionalidad que permite iniciar sesión en el dispositivo. Para esto fue necesario crear una interfaz y los métodos necesarios para conectarse al servicio web para la autenticación del usuario.

Por otra parte, se creó un servicio web para recibir un reporte de gastos. Era necesario enviar tanto el archivo PDF con la información de los gastos, como un resumen de los totales por las categorías de interés para la empresa. Por esta razón, la comunicación entre la aplicación móvil y el servidor para el envío de reportes se hizo a través de una petición HTTP POST Multipart, que contiene el archivo PDF y un JSON con el resumen mencionado anteriormente.

También se creó un módulo para el envío de correos electrónicos. Esto se hizo para notificar al supervisor cuando un nuevo reporte llega al servidor, así como para enviar el archivo PDF adjunto al correo.

Luego de esto se creó la página web a través de la cual un supervisor puede ver la lista de reportes recibidos, así como aprobarlos o rechazarlos.

Para esto, se creó una vista que muestre la lista de reportes pendientes por reivisión (es decir, que aún no han sido aprobados ni rechazados). En esta vista se agregó una opción para poder descargar el archivo PDF de cada reporte, así como opciones para aprobarlo/rechazarlo.

Finalmente, se adaptó la vista mencionada anteriormente para poder mostrar una lista de los reportes ya revisados (es decir, que ya fueron aprobados o rechazados).


%\input{pruebas/0_pruebas.tex}
%
%\chapter{Retos Enfrentados y Logros Adicionales} 

%
\chapter{Conclusiones y Recomendaciones} \label{chap:conclusiones}

En este proyecto de pasantía, realizado para la empresa Digitalica Group C.A, se diseñó y desarrolló un prototipo funcional de una aplicación móvil para el registro de gastos e ingresos, así como un servidor web que se comunica con ella.  La aplicación fue desarrollada para el sistema operativo Android. Con ella, se permite registrar ingresos y gastos, a los cuales se puede asociar un monto, fecha, descripción, fotos y categoría. El servidor web permite hacer la autenticación de usuarios, así como el envío de reportes compuestos por un conjunto de gastos.

El uso de Scrum como marco de trabajo permitió tener 



% Crea el glosario 
%\makeglossaries
%\printglossaries

% Establece las citas y bibliografia
\bibliographystyle{ieeetr}
\bibliography{myrefs}

% Crea el apendice
%\appendix
%\includepdf[pages=-]{apendices/IFFv1.1/IFF.pdf}
%%\documentclass[12pt,letterpaper]{article}
%\usepackage[utf8]{inputenc}
%\usepackage{amsmath}
%\usepackage{amsfonts}
%\usepackage{amssymb}
%\usepackage[spanish]{babel}
%\usepackage{array}
%\usepackage{longtable}
%\author{Jon Ander Ricchiuti}
%\title{IFF v1.1}
%\begin{document}
%\pagenumbering{arabic}
%\maketitle
%\thispagestyle{empty}
%\newpage
\chapter{Intercambio de Información Financiera (IFF)}

\section*{Introducción}

El protocolo de intercambio de información financiera (IFF), describe la forma correcta de comunicación con un prototipo de sistema bancario creado en Synergy Global Business (SGB). El IFF busca estandarizar el intercambio de información financiera que realizará el prototipo bancario. De esta manera la comunicación es sencilla y a la vez robusta. Para la realización de este protocolo de comunicación se utilizó como base el IFX (Interactive Financial Exchange).
\\
\\ 
La forma de comunicación que el IFF	utiliza es de tipo petición-respuesta (request-response). En cada petición se debe especificar un método. Este método define la naturaleza de la petición.
%\pagenumbering{arabic}
%\newpage

\section{Tipo de datos}

Los tipos de datos que se utilizarán en este estándar son los siguientes:
\begin{itemize}
\item Cadena de caracteres.
\item Enumeración.
\item Tiempo y Hora.
\item Entero.
\item Decimal
\item Booleano.
\end{itemize}

\subsection{Cadena de caracteres}
Las cadenas de caracteres se representan con el nombre de ``Cadena de caracteres'' seguido de la longitud de la misma. Esta longitud es representada entre paréntesis de la siguiente forma.
\begin{itemize}
\item Cadena de caracteres (X-Y), indica que la longitud mínima de la cadena de caracteres es ``X'' y la máxima longitud es ``Y''.
\item Cadena de caracteres (X+), indica que la longitud mínima de la cadena de caracteres es ``X'' pero no tiene longitud máxima para la misma.
\end{itemize}
Si la longitud no es especificada entonces no existe restricción sobre el tamaño de la cadena de caracteres.

\subsection{Enumeración}
Son los valores que puede tomar un campo. Estos pertenecen a un conjunto de cadena de caracteres definidas específicamente para ese campo en particular. Los diferentes tipos de enumeración serán especificados en la siguiente sección.

\subsection{Tiempo y Hora}
Tanto la hora como la fecha son cadenas de caracteres que se representan con un fromato particular. Para la hora el formto es ``\%H:\%M:\%S''. Para la fecha el formato es ``\%Y-\%m-\%d \%H:\%M:\%S''. Donde \%Y representa el año, \%m el mes, \%d el dia, \%H la hora, \%M el minuto, \%S el segundo. Todos los componentes de la hora y fecha se representan con caracteres numéricos.

\subsection{Entero}
Es un número entero y puede representarse de dos formas diferentes. Como un entero de cuatro bytes o como un entero de ocho bytes. Para especificar que es un entero de ocho bytes, debe ser escrito de la siguiente forma: Entero(8).
\\
\\
Si no se especifica que un entero es de ocho bytes entonces se asume que es de cuatro  bytes.

\subsection{Decimal}
Es un número con hasta quince dígitos decimales.

\subsection{Booleano}
Un Booleano representa si una condición se cumple o no. En el caso del IFF un Booleano será representado por medio de un carácter. Es decir, el carácter 'T' será el que representa cuando un estado es cierto y 'F' será el valor de cuando el estado no se cumple.

\section{Tipos de Enumeración}
\subsection{Correspondiente a personVerifyType}
\begin{center}
\begin{tabular}{|>{\centering\arraybackslash}p{0.3\textwidth}|>{\centering\arraybackslash}p{0.3\textwidth}|>{\centering\arraybackslash}p{0.3\textwidth}|}
\hline 
\bfseries {Valor} & \bfseries {Descripción} & \bfseries {Por defecto} \\ 
\hline 
Passport & Pasaporte de crédito & N \\ 
\hline 
CI & Cedula de identidad & N \\
\hline 
\end{tabular} 
\end{center}

\subsection{Correspondiente a nameAddrType}
\begin{center}
\begin{tabular}{|>{\centering\arraybackslash}p{0.3\textwidth}|>{\centering\arraybackslash}p{0.3\textwidth}|>{\centering\arraybackslash}p{0.3\textwidth}|}
\hline 
\bfseries {Valor} & \bfseries {Descripción} & \bfseries {Por defecto} \\ 
\hline 
Customer & Es la dirección del cliente & N \\ 
\hline 
ShipTo & Dirección a la cual algo debería ser enviado por correo & N \\
\hline 
Delivery & Dirección a la cual serán enviadas las facturas en papel & N \\
\hline 
\end{tabular} 
\end{center}

\subsection{Correspondiente a addrType}
\begin{center}
\begin{tabular}{|>{\centering\arraybackslash}p{0.3\textwidth}|>{\centering\arraybackslash}p{0.3\textwidth}|>{\centering\arraybackslash}p{0.3\textwidth}|}
\hline 
\bfseries {Valor} & \bfseries {Descripción} & \bfseries {Por defecto} \\ 
\hline 
Seasonal & Habitación vacacional & N \\ 
\hline 
Primary & Habitación principal & N \\
\hline 
Secondary & Habitación secundaria & N \\
\hline
Business & Dirección de negocio & N \\
\hline 
\end{tabular} 
\end{center}

\subsection{Correspondiente a cardStatusCode}
\begin{center}
\begin{tabular}{|>{\centering\arraybackslash}p{0.3\textwidth}|>{\centering\arraybackslash}p{0.3\textwidth}|>{\centering\arraybackslash}p{0.3\textwidth}|}
\hline 
\bfseries {Valor} & \bfseries {Descripción} & \bfseries {Por defecto} \\ 
\hline 
Active & Activa & N \\ 
\hline 
Expired & Vencida & N \\
\hline 
Blocked & Bloqueada & N \\
\hline
\end{tabular} 
\end{center}

\subsection{Correspondiente a accountStatusCode}
\begin{center}
\begin{tabular}{|>{\centering\arraybackslash}p{0.3\textwidth}|>{\centering\arraybackslash}p{0.3\textwidth}|>{\centering\arraybackslash}p{0.3\textwidth}|}
\hline 
\bfseries {Valor} & \bfseries {Descripción} & \bfseries {Por defecto} \\ 
\hline 
Active & Activa & N \\ 
\hline 
Blocked & Bloqueada & N \\
\hline
\end{tabular} 
\end{center}

\subsection{Correspondiente a cardType}
\begin{center}
\begin{tabular}{|>{\centering\arraybackslash}p{0.3\textwidth}|>{\centering\arraybackslash}p{0.3\textwidth}|>{\centering\arraybackslash}p{0.3\textwidth}|}
\hline 
\bfseries {Valor} & \bfseries {Descripción} & \bfseries {Por defecto} \\ 
\hline 
Credit & Tarjeta de crédito & N \\ 
\hline 
Debit & Tarjeta de débito & N \\
\hline 
\end{tabular} 
\end{center}

\subsection{Correspondiente a brand}
\begin{center}
\begin{tabular}{|>{\centering\arraybackslash}p{0.3\textwidth}|>{\centering\arraybackslash}p{0.3\textwidth}|>{\centering\arraybackslash}p{0.3\textwidth}|}
\hline 
\bfseries {Valor} & \bfseries {Descripción} & \bfseries {Por defecto} \\ 
\hline 
Visa &  & N \\ 
\hline 
MasterCard &  & N \\
\hline 
\end{tabular} 
\end{center}

\subsection{Correspondiente a transType}
\begin{center}
\begin{tabular}{|>{\centering\arraybackslash}p{0.3\textwidth}|>{\centering\arraybackslash}p{0.3\textwidth}|>{\centering\arraybackslash}p{0.3\textwidth}|}
\hline 
\bfseries {Valor} & \bfseries {Descripción} & \bfseries {Por defecto} \\ 
\hline 
Withdrawal & Retiro & N \\ 
\hline 
Deposit & Deposito & N \\
\hline 
Transference & Transferencia & N \\
\hline
\end{tabular} 
\end{center}

\subsection{Correspondiente a acctType}
\begin{center}
\begin{tabular}{|>{\centering\arraybackslash}p{0.3\textwidth}|>{\centering\arraybackslash}p{0.3\textwidth}|>{\centering\arraybackslash}p{0.3\textwidth}|}
\hline 
\bfseries {Valor} & \bfseries {Descripción} & \bfseries {Por defecto} \\ 
\hline 
Saving & Ahorro & N \\ 
\hline 
Current & Corriente & N \\
\hline 
Loan & Préstamo & N \\
\hline
\end{tabular} 
\end{center}

\subsection{Correspondiente a contactInfo}
\begin{center}
\begin{tabular}{|>{\centering\arraybackslash}p{0.3\textwidth}|>{\centering\arraybackslash}p{0.3\textwidth}|>{\centering\arraybackslash}p{0.3\textwidth}|}
\hline 
\bfseries {Valor} & \bfseries {Descripción} & \bfseries {Por defecto} \\ 
\hline 
dayPhone & Teléfono de contacto durante el día & N \\
\hline 
evePhone & Teléfono de contacto durante la tarde & N \\
\hline 
dayFax & Fax de contacto durante el día & N \\
\hline
eveFax & Fax de contacto durante la tarde & N \\
\hline
emailAddr & Dirección de correo electrónica & N \\
\hline
\end{tabular} 
\end{center}
	
\section{Recursos}
El protocolo IFF esta basado en recursos. Los recursos son fuentes de información sobre las cuales se realizan las peticiones. Estos recursos son divididos en dos grandes grupos, los concretos y los abstractos.

\subsection{Recursos concretos}
Los recursos concretos son la representación directa del modelo de datos que expone el core bancario para ofrecer sus servicios. Este tipo de recursos es muy sencillo y son los que permiten realizar las operaciones más básicas. \\

A continuación se presentan los recursos concretos.

\subsubsection{Nombre de usuario ``login''}
Contiene la información que relaciona el nombre de usuario electrónico con su información  en la institución financiera.

\begin{center}
\begin{tabular}{|>{\centering\arraybackslash}p{0.2\textwidth}|>{\centering\arraybackslash}p{0.2\textwidth}|>{\centering\arraybackslash}p{0.2\textwidth}|>{\centering\arraybackslash}p{0.2\textwidth}|}
\hline 
\bfseries {Etiqueta} & \bfseries {Tipo} & \bfseries {Uso} & \bfseries {Descripción} \\ 
\hline 
username & Cadena de caracteres (6-20) & Requerido & ID de ingreso del cliente \\ 
\hline 
password & Cadena de caracteres (6+) & Opcional & Clave de ingreso \\ 
\hline 
custPermId & Cadena de caracteres (32+) & Requerido & ID permanente del cliente. Es asignado por la institución financiera para representar al cliente en el sistema \\ 
\hline 
\end{tabular}
\end{center}

\subsubsection{Cliente ``customer''}
Tiene la información que identifica inequívocamente a un cliente.

\begin{center}
\begin{longtable}{|>{\centering\arraybackslash}p{0.25\textwidth}|>{\centering\arraybackslash}p{0.2\textwidth}|>{\centering\arraybackslash}p{0.15\textwidth}|>{\centering\arraybackslash}p{0.2\textwidth}|}
\hline 
\bfseries {Etiqueta} & \bfseries {Tipo} & \bfseries {Uso} & \bfseries {Descripción} \\ 
\hline 
custPermId & Cadena de caracteres (32+) & Opcional & ID permanente del cliente. Es asignado por la institución financiera para representar al cliente en el sistema \\ 
\hline 
personId & Cadena de caracteres (32+) & Opcional & Relación al objeto ``person'' \\ 
\hline 
custLogin & Cadena de caracteres (6-20) & Opcional & ID permanente del cliente. Es asignado por la institución financiera para representar al cliente en el sistema \\ 
\hline 
personalIdent & Cadena de caracteres (8+) & Requerido & Identificación personal presentada por el cliente \\ 
\hline 
personVerifyType & Enumeración & Requerido & El tipo de documento con el cual se verifica la identidad del cliente \\ 
\hline
dtBCustomer & Fecha & Opcional & Momento en el cual la persona se vuelve cliente de la institución \\ 
\hline
dtLLogin & Fecha & Opcional & Último momento en el cual el cliente utiliza su cuenta \\ 
\hline
group & Cadena de caracteres (32+) & Opcional & Relaciona al cliente con el objeto ``grupo'' \\ 
\hline
\end{longtable}
\end{center}

\subsubsection{Información del banco ``bankInformation''}
Agrupa la información esencial de una agencia bancaria.

\begin{center}
\begin{longtable}{|>{\centering\arraybackslash}p{0.2\textwidth}|>{\centering\arraybackslash}p{0.2\textwidth}|>{\centering\arraybackslash}p{0.2\textwidth}|>{\centering\arraybackslash}p{0.2\textwidth}|}
\hline 
\bfseries {Etiqueta} & \bfseries {Tipo} & \bfseries {Uso} & \bfseries {Descripción} \\ 
\hline 
bankId & Cadena de caracteres (4+) & Opcional & ID que identifica a la agencia bancaria \\ 
\hline 
name & Cadena de caracteres & Opcional & Nombre de la agencia \\ 
\hline 
branchId & Cadena de caracteres & Opcional & ID que identifica a la sucursal \\ 
\hline 
branchName & Cadena de caracteres & Opcional & Nombre de la sucursal \\ 
\hline 
postAddr & Cadena de caracteres & Opcional & Dirección \\ 
\hline
city & Cadena de caracteres & Opcional & Ciudad \\ 
\hline
stateProv & Cadena de caracteres & Opcional & Estado o Provincia \\ 
\hline
postalCode &  Cadena de caracteres (4+) & Opcional & Código postal \\ 
\hline
country & Cadena de caracteres & Opcional & País \\ 
\hline
\end{longtable}
\end{center}

\subsubsection{Dirección ``address''}
Representa la dirección suministrada por el cliente.

\begin{center}
\begin{longtable}{|>{\centering\arraybackslash}p{0.2\textwidth}|>{\centering\arraybackslash}p{0.2\textwidth}|>{\centering\arraybackslash}p{0.2\textwidth}|>{\centering\arraybackslash}p{0.2\textwidth}|}
\hline 
\bfseries {Etiqueta} & \bfseries {Tipo} & \bfseries {Uso} & \bfseries {Descripción} \\ 
\hline 
addressId & Cadena de caracteres & Opcional & ID que identifica a la dirección \\ 
\hline 
custPermId & Cadena de caracteres (32+) & Requerido & ID permanente del cliente. Es asignado por la institución financiera para representar al cliente en el sistema \\
\hline 
nameAddrType & Enumeración & Requerido & Define el uso de la información suministrada \\ 
\hline 
addr & Cadena de caracteres & Opcional & Dirección \\ 
\hline
city & Cadena de caracteres & Opcional & Ciudad \\ 
\hline
stateProv & Cadena de caracteres & Opcional & Estado o Provincia \\ 
\hline
postalCode &  Cadena de caracteres (4+) & Opcional & Código postal \\ 
\hline
country & Cadena de caracteres & Opcional & País \\ 
\hline
addrType & Enumeración & Opcional & Define el tipo de dirección \\ 
\hline
startDt & Hora & Opcional & Hora de inicio \\ 
\hline
endDt & Hora & Opcional & Hora de fin \\ 
\hline
\end{longtable}
\end{center}

\subsubsection{Información de contacto ``contactInfo''}
Información suministrada por el cliente para poder ser contactado en caso de necesitarlo.

\begin{center}
\begin{longtable}{|>{\centering\arraybackslash}p{0.25\textwidth}|>{\centering\arraybackslash}p{0.2\textwidth}|>{\centering\arraybackslash}p{0.15\textwidth}|>{\centering\arraybackslash}p{0.2\textwidth}|}
\hline 
\bfseries {Etiqueta} & \bfseries {Tipo} & \bfseries {Uso} & \bfseries {Descripción} \\ 
\hline 
contactInfoId & Cadena de caracteres (32+) & Opcional & ID que identifica la información de contacto del cliente \\ 
\hline 
custPermId & Cadena de caracteres (32+) & Requerido & ID permanente del cliente. Es asignado por la institución financiera para representar al cliente en el sistema \\
\hline 
custContactPref & Enumeración & Requerido & Representa la manera en la cual el cliente será contactado \\ 
\hline 
prefTimeStart & Hora & Opcional & Hora a partir de la cual puede ser contactado \\ 
\hline
prefTimeEnd & Hora de caracteres & Opcional & Hora a partir de la cual ya no puede ser contactado \\ 
\hline
dayPhone & Cadena de caracteres & Opcional (ver descripción) & Teléfono de contacto durante el día. 
\\ & & & \\
& & & Este campo es requerido si ni ``evePhone'', ``dayFax'', ``eveFax'' o ``emailAddr'' es suministrado \\ 
\hline
evePhone & Cadena de caracteres & Opcional (ver descripción) & Teléfono de contacto durante la tarde. \\ & & & \\
& & & Este campo es requerido si ni ``dayPhone'', ``dayFax'', ``eveFax'' o ``emailAddr'' es suministrado \\ 
\hline
dayFax & Cadena de caracteres & Opcional (ver descripción) & Fax de contacto durante el día. \\ & & & \\
& & & Este campo es requerido si ni ``dayPhone'', ``evePhone'', ``eveFax'' o ``emailAddr'' es suministrado \\ 
\hline
eveFax & Cadena de caracteres & Opcional (ver descripción) & Fax de contacto durante la tarde. \\ & & & \\
& & & Este campo es requerido si ni ``dayPhone'', ``evePhone'', ``dayFax'' o ``emailAddr'' es suministrado \\ 
\hline
emailAddr & Cadena de caracteres & Opcional (ver descripción) & Correo electrónico de contacto. \\ & & & \\
& & & Este campo es requerido si ni ``dayPhone'', ``evePhone'', ``dayFax'' o ``eveFax'' es suministrado \\ 
\hline
\end{longtable}
\end{center}

\subsubsection{Información personal ``personalInfo''}
Contiene la información personal de un cliente.

\begin{center}
\begin{longtable}{|>{\centering\arraybackslash}p{0.2\textwidth}|>{\centering\arraybackslash}p{0.2\textwidth}|>{\centering\arraybackslash}p{0.2\textwidth}|>{\centering\arraybackslash}p{0.2\textwidth}|}
\hline 
\bfseries {Etiqueta} & \bfseries {Tipo} & \bfseries {Uso} & \bfseries {Descripción} \\ 
\hline 
personalInfoId & Cadena de caracteres (32+) & Opcional & ID que identifica a la información personal del cliente \\ 
\hline 
custPermId & Cadena de caracteres (32+) & Requerido & ID permanente del cliente. Es asignado por la institución financiera para representar al cliente en el sistema \\
\hline 
lastName & Cadena de caracteres & Requerido & Apellido del cliente \\ 
\hline 
firstName & Cadena de caracteres & Requerido & Dirección \\ 
\hline
middleName & Cadena de caracteres & Opcional & Ciudad \\ 
\hline
tittlePrefix & Cadena de caracteres & Opcional & Titulo por el cual llamar al cliente. Por ejemplo ``Dr.'' \\ 
\hline
nameSuffix & Cadena de caracteres & Opcional & Sufijo agregado al final del nombre del cliente. Por ejemplo ``Jr.'' \\ 
\hline
\end{longtable}
\end{center}

\subsubsection{Preferencia ``preference''}
Permite al cliente definir cierto comportamiento sobre su cuenta. El cliente puede establecer un monto predeterminado para concepto de retiro sobre una de sus cuentas. También, si se le ha hecho una transferencia al cliente y no se especificó cuenta destino, el dinero será transferido a la cuenta que el cliente haya definido por defecto.

\begin{center}
\begin{longtable}{|>{\centering\arraybackslash}p{0.3\textwidth}|>{\centering\arraybackslash}p{0.15\textwidth}|>{\centering\arraybackslash}p{0.15\textwidth}|>{\centering\arraybackslash}p{0.2\textwidth}|}
\hline 
\bfseries {Etiqueta} & \bfseries {Tipo} & \bfseries {Uso} & \bfseries {Descripción} \\ 
\hline 
preferenceId & Cadena de caracteres (32+) & Opcional & ID que identifica a la información de preferencia del cliente \\ 
\hline 
custPermId & Cadena de caracteres (32+) & Requerido & ID permanente del cliente. Es asignado por la institución financiera para representar al cliente en el sistema \\
\hline 
acctId & Cadena de caracteres (32+) & Opcional & ID de la cuenta a la cual se le aplicaran los consumos por concepto de retiros predefinidos \\ 
\hline 
defaultTranfAccount & Cadena de caracteres (32+) & Opcional & ID de la cuenta a la cual se le aplicaran las transferencias sin cuenta de destino especificada \\ 
\hline
withdrawalAmt & Cadena de caracteres (32+) & Opcional (ver descripción) & Monto de retiro por defecto. 
\\ & & & \\
& & & Este campo es requerido si ``acctId'' es especificado \\
\hline
\end{longtable}
\end{center}

\subsubsection{Transferencias a terceros ``registeredRecipient''}
Contiene los datos de alguna cuenta o tarjeta de otro banco junto con la identificación de sus acreedores.

\begin{center}
\begin{longtable}{|>{\centering\arraybackslash}p{0.25\textwidth}|>{\centering\arraybackslash}p{0.2\textwidth}|>{\centering\arraybackslash}p{0.15\textwidth}|>{\centering\arraybackslash}p{0.2\textwidth}|}
\hline 
\bfseries {Etiqueta} & \bfseries {Tipo} & \bfseries {Uso} & \bfseries {Descripción} \\ 
\hline 
recipientId & Cadena de caracteres (32+) & Opcional & ID que identifica la información acerca de un cliente en otra institución financiera \\ 
\hline 
custPermId & Cadena de caracteres (32+) & Requerido & ID permanente del cliente. Es asignado por la institución financiera para representar al cliente en el sistema \\
\hline 
personId & Cadena de caracteres (32+) & Opcional & ID que identifica al objeto ``person'' \\ 
\hline 
acctNum & Cadena de caracteres & Opcional (ver descripción) & representa el número de cuenta en alguna otra institución financiera. 
\\ & & & \\
& & & Este campo es requerido si ``cardSeqNum'' no es especificado \\ 
\hline
cardSeqNum & Cadena de caracteres & Opcional (ver descripción) & representa el número de tarjeta de alguna otra institución financiera. 
\\ & & & \\
& & & Este campo es requerido si ``acctNum'' no es especificado \\ 
\hline
name & Cadena de caracteres & Requerido & Nombre del beneficiario \\ 
\hline
desc & Cadena de caracteres & Requerido & Descripción \\ 
\hline
maxAmtLimit & Cadena de caracteres & Opcional & Máximo monto permitido para realizar la transferencia \\ 
\hline
personalIdent & Cadena de caracteres (8+) & Requerido & Identificación personal presentada por el cliente \\ 
\hline
personVerifyType & Enumeración & Requerido & El tipo de documento con el cual se verifica la identidad del cliente \\ 
\hline
\end{longtable}
\end{center}

\subsubsection{Persona ``person''}
Contiene los datos que identifican a los clientes como personas. También reúne los
datos de los clientes que tienen cuentas en otros bancos, estos datos provienen de
``registeredRecipient''.

\begin{center}
\begin{longtable}{|>{\centering\arraybackslash}p{0.3\textwidth}|>{\centering\arraybackslash}p{0.15\textwidth}|>{\centering\arraybackslash}p{0.15\textwidth}|>{\centering\arraybackslash}p{0.2\textwidth}|}
\hline 
\bfseries {Etiqueta} & \bfseries {Tipo} & \bfseries {Uso} & \bfseries {Descripción} \\ 
\hline 
personId & Cadena de caracteres (32+) & Opcional & ID que identifica a la información de una persona \\ 
\hline 
name & Cadena de caracteres & Requerido & Nombre \\
\hline 
\end{longtable}
\end{center}

\subsubsection{Conocido ``known''}
Contiene la información de las personas conocidas. De esta forma se puede pueden
realizar transferencias a personas en lugar de a cuentas.

\begin{center}
\begin{longtable}{|>{\centering\arraybackslash}p{0.3\textwidth}|>{\centering\arraybackslash}p{0.15\textwidth}|>{\centering\arraybackslash}p{0.15\textwidth}|>{\centering\arraybackslash}p{0.2\textwidth}|}
\hline 
\bfseries {Etiqueta} & \bfseries {Tipo} & \bfseries {Uso} & \bfseries {Descripción} \\ 
\hline 
knownId & Cadena de caracteres (32+) & Opcional & ID que identifica a la información acerca de un conocido	 \\ 
\hline 
personId & Cadena de caracteres (32+) & Requerido & ID que identifica la información acerca de una persona.
\\ & & & \\
& & & En este caso representa a un conocido \\
\hline 
custPermId & Cadena de caracteres (32) & Requerido & ID permanente del cliente. Es asignado por la institución financiera para representar al cliente en el sistema.
\\ & & & \\
& & & En este caso representa al conocedor \\
\hline 
relationship & Cadena de caracteres & Requerido & Describe el tipo de relación entre el conocedor y el conocido. \\ 
\hline 
status & Booleano & Opcional & Representa si se ha validado que estas dos personas se conocen \\ 
\hline 
\end{longtable}
\end{center}

\subsubsection{Miembro de un grupo ``groupMember''}
Un cliente tiene la capacidad de crear grupo de personas conocidas. De esta forma puede establecer en que cuenta serán ubicados los fondos recibidos por parte de algún miembro del grupo. Un miembro del grupo es aquel cliente que pertenezca a un grupo.

\begin{center}
\begin{longtable}{|>{\centering\arraybackslash}p{0.25\textwidth}|>{\centering\arraybackslash}p{0.2\textwidth}|>{\centering\arraybackslash}p{0.15\textwidth}|>{\centering\arraybackslash}p{0.2\textwidth}|}
\hline 
\bfseries {Etiqueta} & \bfseries {Tipo} & \bfseries {Uso} & \bfseries {Descripción} \\ 
\hline 
groupMemberId & Cadena de caracteres (32+) & Opcional & ID que identifica al objeto ``groupMember'' \\ 
\hline 
custPermId & Cadena de caracteres (32+) & Requerido & ID permanente del cliente. Es asignado por la institución financiera para representar al cliente en el sistema \\
\hline 
member & Cadena de caracteres (32+) & Requerido & Representa al cliente miembro del grupo \\
\hline 
groupId & Cadena de caracteres (32+) & Requerido & Identifica al grupo al cual pertenece un miembro de grupo\\
\hline 
\end{longtable}
\end{center}

\subsubsection{Grupo ``group''}
Es una unidad en la cual un cliente puede agrupar a otros clientes del banco y
predefinir una cuenta en la cual los miembros al grupo transferirán.

\begin{center}
\begin{longtable}{|>{\centering\arraybackslash}p{0.2\textwidth}|>{\centering\arraybackslash}p{0.2\textwidth}|>{\centering\arraybackslash}p{0.2\textwidth}|>{\centering\arraybackslash}p{0.2\textwidth}|}
\hline 
\bfseries {Etiqueta} & \bfseries {Tipo} & \bfseries {Uso} & \bfseries {Descripción} \\ 
\hline 
groupId & Cadena de caracteres (32+) & Opcional & ID que identifica al objeto ``group'' \\ 
\hline
acctId & Cadena de caracteres (32+) & Opcional & ID de la cuenta a la cual se le aplicaran los consumos por concepto de retiros predefinidos \\ 
\hline 
name & Cadena de caracteres & Requerido & Nombre del grupo \\
\hline 
descripción & Cadena de caracteres & Opcional & Descripción del grupo \\
\hline 
\end{longtable}
\end{center}

\subsubsection{Estado de la cuenta ``accountStatus''}
Tiene la información del estado en el cual se encuentra la cuenta.

\begin{center}
\begin{longtable}{|>{\centering\arraybackslash}p{0.25\textwidth}|>{\centering\arraybackslash}p{0.2\textwidth}|>{\centering\arraybackslash}p{0.15\textwidth}|>{\centering\arraybackslash}p{0.2\textwidth}|}
\hline 
\bfseries {Etiqueta} & \bfseries {Tipo} & \bfseries {Uso} & \bfseries {Descripción} \\ 
\hline 
accountStatusId & Cadena de caracteres (32+) & Opcional & ID que identifica al objeto ``accountStatus'' \\ 
\hline
acctId & Cadena de caracteres (32+) & Opcional & ID de la cuenta a la cual se le aplicaran los consumos por concepto de retiros predefinidos \\ 
\hline 
accountStatusCode & Enumeración & Requerido & Representa el estado de la cuenta \\
\hline 
effDt & Fecha & Opcional & Fecha en la cual se hizo efectivo dicho estado \\
\hline 
statusModBy & Cadena de caracteres & Opcional & Tiene la información acerca de quien modificó el estado \\
\hline 
statusDesc & Cadena de caracteres & Opcional & Descripción sobre el estado \\
\hline 
\end{longtable}
\end{center}

\subsubsection{Estado de la tarjeta ``cardStatus''}
Contiene la información del estado en el cual se encuentra la tarjeta.

\begin{center}
\begin{longtable}{|>{\centering\arraybackslash}p{0.25\textwidth}|>{\centering\arraybackslash}p{0.2\textwidth}|>{\centering\arraybackslash}p{0.15\textwidth}|>{\centering\arraybackslash}p{0.2\textwidth}|}
\hline 
\bfseries {Etiqueta} & \bfseries {Tipo} & \bfseries {Uso} & \bfseries {Descripción} \\ 
\hline 
card	StatusId & Cadena de caracteres (32+) & Opcional & ID que identifica al objeto ``cardStatus'' \\ 
\hline
cardEmBossNum & Cadena de caracteres (32+) & Requerido & Número de la tarjeta a la cual pertenece el estado \\ 
\hline 
cardStatusCode & Enumeración & Requerido & Representa el estado de la tarjeta \\
\hline 
effDt & Fecha & Opcional & Fecha en la cual se hizo efectivo dicho estado \\
\hline 
statusModBy & Cadena de caracteres & Opcional & Tiene la información acerca de quien modificó el estado \\
\hline 
statusDesc & Cadena de caracteres & Opcional & Descripción sobre el estado \\
\hline 
\end{longtable}
\end{center}

\subsubsection{Tarjeta ``card''}
Contiene la información de la tarjeta.

\begin{center}
\begin{longtable}{|>{\centering\arraybackslash}p{0.25\textwidth}|>{\centering\arraybackslash}p{0.2\textwidth}|>{\centering\arraybackslash}p{0.15\textwidth}|>{\centering\arraybackslash}p{0.2\textwidth}|}
\hline 
\bfseries {Etiqueta} & \bfseries {Tipo} & \bfseries {Uso} & \bfseries {Descripción} \\ 
\hline 
cardEmBossNum & Cadena de caracteres (32+) & Requerido & Número de la tarjeta \\ 
\hline
acctId & Cadena de caracteres (32+) & Requerido & ID de la cuenta a la cual se le aplicaran los consumos por concepto de retiros predefinidos \\ 
\hline
cardType & Enumeración & Opcional & Tipo de tarjeta \\
\hline 
brand & Enumeración & Opcional & Consorcio al que pertenece la tarjeta \\
\hline 
issuerName & Cadena de caracteres & Opcional & Nombre del tarjetahabiente \\
\hline 
issDt & Fecha & Opcional & Fecha en la cual se emite la tarjeta \\
\hline 
expDt & Fecha & Opcional & Fecha en la cual expira la tarjeta \\
\hline 
\end{longtable}
\end{center}

\subsubsection{Balance ``balance''}
Contiene la información del dinero existente en una cuenta y las transacciones que
la han afectado.

\begin{center}
\begin{longtable}{|>{\centering\arraybackslash}p{0.2\textwidth}|>{\centering\arraybackslash}p{0.2\textwidth}|>{\centering\arraybackslash}p{0.2\textwidth}|>{\centering\arraybackslash}p{0.2\textwidth}|}
\hline 
\bfseries {Etiqueta} & \bfseries {Tipo} & \bfseries {Uso} & \bfseries {Descripción} \\ 
\hline 
acctId & Cadena de caracteres (32+) & Requerido & ID de la cuenta a la cual pertenece el balance \\ 
\hline
transId & Entero (8) & Opcional & ID de la transacción que afectó el balance \\
\hline 
curAmt & Decimal & Requerido & Cantidad de dinero en la cuenta para la fecha \\
\hline 
effDt & Fecha & requerido & Fecha en la cual se afectó el balance \\
\hline 
descr & Cadena de caracteres & Opcional & Descripción \\
\hline 
\end{longtable}
\end{center}

\subsubsection{Transacción ``transaction''}
Cualquier movimiento que afecte algún balance.

\begin{center}
\begin{longtable}{|>{\centering\arraybackslash}p{0.2\textwidth}|>{\centering\arraybackslash}p{0.2\textwidth}|>{\centering\arraybackslash}p{0.2\textwidth}|>{\centering\arraybackslash}p{0.2\textwidth}|}
\hline 
\bfseries {Etiqueta} & \bfseries {Tipo} & \bfseries {Uso} & \bfseries {Descripción} \\ 
\hline
transId & Entero (8) & Opcional & ID de la transacción \\
\hline 
acctId & Cadena de caracteres (32+) & Requerido & ID de la cuenta que realizó la transacción \\ 
\hline 
acctOutFlow & Cadena de caracteres (32+) & Opcional (ver descripción) & ID de la cuenta a la cual se le debitará el dinero.
\\ & & & \\
& & & Este campo es requerido en caso de que la transacción debite de alguna forma dinero de la cuenta \\
\hline 
acctInFlow & Cadena de caracteres (32+) & Opcional (ver descripción) & ID de la cuenta que recibirá el dinero.
\\ & & & \\
& & & Este campo es requerido en caso de que la transacción abone de alguna forma dinero a la cuenta \\
\hline
thirdParty & Cadena de caracteres & Opcional (ver descripción) & ID de la cuenta o tarjeta de un beneficiario en otro banco.
\\ & & & \\
& & & Este campo es requerido en caso de existir un tercero involucrado \\
\hline
amt & Decimal & Requerido & Cantidad de dinero que moverá la transacción \\
\hline 
dueDt & Fecha & Requerido & Fecha en la cual se desea la transacción sea efectuada \\
\hline 
curDt & Fecha & Requerido & Fecha en la cual se realizó la transacción \\
\hline 
transType & Enumeración & Opcional & Tipo de transacción realizada \\
\hline 
\end{longtable}
\end{center}

\subsubsection{Transacción ``transaction''}
Cualquier movimiento que afecte algún balance.

\begin{center}
\begin{longtable}{|>{\centering\arraybackslash}p{0.2\textwidth}|>{\centering\arraybackslash}p{0.2\textwidth}|>{\centering\arraybackslash}p{0.2\textwidth}|>{\centering\arraybackslash}p{0.2\textwidth}|}
\hline 
\bfseries {Etiqueta} & \bfseries {Tipo} & \bfseries {Uso} & \bfseries {Descripción} \\ 
\hline
acctId & Cadena de caracteres (32+) & Requerido & ID de la cuenta que realizó la transacción \\ 
\hline 
bankId & Cadena de caracteres & Requerido & ID que identifica a la agencia bancaria \\
\hline 
custPermId & Cadena de caracteres (32+) & Requerido & ID permanente del cliente. Es asignado por la institución financiera para representar al cliente en el sistema 
\\ & & & \\
& & & Este campo es requerido si la cuenta no es compartida \\
\hline
acctType & Enumeración & Requerido & Define de qué tipo es la cuenta \\
\hline
freeForAll & Booleano & Opcional (ver descripción) & En caso de que la cuenta sea compartida, define si cualquiera puede efectuar operaciones o se necesita la aprobación de todos los participantes.
\\ & & & \\
& & & Es requerido si el campo ``custPermId'' no está definido \\
\hline 
members & Entero & Opcional (ver descripción) & Define cuantos clientes comparten la cuenta
\\ & & & \\
& & & El campo es requerido si ``freeForAll'' está definido \\
\hline 
\end{longtable}
\end{center}

\subsection{Recursos abstractos}
A diferencia de los recursos concretos, los recursos abstractos no representan objetos en la base de datos. Estos recursos ofrecen una información más general y son producto de una recopilación de información sobre varios objetos en la base de datos. También se pueden utilizar estos recursos para ofrecer una forma más sencilla de actualizar alguna información en el sistema. Por ejemplo, si se desea registrar un nuevo cliente, se tendrían que mandar varias peticiones para crear los recursos necesarios para que el cliente sea registrado. Para evitar mandar varias peticiones se podría ofrecer un recurso que recopile toda la información necesaria y este se encargue de crear los recursos concretos necesarios para que el nuevo cliente pueda ser registrado.
%\end{document}
%\input{apendices/Ejemplos_del_lenguaje.tex}
%\input{apendices/Gramaticas.tex}

%\printglossary

\end{document}
